\documentclass[12pt]{article}
%% \usepackage{color} enabled in preamble
\usepackage[normalem]{ulem}  %% enable \st{} strikethrough

% This version hides the annotations so that the rendered PDF does not highlight the changes made in this revision
\input{preamble_clean}
%%\usepackage{fontspec}
%%\setmainfont{Times New Roman}
\usepackage{mathptmx}
\newcounter{proto}

% use 'rurl' to hide the 'http://' in text, but include it in the link
\newcommand\rurl[1]{%
	\href{http://#1}{\nolinkurl{#1}}%
}

% use 'button' to define a standard way to display clickable buttons
\newcommand\button[1]{%
	\textbf{#1}%
}

% use 'motif' to define a standard way to display motifs
\newcommand\motif[1]{%
    \textsc{\lowercase{#1}}%
}

% use 'uniprot' to define a standard way to display uniprot names
% TODO: HD/MG check if textsc/lowercase is the best way to represent this
\newcommand\uniprot[1]{%
	\textsc{\lowercase{#1}}%
}

% use 'fileformat' to define a standard way to display file formats
\newcommand\fileformat[1]{%
    \textit{#1}{}%
}%

\begin{document}


% Frontmatter

\thispagestyle{plain}
\begin{center}

\textbf{\LARGE \textit{Current Protocols in Bioinformatics}}\\\vspace{2mm}
\end{center}

\section*{Exploring Short Linear Motifs using the ELM Database and Tools}

{\small \it
Marc Gouw$^{1}$,
Hugo Samano$^{1}$,
Kim Van Roey$^{1}$,
Francesca Diella$^{1}$,
Toby J. Gibson$^{1}$,
Holger Dinkel$^{1,2}$

$^{1}$ Structural and Computational Biology Unit, European Molecular Biology
Laboratory, Meyerhofstrasse 1, 69117 Heidelberg, Germany\\
$^{2}$ Leibniz-Institute on Aging -- Fritz Lipmann Institute (FLI),
Beutenbergstrasse 11, 07745 Jena, Germany
}

\section*{Keywords}

Linear motifs, Bioinformatics, Protein-Protein Interaction, Molecular
switches, Cell regulation

\section*{Significance Statement}

TODO: This Significance Statement could also use some suggestions :) (Max 120
words). - Marc 

Short Linear motifs (SLiMs) are small protein-interaction modules that play
essential roles in signalling, cell cycle, trafficking and disease.
SLiMs are short, degenerate sequences with no stable
secondary structure, and have proven difficult to study both experimentally
and computationally.
The number of different motifs is believed to be in the order of millions.
Yet, despite their abundance and importance, we have a limited
understanding of which motifs exist and their function.
The Eukaryotic Linear Motif (ELM) resource is an online database containing
over 3000 manually curated SLiMs from experimental
literature, and a pipeline to detect SLiMs in protein sequences.
ELM is a valuable resource to the scientific community studying protein
function.

% \emph{instructions: Provide a 120-word-maximum statement about the
% significance of the protocols/topic described in your manuscript. This
% should be understandable to undergraduate- educated scientists outside
% their field of specialty. The goal is to explain the relevance of the
% work in broad context to a broad readership. It will be used in
% promotion of the article following publication.}

\begin{abstract}%
%
% \emph{instructions: brief overview, no references, max 150 words}
%

\textbf{TODO:} this abstract could still use some inspiration.
	suggestions welcome! (max 150 words) - Marc

The Eukaryotic Linear Motif (ELM) resource (\rurl{elm.eu.org}) dedicated to the
characterization and prediction of short linear motifs (SLiMs).
SLiMs are small degenerate peptide segments found in all proteins and
essential to all cellular processes which, despite their abundance, still remain
largely uncharacterized.

The ELM database is a collection of manually annotated SLiM instances curated
from experimental literature.

In this article we explore how to browse and search the database curated SLiM
data, and cover the different types of data integrated in the resource. 

We also cover how to use this resource in order to predict SLiMs motifs in
known as well as unknown proteins, and how to interpret the
results generated by the ELM prediction pipeline.

The ELM database is a very rich resource, and in the following
protocols we give helpful examples of how this knowledge can be used to
improve your own research.

\end{abstract}



% Introduction
\section*{Introduction}%
\label{sec:introduction}

The activity and function of a protein is tightly regulated by its
cellular environment. To interact with their surroundings, proteins use
various types of binding modules that each display distinct binding
properties \citep{10550212}. One prominent type of binding module
consists of short linear motifs (SLiMs) \citep{18508681}. These compact
binding sites are generally located in intrinsically disordered regions
(IDR) of the proteome and commonly bind to
surfaces of a globular domain of a protein \citep{21909575}. SLiMs mediate
different types of interactions that regulate protein functionality, and hence
are important regulators of the dynamic processes involved in cell
signalling \citep{22480932} \citep{24926813}. The number of
SLiM instances in the human proteome is currently suggested to be over
one million \citep{25038412}. Identifying SLiMs and elucidating their
functionality is an essential step in understanding cell regulation. The
Eukaryotic Linear Motif (ELM) resource contributes to this process by
providing the necessary tools to researchers working on motifs. It
consists of a database and a prediction tool. The database provides a
categorised repository of experimentally validated linear motif classes
and instances that were manually annotated from the literature. The ELM
prediction tool in turn relies on annotated data, both from the ELM
database and other resources, to accurately analyse unknown sequences
for candidate motifs and assist researchers in selecting the most
plausible ones for experimental validation and discard likely false
positive hits, saving them valuable time and resources \citep{22110040}.
The following protocols will guide users through the different ELM
applications, explaining how to browse the curated data available in
ELM, how to analyse a protein sequence for putative motifs, and how to
interpret these data and avoid common pitfalls in SLiM discovery.


% Browsing and exploring content
	\section{Exploring the Content of the ELM Database}%
	\label{sec:explore_content}%

	% TODO: this paragraph is technically not part of the protocol but rather introduction
	% I think this part is OK here: It introduces the annotated part of the database.
	% Also, I think we can 'assume' that most Papers will be read from top to bottom?
	The core of the ELM database is a repository of manually annotated motifs and
	instances. As of January 2017, ELM contains over 260 motif classes and over
	3000 experimentally validated and manually curated instances.
	The motif classes and motif instances have been uploaded by a large group of
	annotators from around the globe. The complete catalogue of manually curated
	data can be searched, browsed and explored on the ELM website.

	% TODO: this paragraph is technically not part of the protocol but rather introduction
	% This part: agreed, if we can find a way to put it in the introduction.
	% Although I don't think its too bad here :)
	Each motif class describes a short linear motif, a short sequence of amino acids
	with a dedicated function. Since the motifs are often degenerate, each motif
	class is represented using ``regular expressions'': a symbolic representation
	expressing a complex pattern of letters (or amino acids). For example, the
	regular expression ``\motif{[FY].L.P}'' is to be read as: The first amino acid
	is a Phenylalaline '\motif{F}' or a Tyrosine '\motif{Y}' followed by any one
	amino acid '\motif{.}', a Leucine '\motif{L}', one more arbitrary amino acid
	'\motif{.}' and finally a Proline '\motif{P}'.

	% TODO: this paragraph is technically not part of the protocol but rather introduction
	% I also think this part is a nice part for the introduction to this section
	% (annotated data)
	In all cases the annotator condenses information from the primary and secondary
	literature into manageable abstracts, accompanies the motif definition with
	a list of experimental instances, as well as links to external resources including
	biological pathways, diseases, Gene Ontology and other protein resources. In
	this protocol we explore the several data-types stored in the database, as well
	as links to external resources.

	%
	% Subsection: Necessary Resources
	%
	\subsection*{Necessary Resources}
	\subsubsection*{Software \& Hardware}
	A modern browser such as Firefox, Chrome, or Safari. ELM is best viewed
on a laptop or desktop computer, although tablets and smartphones will
also work.




	%
	% Subsection: Database content overview
	%
	\subsection*{Database content overview}%
	\label{subsec:explore_content_database}%


	% Counter for steps
	\begin{enumerate}%

	%TODO: Improve/expand all captions

	\begin{figure}[h!]%
		\centering%
		\includegraphics[width=\textwidth]{Figures/explore_content/home.png} %
		\caption{%
			The homepage of the ELM resource (\rurl{elm.eu.org}).%
			\textbf{TODO} A bigger caption for this figure. -Marc
		}%
		\label{fig:explore_content_home}
	\end{figure}

	\item The ELM database is an online web resource. Open a browser and navigate
		to \rurl{elm.eu.org} to visit the homepage
		(Figure~\ref{fig:explore_content_home}).
		This page shows a brief explanation of the ELM resource, and a form to
		search for SLiMs (which we cover in further detail in
		\ref{sec:predicting_p53} and \ref{sec:predicting_cv_0974}).
		The column to the right is continually
		updated with the latest news about changes and additions to the database.

	\begin{figure}[h!]
		\centering
		\includegraphics[width=\textwidth]{Figures/explore_content/stats.png}
		\caption{
			The ELM database statistics overview page shows the most up-to-date
			database statistics. As of January 2017, ELM has just over
			3000 annotated instances in 262 different motif classes.
		}
		\label{fig:explore_content_stats}
	\end{figure}

	\item On the ELM homepage click on the menu link \button{ELM DB} for an overview of
		the database statistics (Figure~\ref{fig:explore_content_stats}).
		This page displays the types and amounts of annotations contained in
		the database and a few links to third-part databases.
		Each line contains at least one link that will take you
		to the corresponding contents page.
		For example: Clicking on \button{ELM classes} will take you to the
		page showing all classes annotated in ELM.
		We will be exploring these content overview pages in this protocol.
	% HD:I find the following sentence confusing, as the user doesn't know whether he should click on that 'instances' button or not... will be explained later anyways
	%	(for example, clicking on
	%	\button{ELM instances} will take you to the page displaying all of the
	%	annotated instances in the database).
	%
	% Marc: I completley agree. I added some things which I hope makes it clearer.

	%
	% Subsection: Browsing motif classes and instances
	%
	\subsection*{Browsing motif classes and annotated instances}
	\label{subsec:explore_content_classes_and_instances}

	\begin{figure}[h!]
		\centering
		\includegraphics[width=\textwidth]{Figures/explore_content/elms.png}
		\caption{
			The list of all motif classes annotated in the ELM database.
			\textbf{TODO} A better (bigger) caption would be nice. -Marc
		}
		\label{fig:explore_content_elms}
	\end{figure}


	\item Click on the sub-menu \button{ELM classes} under \button{ELM DB} to visit
		the page listing all of the ELM classes
		(Figure~\ref{fig:explore_content_elms}).
		For each class, the following information is provided: ELM identifier,
		short description, regular expression, number of instances annotated
		for each class, and number of structures available. For details on each
		class, click on the ELM identifier; to get a list of annotated
		instances for an individual class, click on the number of instances.

		\sdesc{Use the search bar at the top of the page to filter for certain
			motif classes. For example, typing ``MAPK'' and hitting \button{submit}
			will perform a full-text search on all motif classes in the ELM
			database containing the term ``MAPK''. The green buttons on the
			left can also be used to filter this table. For example,
			toggling the ``DOC'' button will remove all DOC classes
			from the table (and clicking it again will bring them back).
			Lastly, the yellow \button{tsv} link can be used to export all
			motif classes as a ``tab separated values'' file.}

	\begin{figure}[h!]
		\centering
		\includegraphics[width=\textwidth]{Figures/explore_content/doc_cyclin_1_class.png}
		\caption{
		The motif details page for \motif{DOC\_CYCLIN\_1}. This page
		contains all of the manual annotation details for the
	    \motif{DOC\_CYCLIN\_1} motif, the biological background summarized from
		the scientific literature including links to the primary
		literature and to external resources (Pubmed \citep{27899561},
		the Gene Ontology \citep{27899567}, PDB \citep{10592235} and
		more).
		}
		\label{fig:explore_content_doc_cyclin}
	\end{figure}

	\item Search the table for the term \motif{DOC\_CYCLIN\_1} and click on
		\button{DOC\_CYCLIN\_1} in the left column to
		navigate to the page with details about the
		\motif{DOC\_CYCLIN\_1} motif class
		(Figure~\ref{fig:explore_content_doc_cyclin}).
		This page contains a description of the
		functional site class (a Cyclin recognition site), and a short
		description of the ELM and its regular expression, as well as a
		probability score, the taxonomic distribution of the motif and which
		domain (if any) is responsible for the interaction.

		\sdesc{The probability score is the probability that the regular
			expression represents a random selection of amino acids
			(similar to an information content score). A lower score
			indicates that the motif pattern is more difficult to find by
			chance in a random sequence.}

	% TODO: Could explain more of the content, eg. Probability, interaction domain etc.
	% TODO: be more consistent in using bold, italics, inverted-commas - it should be clear what are clickable links
	\item Scroll further down the \motif{DOC\_CYCLIN\_1} page
		(Figure~\ref{fig:explore_content_doc_cyclin}) to view
		more details about this motif
		(Figure~\ref{fig:explore_content_doc_cyclin_1_abstract_instances}).
	    The ``abstract'' contains a description of the biological relevance of the
	    motif (for example its involvement in cellular processes and pathways).
		annotation. Click on the \button{show} button next to the ``selected
		references'' header for a list of publications relevant to this motif.
		Click on \button{show} next to ``GO terms'' for a complete list of all
		Gene Ontology (GO) terms annotated for this motif.

	\begin{figure}[h!]
		\centering
		\includegraphics[width=\textwidth]{Figures/explore_content/doc_cyclin_1_abstract_instances.png}
		\caption{
			The second part of the \motif{DOC\_CYCLIN\_1} motif details page
			shows the motif abstract, selected references, GO terms, and the list of annotated
			instances.
		}
		\label{fig:explore_content_doc_cyclin_1_abstract_instances}
	\end{figure}

	\item Scroll further down the \motif{DOC\_CYCLIN\_1} page to view
		the ``Instances'' header
		(Figure~\ref{fig:explore_content_doc_cyclin_1_abstract_instances})
		This table contains the list of all annotated \motif{DOC\_CYCLIN\_1}
		instances in the database for this motif. This includes the protein
		identifier, the start and end positions of the instance, the specific
		sequence matching the regular expression representing the motif and
		the ``logic'' of the instance.
		The ``\#~Ev.'' indicates the number of experimental evidences
		associated with the annotation. ``Organism'' indicates the 
		species in which the protein is found. Lastly the ``Notes'' column
		contains links to any ``interactions'' or ``switches'' present in the
		database, as well as links to PDB, if the structure exists in the PDB.

		\sdesc{The instance ``logic'' is an annotation of whether 
			-- at the time the entry as prepared -- this is a
			\textit{bona-fide} instance, or whether it is a non-functional
			instance. \textit{TP} (True positive) indicates the instance is
			annotated with experimental evidence showing, that it is functional.
			\textit{FP} (False Positive) instances have experimental
			evidence suggesting function, but are believed to be
			non-functional, after careful examination by our annotators.
			\textit{TN} (True Negative) instances have been experimentally
			determined to be non-functional, and \textit{U} (Unknown)
			instances do not have enough evidence to determine whether it
			is functional or not. The overwhelming majority of instances in
			ELM are \textit{TP}s.}


	\begin{figure}[h!]
		\centering
		\includegraphics[width=\textwidth]{Figures/explore_content/instances.png}
		\caption{
			The ``instances'' page is used to search for instances in
			the ELM database.
			\textbf{TODO} Bigger caption for this figure. -Marc
		}
		\label{fig:explore_content_instances}
	\end{figure}

	\item \label{sec:explore_content_instances} Click on the sub-menu \button{ELM instances} under \button{ELM DB} to visit
		the page where you can search and browse the instances annotated in ELM
		(Figure~\ref{fig:explore_content_instances}).
		Note that only the first hundred instances matching the search criteria are shown.
		The search form can be used to filter results by a full text search, by
		instance logic, or by organism.

	% TODO: explain GFF, PIR, FASTA?
		\sdesc{This table can be filtered by motif class using the green toggle
			filters on the left hand side. Lastly, the yellow buttons at
			the top of the page can be used to download the instances in
			the following formats: \fileformat{GFF}, \fileformat{PIR},
			\fileformat{FASTA} or \fileformat{TSV}.}

	\begin{figure}[h!]
		\centering
		\includegraphics[width=\textwidth]{Figures/explore_content/doc_cyclin_1_instance.png}
		\caption{
		The instance details page for the \motif{DOC\_CYCLIN\_1} instance annotated
		for protein \uniprot{P53\_HUMAN} with start/end position ``381--385''. This page
		also contains links to many external databases including UniProt
		\citep{25348405}, PDB \citep{12037327}, NCBI taxonomy and Pubmed
		\citep{27899561}, and KEGG pathways \citep{26476454}, as well as the
		PSI-MI controlled vocabulary \citep{17925023}.
		}
		\label{fig:explore_content_doc_cyclin_instance}
	\end{figure}

	\item Type ``p53\_human'' in the search box to search for ELM Instances in this
		protein. Find the row for the ELM class \motif{DOC\_CYCLIN\_1} and click on
		the instance sub-sequence (highlighted in red) to go to the instance
		details page of this
		instance (Figure~\ref{fig:explore_content_doc_cyclin_instance}).
		The top part of the page contains details about the instance
		and the protein it was identified in, as well as a link to the UniProt entry for
		the protein \citep{25348405}.

	\item Scroll down to the ``Instance Evidence'' header to view details on the
		experimental evidence used to annotate this instance. Each
		experimental method is annotated using the Proteomics Standards
		Initiative Method Identifier (PSI-MI) \citep{17925023}, as well as the
		references in which the experiments were published.

		\sdesc{
			The ``biosource'' indicates whether method is \emph{in vivo},
			\emph{in vitro}, \emph{in silico} or a combination of these.
			The ``logic'' column indicates whether this experiment
			supports or contradicts this instance being functional.
			Each method is also annotated with a reliability assessment, which can
			be any of certain, likely, unlikely or unspecified.}

	%
	% Subsection: Switches, pathways and other external resources.
	%
	\subsection*{Finding Switches and molecular interactions}
	\label{subsec:explore_content_external_resoureces}

	\begin{figure}[h!]
		\centering
		\includegraphics[width=\textwidth]{Figures/explore_content/doc_ww_pin_1_4_instance.png}
		\caption{
		The instance details page for the \motif{DOC\_WW\_Pin1\_4}
		instance found in human p53 (\uniprot{P53\_HUMAN}) with start/end position
		``30--35''.
		}
		\label{fig:explore_content_doc_ww_instance}
	\end{figure}

	\item Repeat the previous search by clicking on the sub-menu \button{ELM instances}
		under \button{ELM DB} and type ``p53\_human'' in the search box. This time,
		find the ELM instance of the motif \motif{DOC\_WW\_Pin1\_4} with the
		start/end position ``30--35''. (You can sort the table by clicking on the
		header lines: click on ``Start'' to sort by start position). Click on the
		start/end position or the sub-sequence that will take you to the details
		page (Figure~\ref{fig:explore_content_doc_ww_instance}). This
		page is similar to that described for the p53 instance \motif{DOC\_CYCLIN\_1}
		(Figure~\ref{fig:explore_content_doc_cyclin_instance}).
		Additionally, for this instance, there is information available about
		its interaction partner and a molecular switch, which is mediated by
		this motif instance.

	\item Scroll down to the ``Interactions'' header to view information about this
		instance's interactions
		(Figure~\ref{fig:explore_content_doc_ww_instance}). This instance
		interacts with \uniprot{PIN1\_Human} via the ``WW'' domain (PFAM identifier
		PF00397; found on position 7--37 in \uniprot{PIN1\_Human}. If available,
		binding affinities are also shown here. Interaction data is made
		available in \fileformat{MiTab} and \fileformat{XML} format
		\citep{17925023}, and can be downloaded by clicking on the yellow
		buttons in the right column.

	\item Scroll further down to the ``Switches'' section for a brief overview of
		the switches details of this instance obtained from "switches.ELM"
		\citep{23550212} (Figure~\ref{fig:explore_content_doc_ww_instance}). This
		particular instance is part of a phosphorylation-dependent molecular switch --
		only if p53 is phosphorylated on residue Serine-33 can it bind to the protein
		``Peptidyl-prolyl cis-trans isomerase NIMA-interacting 1 (PIN1)``.
		Clicking on the diagram will open an external link to the
		\rurl{switches.elm.eu.org} website where more detail can be found.

	%
	% Subsection: Links to external resources
	%
	\subsection*{Exploring Links to External Protein Resources}
	\label{subsec:explore_content_links_to_external_resources}

	\begin{figure}[h!]
		\centering
		\includegraphics[width=\textwidth]{Figures/explore_content/methods.png}
		\caption{
			The list of all experimental methods used in the ELM database,
			along with their PSI-MI identifiers.
		}
		\label{fig:explore_content_methods}
	\end{figure}

	\item Click on the sub-menu \button{ELM methods} under \button{ELM DB} to see a
		list of all experimental methods which have been used to identify
		motifs and instances (Figure~\ref{fig:explore_content_methods}).
		This table shows the internal method
		identifier in the first column, a link to the corresponding entry in
		the PSI-MI database \citep{17925023}, and the method name as annotated
		by the PSI-MI controlled vocabulary, as well as the type of experiment
		(\textit{in vitro}, \textit{in vivo}, \textit{in silico}, or a combination of these).
	    Clicking on the link in the ``instances'' column
		will list all instances annotated using that method.

		\sdesc{The filter bar on the top page can be used to filter the list of
			methods. The \fileformat{TSV} link creates a downloadable file
			in ``tab separated values'' format.}

	\begin{figure}[h!]
		\centering
		\includegraphics[width=\textwidth]{Figures/explore_content/pdbs.png}
		\caption{
		The list of all known PDB structures for which annotated motif instances exist in the ELM database.
		}
		\label{fig:explore_content_pdbs}
	\end{figure}

	\item Click on the sub-menu \button{ELM PDB structures} under \button{ELM DB} to
		see a list of all macromolecular structures in the ELM database
		(Figure~\ref{fig:explore_content_pdbs}).
		Structures annotated in ELM ideally (but not always) show
		both interaction partners (motif and domain). This page also contains
		links to RCSB/PDB \citep{12037327}, the individual instance and the motif
		class of that instance.

	% duplicate information:
	%	\sdesc{The filter bar on the top page can be used to filter the list of
	%		structures shown. The yellow \fileformat{TSV} link creates a
	%		downloadable file in ``tab separated values'' format.}

	\begin{figure}[h!]
		\centering
		\includegraphics[width=\textwidth]{Figures/explore_content/interactiondomains.png}
		\caption{
		A list of all interactions annotated in the database.
		}
		\label{fig:explore_content_interaction_domains}
	\end{figure}

	\item Click on the sub-menu \button{ELM binding domains} under \button{ELM DB}
		to see a complete list of all the interaction domains in ELM
		(Figure~\ref{fig:explore_content_interaction_domains}).
		This table shows the ELM classes that have been annotated
		with a corresponding interaction domain divided by the ELM
		class, a link to the PFAM \citep{26673716}, SMART \citep{25300481} or
		InterPro \citep{27899635} domain, as well as the name of the
		interacting domain followed by a brief description.

	% duplicate information:
	%	\sdesc{The filter bar on the top page can be used to filter the list of
	%		interactions shown. The \emph{tsv} link creates a downloadable
	%		file in ``tab separated values'' format.}

	%
	\begin{figure}[h!]
		\centering
		\includegraphics[width=\textwidth]{Figures/explore_content/switches.png}
		\caption{
		    A list of all molecular switches that are based on instances
		    from ELM and which are annotated at the \rurl{switches.elm.eu.org}
		    database.
		}
		\label{fig:explore_content_switches}
	\end{figure}

	\item Click on the sub-menu \button{ELM switches} under \button{ELM DB} to see a complete
		list of all the molecular switches annotated in ELM
		(Figure~\ref{fig:explore_content_switches}). This table shows
		the motif class, contains a link to UniProt, as well as the start and stop
		positions of the motif mediating the switch. The last two columns show
		links to the \rurl{switches.elm.eu.org} website, and a brief description
		of the switch (taken from the switches.ELM database, see \cite{23550212}).

	% duplicate information:
	%	\sdesc{The filter bar on the top page can be used to quickly filter
	%		the list of interactions shown.}

	%
	% Subsection: Exploring KEGG pathways
	%

	\subsection*{Visualizing KEGG pathways from ELM}
	\label{subsec:explore_content_kegg}

	\begin{figure}[h!]
		\centering
		\includegraphics[width=\textwidth]{Figures/explore_content/pathways.png}
		\caption{
		    A list of organisms for which pathways from KEGG have been mapped to protein sequences of instances in ELM. The number in brackets
		    denotes the number of different pathways per organism.
		}
		\label{fig:explore_content_pathways}
	\end{figure}

	\item Click on the sub-menu \button{ELM pathways} under \button{ELM DB} to see a list of all
		KEGG pathways contained in ELM
		(Figure~\ref{fig:explore_content_pathways}).
		Pathways are taken from and mapped onto the ``Kyoto Encyclopedia of Genes and Genomes'' (KEGG
		\citep{26476454}).

	\begin{figure}[h!]
		\centering
		\includegraphics[width=\textwidth]{Figures/explore_content/pathways_example.png}
		\caption{
			A list of all KEGG pathways for the taxon \textit{Gallus gallus} involving proteins annotated in ELM. Note that multiple
			instances from a single sequence can be annotated for one pathway. The color scheme on the right is used for coloring instances
			on the KEGG website (see Figure~\ref{fig:explore_content_pathways_kegg}).
		}
		\label{fig:explore_content_pathways_example}
	\end{figure}

	\item On the ``ELM pathways'' page
		(Figure~\ref{fig:explore_content_pathways_example})
		click on the link \button{Gallus gallus} to navigate to the page
		containing all pathways annotated for
		chicken.

	\begin{figure}[h!]
		\centering
		\includegraphics[width=\textwidth]{Figures/explore_content/pathways_kegg.png}
		\caption{
		An overlay of ELM annotations of proteins in the Adherens junction pathway in \textit{Gallus gallus}. The coloring of genes/proteins is as follows:
	light blue=\motif{CLV} (cleavage site),
	dark blue=\motif{DOC} (docking site),
	yellow=\motif{DEG} (degradation motif),
	green=\motif{LIG} (ligand binding motif),
	pink=\motif{MOD} (modification site),
	orange=\motif{TRG} (targeting motif),
	red=multiple classes per sequence.
	Light green boxes are colored by KEGG and represent the KEGG hyperlinks to GENES entries
	    %, indicating the presence of genes in the genome and also the completeness of the pathway
	    (see KEGG help \url{http://www.kegg.jp/kegg/document/help_pathway.html}).
		}
		\label{fig:explore_content_pathways_kegg}
	\end{figure}

	\item On the page with pathways annotated for chicken
		(Figure~\ref{fig:explore_content_pathways_example}),
		click on \button{Adherens junction} to the KEGG entry for this pathway,
		with each protein's color corresponding to ELM classes (see the color
		legend right side of Figure~\ref{fig:explore_content_pathways_kegg}).

	%
	% Subsection: Infections and Diseases
	%
	\subsection*{Browsing Infections and Diseases}
	\label{subsec:explore_content_infections_and_diseases}

	\begin{figure}[h!]
		\centering
		\includegraphics[width=\textwidth]{Figures/explore_content/viruses.png}
		\caption{
			A table of the ELM instances abused by viruses.
		}
		\label{fig:explore_content_viruses}
	\end{figure}

	\item \label{sec:explore_content_viruses} Click on the sub-menu \button{ELM virus instances} under \button{ELM DB}
		to see a list of all instances in ELM that have been annotated as being
		abused by viruses (Figure~\ref{fig:explore_content_viruses}).
		(The columns are identical to those listed in
		step~\ref{sec:explore_content_instances}, see Figure~\ref{fig:explore_content_instances}).

		\sdesc{The green buttons on the left can be used to filter this table
			by motif class. Click on the yellow links on the top right of
			the page to download the (complete) table in \fileformat{GFF},
			\fileformat{PIR}, \fileformat{FASTA} or \fileformat{TSV}
			format.}

	\begin{figure}[h!]
		\centering
		\includegraphics[width=\textwidth]{Figures/explore_content/diseases.png}
		\caption{
		A list of diseases that are mediated by short linear motifs. Disease description has been adapted from
		the ``Online Mendelian Inheritance in Man`` (OMIM) database \citep{17357067} and enriched with information about
		the role of the motif.
		}
		\label{fig:explore_content_diseases}
	\end{figure}

	\item Click on the sub-menu \button{ELM diseases} under \button{ELM DB} to see
		a list of diseases which are mediated by short linear motifs accompanied by
		a short description of the disease as well as the role of the motif.
		(Figure~\ref{fig:explore_content_diseases}). Disease information is taken
		from the ``Online Mendelian Inheritance in Man`` (OMIM) database
		\citep{17357067}.

	%
	% Subsection: Help page
	%
	\subsection*{Finding Help and Frequently Asked Questions}
	\label{subsec:explore_content_help}

	\begin{figure}[h!]
		\centering
		\includegraphics[width=\textwidth]{Figures/explore_content/help.png}
		\caption{
			The ELM ``Help'' and ``Questions \& Answers'' page.
			Users are encouraged to contact the authors via email if their questions
			regarding the ELM database can not be answered by this page.
		}
		\label{fig:explore_content_help}
	\end{figure}

	\item Click on the \button{Help} button on the right of the top navigation menu
		to visit the ELM Help page. This page has answers to the most
		Frequently asked questions, which you can see by clicking on a
		particular question. For example: Click on ``Regular expressions'' for
		a detailed description of the symbols used to build regular expressions
		to define motif classes.
	\end{enumerate}


	%%% General Search
	\section{Exploring the Content of the ELM Database Using the General Search}
	\label{sec:general_search}

	A general search text box is available to query the entire collection of
	manually curated information in the ELM database. This search field can be found in the header
	of the ELM database website (see for example Fig.
	\ref{fig:general_search_TP53_instances}). This search peforms a full-text
	query across multiple selected data sources in the ELM database, including
	ELM classes, instances, candidates, and switches. Using this general search is helpful in
	getting information about a particular protein and its annotation status in the ELM database
	(eg. full instance vs. candidate).

	%
	% Subsection: Necessary Resources
	%
	\subsection*{Necessary Resources}
	\subsubsection*{Software \& Hardware}
	A modern browser such as Firefox, Chrome, or Safari. ELM is best viewed
on a laptop or desktop computer, although tablets and smartphones will
also work.



	%
	% Subsection: General search
	%
	\subsection*{Using the General Search}
	\label{subsec:general_search_using}
	A modern browser such as Firefox, Chrome, or Safari. ELM is best viewed
on a laptop or desktop computer, although tablets and smartphones will
also work.



	\begin{enumerate}

	\begin{figure}[h!]
		\centering
		\includegraphics[width=\textwidth]{Figures/general_search/P04637_instances.png}
		\caption{
			The instances retrieved when performing a general search for
			\uniprot{P53\_HUMAN} using its UniProt identifier ``P04637''.
		}
		\label{fig:general_search_P04637_instances}
	\end{figure}

	\begin{figure}[h!]
		\centering
		\includegraphics[width=\textwidth]{Figures/general_search/P04637_switches.png}
		\caption{
			The switches found when performing a general search for
			\uniprot{P53\_HUMAN} using its UniProt identifier ``P04637'',
		}
		\label{fig:general_search_P04637_switches}
	\end{figure}


	\item Use the general search field (on the top of the page) to do a general
		search for p53 using its UniProt identifier by typing ``P04637'' in the
		search field and hitting ``Enter''. This will perform a query
		across multiple tables of the ELM database to find any matches to the
		search query ``P04637''.
		In this case, the results are grouped into matching instances
		(Figure~\ref{fig:general_search_P04637_instances})
		candidate classes and switches
		(Figure~\ref{fig:general_search_P04637_switches}).
		As there are no classes with ``P04637'' in the name or description, no classes are
		returned for this section of the query.

	% TODO: describe candidate classes (&instances) better
		\sdesc{
			The ``candidate classes'' are a collection of putative future ELM classes, which are not yet fully annotated, often submitted by
			ELM users. Keep in mind that these are a first draft on ELM classes and are still pending curation.
		}

	\begin{figure}[h!]
		\centering
		\includegraphics[width=\textwidth]{Figures/general_search/TP53_instances.png}
		\caption{
			The results retrieved when performing a general search for
			\uniprot{P53\_HUMAN} using the query ``p53''.
		}
		\label{fig:general_search_TP53_instances}
	\end{figure}

	\item Perform a search using the keyword ``p53'' in the general search field
		instead of its UniProt identifier ``P04637''.
		The set of results retrieved using this term as the search query
		(Figure~\ref{fig:general_search_TP53_instances})
		are different, returning 31 instances and 44
		switches (instead of 14 and 11). The reason for this is that the
	% TODO: HD: maybe discuss this rather in the commentary section?
	% Marc: I'm OK with this here. This is very relevant to the 'general search' only - Marc
		phrase ``p53'' also matches the UniProt identifier of
		\uniprot{CDH1\_YEAST} (P53197)! This is probably not what you had in mind when using this search term, so it is important to keep in mind
		to check for such false positive search results, when searching the database.

	\end{enumerate}

	\clearpage

	%%% P53
	\section{Detecting Short Linear Motifs in Protein Sequences}
	\label{sec:predicting_p53}

	One of the most useful (and used) features in ELM is the ability to
	detect motifs in proteins and sequences. Given a protein's amino acid
	sequence, the ``ELM Predictions'' pipeline searches for occurrences of
	each motif class using regular expressions, applies a set of filters to
	minimize false positives and creates a diagram to visualize the resulting
	set of putative motifs.

	In this protocol we will be viewing the manually annotated data of a
	typical protein, using p53 (UniProt ID: \uniprot{P53\_HUMAN}/P04637) as an
	example. We will cover how to find the manually annotated motifs and instances,
	the references used to annotate each
	instance, the experimental protocols used, and additional information including
	relationships to biological pathways (KEGG), diseases (OMIM) and molecular
	switches (switches.ELM).

	%
	% Subsection: Necessary Resources
	%
	\subsection*{Necessary Resources}
	\subsubsection*{Software \& Hardware}
	A modern browser such as Firefox, Chrome, or Safari. ELM is best viewed
on a laptop or desktop computer, although tablets and smartphones will
also work.



	\begin{enumerate}

	%
	% Subsection:Predicting ELM instances
	%

	\subsection*{Predicting ELM instances: Input form}
	\label{subsec:predicting_p53_input}

	\begin{figure}[h!]
		\centering
		\includegraphics[width=\textwidth]{Figures/predicting_p53/elm_search.png}
		\caption{
		The ELM input page for predicting motifs in a protein. Here, all fields have been filled in; strictly only the protein ID field is
		necessary to perform a search.
		}
		\label{fig:predicting_p53_elm_search}
	\end{figure}

	\item Open a browser, and navigate to the ELM homepage: \url{http://elm.eu.org}.\label{sec:predicting_p53_elm_search}
		Enter the UniProt ID \uniprot{P53\_HUMAN} in the search field labelled
		``Enter a UniProt identifier or accession number''. While typing, the page should
		autocomplete your input ``\uniprot{P53\_HUMAN} / P04637 (\textit{Homo
		sapiens})'' and already pre-fill other fields of the input form (Figure~\ref{fig:predicting_p53_elm_search}).
		Click on this entry to confirm that you want to search for
		motifs in this protein. Click on \button{Submit} to send the query to
		the server.

		\sdesc{The autocompletion mechanism queries UniProt for the given protein
			identifier; if it succeeds, then additional information from
			UniProt will be used to auto-populate the input boxes. In this
			example, \uniprot{P53\_HUMAN} is recognized as a human protein,
			and so ``Homo sapiens'' is automatically filled in the
			``Taxonomic Context'' field. Also, p53 has been annotated (by
			UniProt) to be localized to nucleus, cytosol, endoplasmic
			reticulum and mitochondrion, so these are also automatically
			applied as search criteria. The motif cutoff of ``100'' is a
			sufficiently high (lenient) threshold to ensure no motif class
			is filtered out based on motif probability.}

	\item Select the search criteria (optional). 
	%	It is possible to limit the results by ``Cell compartment'', ``Taxonomic Context'' or by changing the ``Motif Probability Cutoff''.
		To restrict the search to include
		motifs that are active only in certain cellular compartments, select one or
		more from the ``Cell compartment'' list (use the ``control'' key to select more than one
		option). It is also possible to select a ``Taxonomic Context'' to
		restrict the search to motifs from certain species. Start typing a
		species name in the ``Taxonomic Context'' input field to get an
		auto-completed list of species to select from. Additionally, a ``Motif
		probability cutoff'' can be used to only retain ELM classes whose
		pattern probability is below the given value.
		These filters are implemented later in \ref{sec:predicting_cv_0974}).
		For now, leave all filters at the default values that were
		auto-populated for p53. 


	%
	% Subsection: Understanding results: graphical summary
	%

	\subsection*{Interpreting the prediction results: Graphical Summary}
	\label{sec:predicting_p53_graphical_summary}

	\item \label{sec:predicting_p53_submit} Click \button{submit} to start
		searching for motifs. You will be brought to an intermediate page
		indicating that your results are being processed, and should be
		redirected to the final results page within a minute. You can bookmark
		this page: The results are stored for a week.

		\sdesc{The results are summarized in the first figure on the results
			page (see Figure~\ref{fig:predicting_p53_results_summary}).
			The graphical summary shows the results generated by the ELM
			prediction pipeline, combined with additional filters and
			information from external resources. The visualization should
			help you to interpret the results and to assess whether or not
			a motif is present in a sequence, as well as how likely it is
			to be functional, based on its structural context and
			evolutionary conservation. Motif instances that are manually
		annotated in the database appear as red (TP) or yellow (FP) ovals in
	the graphic. Blue/gray squares represent predicted motif occurrences.}

	\begin{figure}[h!] \centering
		\includegraphics[width=\textwidth]{Figures/predicting_p53/elm_results_summary.png}
		\caption{ The graphical results summary of the ELM Prediction pipeline
			for \uniprot{P53\_HUMAN}. Note that not all motif detections
			are shown (the image is truncated at the bottom). The top five
			rows show a set of structural features. Annotated and predicted
			motifs are shown as differently colored ovals/boxes. The
			information screens for three motifs are shown:
			\motif{CLV\_C14\_Caspase3-7}, \motif{DEG\_MDM2\_SWIB\_1} and
			\motif{CLV\_Separin\_Metazoa}. Note: This figure shows all
			three pop-up info screens at the same time; normally they are
			only shown when the cursor is placed over a motif.
		}
		\label{fig:predicting_p53_results_summary}
	\end{figure}

	\item \label{sec:predicting_p53_results_phosphorylation} The first row contains phosphorylation sites, as retrieved from
		PhosphoELM \citep{21062810}, and shows whether the phosphorylated amino
		acid is a serine, threonine or tyrosine. PhosphoELM is a database of
		manually annotated phosphorylation sites, obtained from scientific
		publications from low and high-throughput experiments. You can follow
		the link to PhosphoELM by clicking on the phosphorylation site in the
		image to get more information on individual phosphorylation sites.

		\sdesc{Phosphorylation sites are only available when the search is
			performed with a protein accession (eg. \emph{not} with a FASTA
			sequence alone) in step \ref{sec:predicting_p53_elm_search} and there is relevant information
			annotated in the PhosphoELM database. Phosphorylation sites
			are relevant to interpret ELM motif predictions, when the
			predicted motifs is required to be phosphorylated (as in several
			docking and ligand binding motifs) and for predicting
			phosphorylation motifs.}

	\item The second row shows SMART and PFAM domains detected by the SMART
		database \citep{9600884, 25300481, 9600884}
		(Figure~\ref{fig:predicting_p53_results_summary}). Hover the
		mouse over these domains to see their names and exact start and end
		positions.

		\sdesc{In order to be functional, motifs need to be accessible, and
			therefore they are usually not found within globular domains
			or structured regions \citep{21909575}. Any motif detected
			by the ELM prediction pipeline inside of a SMART domain are
			less likely to be functional, and therefore are shown with a gray box
			background (see commentary section "Structure Filter" at page \pageref{StructureFilter}).}

	\item The third row shows globular and disordered regions in the
		sequence, as predicted by GlobPlot \citep{12824398}. The fourth
		and fifth rows
		contain results from IUPred \citep{15955779}, another
		predictor of disordered protein regions. Protein segments with
		an IUPred score above 0.5 are considered to be disordered
		(see commentary section "Disorder Filter" at page \pageref{DisorderFilter}).

		\sdesc{Motifs are typically only functional when found in intrinsically
			disordered regions. Any motif occurrence detected by the ELM
			prediction pipeline that falls within disordered regions are
			more likely to be functional.}

	\item The 5th row (Figure~\ref{fig:predicting_p53_results_summary}) contains
		information on the protein's structure (see commentary section "Structure Filter" at page \pageref{StructureFilter}). 
		The secondary structure is
		predicted by mapping the motif occurrence onto high quality
		reference domain structures \citep{19852836}. Check the graphical
		representation, and whether the output of the secondary structure filter and
		the disorder predictors agree with each other with respect to which parts of the
		sequence are considered structured or disordered.

	\item The remainder of the figure (below ``secondary structure'' output)
		displays predicted and annotated motif instances, overlayed with the
		structural context from rows 2 and 3 (SMART domains and GlobPlot). A
		blue square indicates a single motif occurrence, and intensity of the
		color indicates the conservation of this sequence across a group of
		homologous proteins.
		Boxes in gray are motif occurrences that have been filtered out by the
		structure filter. Boxes that are blue \& gray are neutral:
		Within a globular domain, however part of a loop
		according to the secondary structure predictions.
		 If the sequence is already present in the ELM database,
		any motif instances that have already been annotated are shown as
		ovals. Lastly, any motifs detected that are annotated to be
		functional in homologous sequences, are shown as red \& blue
		rectangles (see commentary section "Instance Mapper" at page \pageref{InstanceMapper}).

		\sdesc{ In the case that not enough homologous sequences were detected
			to build an alignment, no conservation score can be calculated.
			Therefore all of the motif occurrences will be shown in a
			uniform shade of blue. }

	\item Place the cursor over the blue box for motif occurrence
		\motif{CLV\_C14\_Caspase3-7} at the end of the sequence (
		position 388--392). This will trigger the green and yellow
		information screen shown on the top right in Fig.
		\ref{fig:predicting_p53_results_summary}.
		This motif is in a disordered region, and has not been
		filtered out by the structural filter. Also, its conservation score
		of 0.910 is very high, indicating that this motif is highly conserved.

		\sdesc{The conservation score is based on how conserved the sequence is
			across a set of homologous proteins
			(see commentary section "Conservation Filter" at page \pageref{ConservationFilter}).}

	\item \label{sec:predicting_p53_results_structure_filter} Place the cursor over the blue \& gray rectangle for motif
		\motif{CLV\_Separin\_Metazoa} at position 171--175, a motif
		which was flagged as ``neutral'' by the ELM prediction pipeline.
		This will trigger the information screen (with the pink header) shown
		in Figure~\ref{fig:predicting_p53_results_summary} to appear.
		This motif resides inside the p53 PFAM domain, and thus has been
		subjected to ``structural filtering''. However, the secondary structure
		prediction suggests that this motif occurs within the looped region of this
		domain, so may be accessible.

		\sdesc{The information screen pop-up
		shows scores for all of the individual criteria used by the secondary
		structure filter: The name of the domain, the accessibility
		score, secondary structure score, a combined total score,
		and the associated total score P-value \citep{19852836}.}

	\item Place the cursor over the red oval for \motif{DEG\_MDM2\_SWIB\_1} at
		position 19--26. This motif is an annotated instance in the
		ELM database, and is therefore a bona-fide experimentally validated
		instance.

	%TODO: Marc: a little more description here?

	\begin{figure}[h!]
		\centering
		\includegraphics[width=\textwidth]{Figures/predicting_p53/elm_results_alignments_filtering_domains.png}
		\caption{
		This section of the results contains additional details on the
		aligned homologues used to calculate the conservation score,
		filtering results and globular domains.
		}
		\label{fig:predicting_p53_elm_results_alignment_filtering_domains}
	\end{figure}

	\item \label{sec:predicting_p53_results_alignment} Scroll down to below the results graphic to find additional information
		on the ELM prediction pipeline's results
		(Figure~\ref{fig:predicting_p53_elm_results_alignment_filtering_domains}).
		The first
		section contains links to download or view the multiple sequence
		alignments of homologous proteins used to calculate the conservation
		score. Click on the link ``Click here to enable the multiple sequence
		alignment viewer'' to open the alignment in Jalview (note: this
		requires the Java browser plugin, which might not be available on some
		browsers, and is now officially deprecated.). Alternatively you can also download the ``alignment'',
		``conservation features'' and ``phosphosite features'' files separately
		to view on a desktop (non-browser) installation of Jalview
		\citep{19151095}.

		\sdesc{ The search for possible homologues is performed against the
			UniRef90 database, a dataset of protein sequences with less
			than 90 percent identity between any two of them
			\citep{17379688}. It may occur that the BLAST results
			are not finished when the results page is shown: We suggest to
			refresh the page if you see the message ``Either not enough
			data available to calculate a sequence alignment or the
			calculations haven't finished yet''. In some cases it is also
			possible that no homologues will be detected. If you have
			refreshed the page after waiting for more than 3 minutes, this
			is most likely the case.}

	\item Scroll down to the section titled ``Filtering Summary'' to view some
		statistics about how many motifs and instances were filtered out
		(Fig.
		\ref{fig:predicting_p53_elm_results_alignment_filtering_domains}).
		The first two lines contain information on 
		which filters were applied in step \ref{sec:predicting_p53_elm_search} of this protocol.
		In this case 4 motifs representing 26 instances were filtered
		out as they did not occur in \textit{Homo sapiens}. An additional 5
		motifs (representing 11 instances) were filtered out because they are
		not annotated to the cell compartments automatically filled in on the
		search page (step \ref{sec:predicting_p53_elm_search}).
		The next three lines (``SMART'' \& ``Structural score'') show how many
		motifs and instances were not removed by the SMART and Secondary
		structure filters. A total of 42 motifs (representing 106 instances)
		passed the structural filter.

		\sdesc{Note that the graphical summary above does not contain sequences
			filtered out by the ``cell compartment'' and ``taxonomic
			context'' filters. However those filtered out by
			the SMART and Structural scores are shown in the graphic above
			(as gray rectangles).}

	\begin{figure}[h!]
		\centering
		\includegraphics[width=\textwidth]{Figures/predicting_p53/elm_results_known.png}
		\caption{
			The ELM prediction pipeline section displaying the p53 motifs
			that are ``known'', and have been annotated in the ELM
			database.
		}
		\label{fig:predicting_p53_elm_results_known}
	\end{figure}

	\item Scroll down to the section with the header ``Globular domains/ TM domains
		and signal peptide detected by the SMART server''
		(Figure~\ref{fig:predicting_p53_elm_results_alignment_filtering_domains}).
		This section contains information on which domains were detected by the
		SMART server, and their positions. Clicking on their names will bring
		you to the entry for that domain on the SMART or PFAM homepage.
		In this case the only domain detected is the ``p53'' PFAM domain.

	\item On the results page, scroll down to the heading: ``The ELMs in the
		following table are known instances annotated from the literature''
		(\ref{fig:predicting_p53_elm_results_known}).
		This table has details of the motifs and instances which have been
		manually annotated in the ELM database. The columns show each motif
		name, the sequence(s) that matched the motif as well as their starting
		and ending positions and the logic of the annotation followed by a
		short description of each motif, to which cell compartments its has
		been associated, and finally the regular expression of the motif.

	\begin{figure}[h!]
		\centering
		\includegraphics[width=\textwidth]{Figures/predicting_p53/elm_results_motifs.png}
		\caption{
		This table contains the list of putative motifs detected in the query sequence (only
		the top part of the table is shown). These are ``predictions`` in the sense
		that each of these motifs could be found in the sequence after applying (structure/context) filtering, however no experimental evidence has been annotated (yet) to determine if they are biologically functional.
		}
		\label{fig:predicting_p53_elm_results_motifs}
	\end{figure}

	\item Scroll further down to the section title ``Results of ELM motif search
		after globular domain filtering, structural filtering and context
		filtering'' to obtain an overview of all of the motifs and motif
		instances detected
		(\ref{fig:predicting_p53_elm_results_motifs})
		Each of the rows is a ``predicted'' motif: A sequence matching a
		motif's regular expression has been detected that has also passed the
		``structural filter''.
		Each row displays the motif identified, the matching peptide
		sequence and its position. Additional information is shown about the
		motif, its cell compartment and its regular expression. If the motif
		was detected in a homologue, the column ``PHI-Blast Instance
		mapping'' contains a link to the multiple sequence alignment of the
		homologous proteins. If a motif instance has been filtered out
		by the ``structural filter'', the ``Structural filter info'' column
		contains a link to a page with details on why.
		The last column contains information on the Probability filter: the
		probability reflects the chance to observe this motif in any random
		amino acid sequence (see section \ref{sec:explore_content})

	\begin{figure}[h!]
	\centering
		\includegraphics[width=\textwidth]{Figures/predicting_p53/elm_results_motifs_filtered.png}
		\caption{
		This table contains the list of putative motifs detected in the query sequence (only
		the top part of the table is shown) which were excluded by the
		structural or context filter.
		}
		\label{fig:predicting_p53_elm_results_motifs_filtered}
	\end{figure}

	\item Scroll further down to the heading ``List of excluded ELMs falling inside
		SMART/PFAM domains and/or scoring poorly with the structural filter (if
		applicable).''
		(Figure~\ref{fig:predicting_p53_elm_results_motifs_filtered})
		This table is similar to the one described above, but shows motif
		matches which were rejected by the structural filter.
	\end{enumerate}

	\clearpage

	%%% Predicting with CV_0974
	\section{Detecting Short Linear Motifs in Novel Protein Sequences}
	\label{sec:predicting_cv_0974}

	% TODO: read this section to check for consistency, etc.

	The ELM motif detection pipeline is a very powerful way to obtain a lot of
	information on which motifs are present, in which structural context they
	are. However, determining which motifs are actually true positive detections
	requires interpreting all of these results, as well as incorporating as much
	biological knowledge as possible. In this protocol we will be following a
	typical example of how one might use the ELM pipeline to search for
	motifs in novel sequences.

	% TODO: this might need to go into commentary section:
	Some pathogens have evolved short linear motifs in effector proteins to modify
	the intracellular signalling of their host cell for their own convenience
	\citep{25475989}. The Gram-negative bacteria \textit{Chromobacterium violaceum} is a
	opportunistic pathogen of humans whose mechanism of pathogenicity remain poorly
	understood. Its genome encodes a type III secretion system (T3SS) that is
	used by different pathogens to translocate bacterial proteins into the infected cells.
	Interestingly, the genes encoding this T3SS as well as other genes located in
	the same genomic location are very similar to the ones in \textit{Salmonella spp.},
	except for a couple of genes including the modular protein SptP
	\citep{15100995}. SptP in \textit{Salmonella spp.} is a secreted protein tyrosine
	phosphatase \citep{8866485} whose closest homolog in \textit{C. violaceum} is
	the protein \uniprot{CV\_0974}. To further understand the possible biological
	function of the protein \uniprot{CV\_0974} we will use it as an example
	application of the ELM server motif detection pipeline.

	%
	% Subsection: Necessary Resources
	%
	\subsection*{Necessary Resources}
	\subsubsection*{Software \& Hardware}
	A modern browser such as Firefox, Chrome, or Safari. ELM is best viewed
on a laptop or desktop computer, although tablets and smartphones will
also work.


	\subsubsection*{Files}
	You need to download the following file from UniProt containing the FASTA
	sequence of \uniprot{CV\_0974/Q7NZE8}:
	\url{http://www.uniprot.org/uniprot/Q7NZE8.fasta}

	\begin{enumerate}

	%
	% Subsection: Submitting a query
	%
	\subsection*{Submitting a query to ELM}
	\label{subsec:predicting_cv_0974_submitting}

	\begin{figure}[h!]
		\centering
		\includegraphics[width=\textwidth]{Figures/predicting_cv_0974/elm_search.png}
		\caption{
		The input query page for finding motifs in ELM. The sequence
	    for \textit{C. violaceum} protein \uniprot{CV\_0974} is used as an example for this protocol.
		}
		\label{fig:predicting_cv_0974_search}
	\end{figure}

	\item \label{subsec:predicting_cv_0974_search_fasta} Click on the ``ELM Predictions'' button in the menu to access the search
		query page (Fig. \ref{fig:predicting_cv_0974_search}).
		Here you should provide the amino acid sequence of \uniprot{CV\_0974}
		from UniProt (\url{http://www.uniprot.org/uniprot/Q7NZE8.fasta}), and
		enter it as FASTA format into the ``sequence input text box''.

	\item \label{subsec:predicting_cv_0974_search_taxon} In the ``Taxonomic Context'' field, enter the text ``Homo
		sapiens''. This will limit the search to motif classes
		which have been annotated for human proteins.

	    \sdesc{Certain human diseases occur when motifs are hijacked by
		    opportunistic pathogens (see Step~\ref{sec:explore_content_viruses} in
		    \ref{sec:explore_content} and Figure~\ref{fig:explore_content_viruses}).
		    By limiting the search to
		    human motifs, we will identify motifs which are known to exist
		    in humans and thus may be the target of motif hijacking.
		}

	\item \label{subsec:predicting_cv_0974_search_compartment} The bacterial protein \uniprot{CV\_0974} is likely to be an effector protein, similarly to its
	    homologue SptP. As SptP localizes to the cytosol, and we assume the same
	    for \uniprot{CV\_0974} so you should select ``cytosol'' in the ``Cell compartment''
	    field.

	\item \label{subsec:predicting_cv_0974_search_cutoff} For now, you should set
	    the ``Motif probability cutoff'' to the same value you used in
		\ref{subsec:predicting_p53_input} and enter ``100'' into this box.

	\subsection*{Interpreting the results}
	\label{subsec:predicting_cv_0974_submitting}

	\begin{figure}[h!]
		\centering
		\includegraphics[width=\textwidth]{Figures/predicting_cv_0974/elm_results_summary.png}
		\caption{
		The graphical results summary of the ELM prediction pipeline for the
		protein ``Probable Tyrosine phosphate'' (\uniprot{CV\_0974}). Note that not all motif
		detections are shown as the image is truncated at the bottom.
	% TODO: no conservation shown?!
	%	The top five rows show a handful of structural features. The motif occurrence
	%	are shown as blue boxes, the intensity of which indicates the
	%	conservation score. See steps
	%	\ref{sec:predicting_p53_results_phosphorylation} to
	%	\ref{sec:predicting_p53_results_structure_filter} of
	%	\ref{sec:predicting_p53}
	%	for more information.
		}
		\label{fig:predicting_cv_0974_results_summary}
	\end{figure}

	%
	% Subsection: Understanding results: graphical summary
	%

	\subsection*{Interpreting the prediction results}
	\label{sec:predicting_cv_0974_graphical_summary}

	\item Hit \button{Submit} to send the query to the ELM prediction pipeline.
	    The results are summarized in the first figure on the results page
		(see Fig. \ref{fig:predicting_cv_0974_results_summary})
	    See steps \ref{sec:predicting_p53_submit} -- \ref{sec:predicting_p53_results_structure_filter}
	    of \ref{sec:predicting_p53} for a description of the
	    graphical summary output. In this case, there are a lot of putative motif
	    hits and it will be difficult to investigate all of these. So you should try
	    to limit your search to most promising candidates first, by trying to remove
	    false positive hits.

	\item Go back to the ``ELM Predictions'' page (it's probably still open in
	    another browser tab) and enter the same values as used before in
		steps~\ref{subsec:predicting_cv_0974_search_fasta}-\ref{subsec:predicting_cv_0974_search_cutoff}.
		Now however, in order to remove false positive hits,
	    you should set the ``Motif probability cutoff value'' to a more
		stringent threshold. Enter ``0.001'' for this value. This will exclude
		all motif with a probability score higher than this, limiting the
		results to motifs which are less likely to be found by random
		chance (False Positives).

	% TODO: Marc: Please create and link new figure here, showing the result of the
		% search (with cutoff of 0.001 there should be only 22 instances of 17 classes)
	\item Hit \button{Submit} to send the query to the ELM prediction pipeline.
	    The results are summarized in the first figure on the results page
		(see Fig. \ref{fig:predicting_cv_0974_results_summary_limited}).
	    You notice the amount of motif classes detected has been
	    reduced drastically (from 54 to 11). % (instances from 156 to 22)

	\item By focussing on motif instances that reside in disordered context (in the
	    amino- or carboxy-terminus of this protein)
	    you have successfully limited the list of ELM motif classes down to 11.
	% \motif{DOC\_MAPK\_DCC\_7},
	% \motif{LIG\_Actin\_RPEL\_3},
	% \motif{LIG\_BIR\_III\_2},
	% \motif{LIG\_BIR\_II\_1},
	% \motif{LIG\_CSK\_EPIYA\_1},
	% \motif{LIG\_LIR\_LC3C\_4},
	% \motif{LIG\_SH2\_GRB2}, and
	% \motif{TRG\_AP2beta\_CARGO\_1}
	    %
	    (Now would be a good point in time to read up on the details of the
	    remaining motif classes to get a better idea, about which motifs might be
	    functional in your protein. Click on the motif identifiers on the left side
	    of the graphical summary to go to the details page for the motif classes).
	    For this protocol, however, just click on the motif identifier
	    \motif{LIG\_CSK\_EPIYA\_1} on the left side of the graphical summary to go
	    to the details page for this particular motif class.

	\item Find the entry for \motif{LIG\_CSK\_EPIYA\_1} in the graphical summary
	    and check the structural context for the two \motif{LIG\_CSK\_EPIYA\_1} motifs.
	    Both of them fall outside of the SMART
	    domain PTPC, and reside in a region with a protein disorder (IUPred) score
	    higher than 0.5. 

	% TODO: Is conscore available for this protein?
	%    \sdesc{You may notice that there are no conservation scores for either of
	%        these motifs. A closer examination of the alignments of homologous
	%        proteins shows that none of the other proteins have this part of the
	%        sequence, and are gapped, thus it is not possible to calculate a
	%        conservation score.}
	%
		In the ``Functional site description'' it is stated that 
		'bacterial proteins usually have repeats of EPIYA motifs.' The ELM
		prediction results did indeed also detect two EPIYA motifs in a 20
		amino acid range, lending further support to the likelihood that there
		are indeed two functional EPIYA motifs in \uniprot{CV\_0974}, which in
		turn suggests that these motifs may be involved in \textit{C.
		violaceum's} pathogenicity.

	    \sdesc{When this motif is tyrosine-phosphorylated it is recognized by C-terminal
	    kinase. Different effector proteins from human pathogens like
	    \textit{Bartonella henselae}, \textit{Helicobacter pylori} and
	    \textit{Haemophilus ducreyi} have been reported to use this motif to
	    interfere with the host signalling network to induce proliferation or to
	    avoid phagocytosis \citep{19380118, 12446738, 24902122}.
	    }

	%\item Click on the motif identifier \button{LIG\_CSK\_EPIYA\_1} on the left side of the
	%        graphical summary to go to the details page for this
	%        motif class. In the ``Functional site description'' it is stated that 'The
	%        bacterial proteins usually have repeats of EPIYA motifs.' The ELM
	%        prediction results did indeed also detect two EPIYA motifs in a 20
	%        amino acid range, lending further support to the likelihood that there
	%        are indeed two functional EPIYA motifs in \uniprot{CV\_0974}, which in
	%        turn suggests that these motifs may be involved in \textit{C.
	%        violaceum's} pathogenicity.
	%

	\item It is now up to the experimenter to decide whether to undertake
		validation experiments that would confirm the \motif{LIG\_CSK\_EPIYA\_1}
		predictions and function in controlling the host cell. Bioinformatics
		predictions are never certain, and this is especially true for SLiM
		prediction.  At this point it would be important to consult other
		bioinformatics tools and to search literature published after the motif
		was annotated. 
		(See the \ref{sec:guidelines-for-understanding-results})

	\end{enumerate}

	\clearpage

	%%% Searching via REST API
	\section{(Alternative Protocol) Searching the ELM database using the REST API}
	\label{sec:search_REST}

	Many researchers are interested in large-scale analyses rather than
	information about individual protein sequences. To this end, individual
	queries to the ELM webserver with a single protein ID at a time, are not
	practical.

	For this reason, as much information as practicable is made available via a
	REST interface \citep{Fielding2002}. This allows the user to interact
	with the ELM database and ELM webserver via scriptable URL requests.
	Each request can easily be tested in the browser before it is being
	automated in a script.

	In this section we will explore the various ways in which data can be
	downloaded both in using the browser as well as via the commandline.

	%
	% Subsection: Necessary Resources
	%
	\subsection*{Necessary Resources}
	\subsubsection*{Software}
	Ideally use \code{curl} \rurl{curl.haxx.se/} on the commandline. This
program can be launched from the terminal in any of the major operating
systems: OSX, Windows and Linux. Of course \code{curl} is only one of
many different ways to access web content programatically, and we
suggest anyone to use which ever program they feel is better suited for
their tasks.



	\begin{enumerate}

	%
	% Subsection: Necessary Resources
	%
	\subsection*{Downloading all ELM
	classes}\label{downloading-all-elm-classes}

	\begin{figure}[h!]
		\centering
		\includegraphics[width=\textwidth]{Figures/search_REST/elm_downloads_html.png}
		\caption{
		The ELM downloads page, which holds
		information about the different types of data (such as ``Classes'',
		``Instances'', etc; see menu to the right) that can be obtained from the
		server. The orange boxes are clickable links, the URL following them are
		used to highlight the URL scheme used by the server (\textbf{bold} font denotes
		specifics used in the examples such as query terms, or formats).
		}
		\label{fig:search_REST_downloads}
	\end{figure}

	\item Direct your browser to the URL '\rurl{elm.eu.org/downloads}' or select
	    \button{ELM Downloads} from the main menu
		(Figure~\ref{fig:search_REST_downloads}).
		This page contains links and descriptions on how to download ELM data
		in text format. The datasets are split into several smaller collections
		(for example ``Classes'', ``Instances'', etc). Each table contains
		links (in orange) to download the data in appropriate formats.

	    \sdesc{Each table also shows the `last modified date' indicating when
		    the data was last updated. This information is useful if you want to know
		    when to update your local data with the most up to date ELM
		    data as it allows you to determine whether you need to update or not. }

	\item Click on the first orange \button{html} link in the table ``Classes'' to
		navigate to the following URL:
		'\rurl{elm.eu.org/elms/elm\_index.html}'. This page shows all of the
		annotated ELM classes in the database. This page is the same one as
		shown in Figure~\ref{fig:explore_content_elms}. 

	\item Navigate to the following URL: '\rurl{elm.eu.org/elms.html?q=CSK}'
		specifying \cl{q=CSK} to limit the list of ELMs to those matching the
		search query ``CSK''. This page is again similar to the one shown in
		Figure~\ref{fig:explore_content_elms}, but with less classes.

		\sdesc{ This search result is identical to the result you would obtain
			by doing a ``manual'' search on the ELM Classes page
			(eg. typing 'CSK' in the search box and clicking \button{submit}) as
			described in step \ref{sec:predicting_p53_submit} of \ref{sec:explore_content}
			(see Figure~\ref{fig:explore_content_elms}).}


	\item Open the following URL: '\rurl{elm.eu.org/elms.tsv?q=CSK}' to download a
		list of classes that match the search query ``CSK'' (as in the previous
		step) in the ``tab separated values'' format. Note that this time we
		used the file extension '.tsv' instead of '.html' as before. By
		exchanging the '\fileformat{.html}' part of the URL with
		'\fileformat{.tsv}', we ask the webserver to give us the data in
		``tab-separated values'' format.

		\sdesc{ Depending on which browser you are using, the file may open
			directly in your browser, or you may be prompted to download
			the file or save it to a separate location. In the latter two
			cases you can open the downloaded file using a (plain) text
			file viewer, or possibly a spreadsheet viewer (such as
			Microsoft Excel or LibreOffice Calc).}

	\begin{figure}[h!]
		\centering
		\includegraphics[width=\textwidth]{Figures/search_REST/elm_curl_classes_CSK.png}
		\caption{
		A screenshot of a terminal window using
		\cl{curl} to download all ELM classes matching the term `CSK'.
		}
		\label{fig:search_REST_curl_csk}
	\end{figure}

	\item Type the following command into a command line terminal to
		download the same data from the previous step directly into the
		terminal:
		\cl{curl 'http://elm.eu.org/elms/elms\_index.tsv?q=CSK'}. The output
		should look similar to Figure~\ref{fig:search_REST_curl_csk}.
		The column names are the same as shown in Figure~\ref{fig:explore_content_elms}.

		\sdesc{ Use the curl option ``\cl{-o}'' to save the results directly to
			a file. For example: \cl{curl -o classes.tsv
			'http://elm.eu.org/elms/elms\_index.tsv?q=CSK'} will save the
			data to a file called \emph{classes.tsv}.}

	\begin{figure}[h!]
		\centering
		\includegraphics[width=\textwidth]{Figures/search_REST/elm_curl_instances_p53_human.png}
		\caption{
		Screenshot of a terminal window using \cl{curl} to download all ELM
		instances annotated for sequence P53\_HUMAN.
		}
		\label{fig:search_REST_curl_p53}
	\end{figure}

	\item To download a list of all motif instances detected in the protein sequence of human p53, type the
		following command into a terminal: \cl{curl
		'http://elm.eu.org/instances.gff?q=p53\_human'}. The output should look
		similar to that shown in Figure~\ref{fig:predicting_REST_curl_p53}. The
		output is in the ``General Feature Format''
		(see \rurl{www.ensembl.org/info/website/upload/gff.html\#moreinfo}),
		with the FASTA formatted sequence appended to the end of the output.

		\sdesc{Many other file formats are available for downloading instances
		annotations; see the downloads page for available options including the 
		\fileformat{FASTA}, \fileformat{GFF},
		\fileformat{PIR}, or \fileformat{PSI-MI} format (either
		\fileformat{XML} or \fileformat{MiTab}). 
		}


	\item To download a list of all instances matching the search query ``CLV'' annotated for the taxon
		``yellow fever mosquito (\textit{Aedes aegypti})'', enter the following
		command into a terminal:
		\cl{curl `http://elm.eu.org/instances.tsv?q=CLV\&taxon=aedes+aegypti'}
		(In general, any species name can be used, however remember to replace all ``spaces''
		with ``+''). This should return a single instance, the only one
		matching \motif{CLV} in \textit{A. aegypti}.

	\item More data (interactions, domains, methods, etc.) can be downloaded from
		ELM in analogous fashion as shown in the preceding steps. Take a look
		at the ELM Downloads page (\rurl{elm.eu.org/downloads}, figure
		\ref{fig:search_REST_downloads}) for an overview of which datasets can
		be downloaded, and what different filters and formats are available
		for each dataset.

	\end{enumerate}

	\clearpage

	%%% Predicting via REST API
	\section{(Alternate Protocol) Detecting Short Linear Motifs in Sequences using the REST API}
	\label{sec:predicting_REST}

	Querying ELM for motifs in a given sequence (as described in
	\ref{sec:predicting_p53} and \ref{sec:predicting_cv_0974}), gives you a nice
	overview of putative and annotated motifs in your query protein with a
	graphical representation
	using colors to highlight different regions of the protein sequence (eg.
	disordered vs. globular). It is however difficult to analyse a large set
	of protein sequences in this manner. Therefore, the ELM server
	provides an interface which you can use to submit your sequence in a
	programmatic way. Of course, this way, you won't receive the graphical
	output representation, but are limited to textual data representation.

	Currently, there exists a single URL (\rurl{elm.eu.org/start\_search/})
	to accept such queries. You can choose to either submit a UniProt name
	or accession (eg. '\url{elm.eu.org/start\_search/P53\_HUMAN.tsv}') or
	submit your raw sequence (e.g. '\rurl{elm.eu.org/start\_search/MAPRGFSCLLLLTSEIDLPVKRRA}'.)
	If the URL ends in `.tsv' then the server assumes you
	are using a UniProt id or accession; if it doesn't, then it assumes you
	are using raw sequence. See below for details.

	%
	% Subsection: Necessary Resources
	%
	\subsection*{Necessary Resources}
	\subsubsection*{Software}
	Ideally use \code{curl} \rurl{curl.haxx.se/} on the commandline. This
program can be launched from the terminal in any of the major operating
systems: OSX, Windows and Linux. Of course \code{curl} is only one of
many different ways to access web content programatically, and we
suggest anyone to use which ever program they feel is better suited for
their tasks.

%
	(Note that in some of the following screenshots, we append the following
	command to the commandline ``\cl{| column -t}'' to make the output more
	readable. All this does is to properly align all columns at tabstops.)

	\begin{enumerate}

	%
	% Subsection: Submitting a query via the REST API
	%
	\subsection*{Submitting a query to ELM via the REST API}
	\label{subsec:predicting_REST_submitting}

	\begin{figure}[h!]
		\centering
		\includegraphics[width=\textwidth]{Figures/predicting_REST/curl_P53.png}
		\caption{
		The commandline output when \cl{curl} is used to
		download all motifs predicted in human p53.}
		\label{fig:predicting_REST_curl_p53}
	\end{figure}


	\item Use \cl{curl} to query ELM for all motifs predicted to occur in human
	    p53 by typing the following into a terminal: `\cl{curl
		`http://elm.eu.org/start\_search/P53\_HUMAN.tsv}'.
		See Figure~\ref{fig:predicting_REST_curl_p53} for example output.
		Each resulting row represents a
		motif detection, and the first column ``elm\_identifier'' indicates
		which ELM class was identified, multiple matches to the same class are represented in multiple lines.
		The columns ``start'' and ``stop'' show
		the first and last amino acid positions that matched the motif.
		The column ``is\_annotated'' is True if this motif has been
		annotated in the ELM database as an (experimentally validated) motif
		instance. The column ``is\_phiblastmatch'' is True if a match was found
		by the ELM Instance mapper indicating that an experimentally validated
		instance in a homologous sequence was found
		(see commentary section "Instance Mapper" at page \pageref{InstanceMapper}).
		The column ``is\_filtered'' shows whether or not this motif was rejected by any
		of the ELM prediction filters (structure, topodom, taxon), whereby
		``topodomfilter'' uses information from UniProt to determine the protein's ``topology''
		with respect to trans-membrane domains or extracellular regions.
		The columns ``taxonfilter'' and ``structure'' indicate that an instance
		has been filtered by the taxonomy or secondary structure filter, respectively
		(see commentary sections "Taxon Filter" and "Structure Filter" 
		%at page \pageref{TaxonFilter}
		).

		\sdesc{In FigureF\ref{fig:predicting_REST_curl_p53}
			we use a slightly more
			advanced command to get the output to look nice in the
			terminal. We specified the \cl{-s} option to silence all
			\cl{curl} output other than the downloaded file, and piped
			the output directly to the
			\cl{column} command (this command exists on most Linux and
			OSX machines).}



	\begin{figure}[h!]
		\centering
		\includegraphics[width=\textwidth]{Figures/predicting_REST/predictions_query.png}
		\caption{
		It is possible to send amino acid sequences to the ELM Prediction
		pipeline. In this case we have used
		the curl option ``\cl{-o}'' to download directly to the file
		\cl{query.tsv}, and use a combination of the \cl{head} and
		\cl{column} commands to display the first 10 rows to the terminal.
		}
		\label{fig:predicting_REST_query}
	\end{figure}

	\item Use \cl{curl} to query ELM via protein sequence by using the URL
		'\rurl{elm.eu.org/start\_search/MAPRGFSCLLLLTSEIDLPVKRRA}'
		(Figure~\ref{fig:predicting_REST_query}).
		In this case the query is an arbitrary short
		peptide sequence, but this can (of course) contain any sequence you are
		interested in analysing. The output format is exactly the same as in the
		previous step.

		\sdesc{ This way of querying ELM is unfortunately not stable for long
			protein sequences. Different browsers and computers have
			different maximum lengths for URLs, and the excess text is
			often simply ignored. We recommend not using this method for
			sequences longer than 2000 amino acids.}

	\end{enumerate}

	\clearpage

	%%% Commentary
	\section*{Guidelines for Understanding Results}
	\label{sec:guidelines-for-understanding-results}

	TODO: Does anyone want to edit this section? It still might be missing things? -Marc

	The annotations in the ELM database have all been created manually, and in most cases
	you can safely assume that the annotations present in the database are of
	high confidence.
	These annotators have a lot of experience in reading the scientific literature,
	and know how to distinguish high confidence from suggestive experiments
	\cite{26581338}.
	However, bioinformatics databases are never perfect, and some annotations in
	ELM may be ~10 years old, so there could be updated literature.
	You are also encouraged to dig deeper into each annotation.
	In all cases, each entry contains descriptions of the experiments performed and
	links to the original research.

	Understanding the results generated by the ELM prediction pipeline 
	can sometimes require some extra work.
	In \ref{sec:predicting_p53} and \ref{sec:predicting_cv_0974} we gave a few
	examples of how to read and interpret the results from the ELM prediction
	pipeline.
	These are bioinformatics predictions, and therefore will rely
	on a heuristic which might make mistakes.
	In general the  prediction pipeline attempts to make as many predictions as
	possible, at the risk of making some False Positive predictions as well.

	In cases all of the intermediate results generated by the ELM prediction
	pipeline are made available to aid you in deciding which predictions are worth
	further investigation.
	Looking at the multiple sequence alignments used to generate the conservation
	score can (for example) help determine why a seemingly likely motif may have a
	(falsely) low confidence score.
	However, contrary you may have reason to believe that a predicted motif that was
	rejected by ELM's structure filter may actually exposed in a different
	structural conformation.
	We highly recommend you do not use ELM in isolation. Always make sure to check
	the literature and to use other bioinformatics tools to check alignments, structures and conservation (See the \ref{sec:internet-resources-with-annotations} for a handful of suggestions)

	\section*{Commentary:}\label{commentary}

	\emph{instructions: A brief discussion of the theory and applications of
	your}

	\subsection*{Background Information}\label{background-information}

	In order to interpret the data contained in ELM and the results produced by the
	ELM prediction tool, it is important to have a basic understanding of SLiM's
	and how they are affected by their structural and biological context. This
	background information summarises the different functionalities of SLiMs,
	describes the degenerate nature of motif sequences, and emphasises the need for
	contextual data for confident SLiM prediction.

	\subsubsection*{ELM categorises SLiMs depending on their functionality}

	SLiMs mediate different types of interactions, and based on this functionality,
	the ELM classes annotated in the ELM database are grouped into six main ELM
	types (Figure~\ref{fig:SLiMclasses}, \cite{24214962}). They can function as ligand binding sites or
	as sites for post-translational modification (PTM). Some ligand SLiMs are
	recognised by components of the cellular transport machinery and function as
	localisation signals that target proteins to specific sub-cellular compartments
	(\motif{TRG} type). Other ligand SLiMs are abundantly present in interfaces that mediate
	the assembly of large macromolecular complexes and in highly modular scaffold
	proteins that act as multivalent platforms for protein complex assembly
	(\motif{LIG} type). Docking motifs are ligand SLiMs that recruit modification enzymes to
	their substrates by binding to a site on the enzyme that is distinct from the
	active site (\motif{DOC} type). A subset of these, known as degrons, recruit ubiquitin
	ligases, which subsequently polyubiquitylate their substrates and hence target
	them for proteasomal degradation (\motif{DEG} type). SLiMs that act as sites for PTM can
	be targeted by specific enzymes for the addition or removal of a small chemical
	group (e.g. phosphorylation), a sugar molecule (e.g. glycosylation), a protein
	(e.g. ubiquitylation), or another moiety (e.g. lipidation) (\motif{MOD} type). Other PTM
	SLiMs mediate proteolytic cleavage by acting as target site for proteolytic
	enzymes (\motif{CLV} type), or are recognised for structural modification by isomerases
	that catalyse cis-trans isomerisation of the peptide backbone (\motif{DOC} type), see
	\cite{24926813} and \cite{24773235}.

	\subsubsection*{ELM regular expressions reflect the degenerate nature of SLiMs}

	As their name suggests, SLiMs are compact, being composed of a limited number of
	adjacent amino acids. Most of a motif's binding specificity however is conferred
	by only a subset of these amino acids. Those few residues that directly interact
	with the binding partner are evolutionarily conserved, although in many cases a
	subset of amino acids that share certain properties (such as similar charge,
	size or hydrophobicity) are allowed in these hotspot positions. In the motif
	positions that contribute little to the interaction, there are even less
	constraints, i.e. a broader range of amino acids is allowed in these positions
	\citep{21909575}. This sequence flexibility is captured in the regular
	expressions that are defined for each motif class. A first consequence of this
	degeneracy is that SLiMs co-operatively engage in interactions of relatively low
	affinity. Hence these binding events are transient and reversible, and can be
	readily modulated, for instance by PTM. These characteristics make SLiM-based
	interactions ideal mediators of the dynamic processes involved in cell
	signalling \citep{22480932}. Another consequence is that it might take only a few
	or even a single point mutation to generate or disrupt a functional motif in a
	protein. The associated ability to evolve convergently might underlie the
	proliferation of SLiMs and the rewiring of interactomes \citep{26589632,
	22346764}. Conversely, several SLiM-associated diseases have been
	characterised to date, for instance Liddle syndrome \citep{15483078}.

	\subsubsection*{ELM integrates data to increase the confidence of SLiM prediction}

	Due to their degenerate nature, motif sequences contain only very little
	information, and many short sequences in a proteome will match motif patterns.
	However, most of these matches will not represent functional motifs, and hence,
	when scanning a proteome for putative motifs using only the motif sequence
	patterns will yield a large number of false positive instances, far exceeding
	the number of true motifs.
	This is painful both for bioinformaticians and experimentalists.
	Therefore, reliable motif detection cannot go without
	experimental validation of candidate motifs, using different types of
	experiments and techniques \citep{26581338}. This however does not mean that
	bioinformatics analysis cannot guide researchers towards a subset of candidate
	motifs that have a higher probability to be functional and help rule out those
	candidate motifs that are likely to be false positives. Taking into account
	additional information, besides a match to a sequence pattern defining a SLiM,
	can greatly narrow the selection of putative motifs for experimental validation.
	Additional data for in silico analysis include conservation of the motif
	sequence, the location of the motif within the protein's structure and its
	accessibility for its binding partner, validated interaction with the binding
	partner, and in-cell co-localisation with the binding partner. The availability
	and usefulness of these additional data for SLiM discovery depends on their
	extensive and correct biocuration. A vast and increasing amount of biological
	data is available in a wide variety of sources, including the literature and
	large-scale datasets. In order to facilitate integration of data, they need to
	be collected, annotated and formatted in central data and knowledge
	repositories. The ELM database provides such a repository for experimentally
	validated linear motif classes and instances. The ELM prediction tool in turn
	relies on annotated data, both from the ELM database and other resources, to
	accurately analyse unknown sequences for candidate motifs and assist researchers
	in selecting the most plausible ones for experimental validation and discard
	likely false positive hits, saving them valuable time and assets
	\citep{22110040}.

	\subsubsection*{ELM Filters}

	\paragraph*{Disorder Filter}\label{DisorderFilter}

	ELM uses two different predictors of globularity/disorder: Firstly, GlobPlot
	developed by \cite{12824398}, uses amino acid propensities derived from a set of
	proteins to detect regions of globularity/disorder in any given protein
	sequence. Secondly, IUPred by \cite{15955779}, which, unlike GlobPlot, has not
	been trained on any dataset, but rather uses a position-specific scoring scheme
	assessing the tendency of any given amino acid to reside in either an ordered
	or disordered region. IUPred assigns a score between 0.0 to 1.0 to each amino
	acid of a protein sequence, whereby protein segments with an IUPred score above
	0.5 are considered to be disordered. ELM displays IUPred scores as a colored
	line on either green (disordered) or red (globular) background, see rows four
	and five of Figure~\ref{fig:predicting_p53_results_summary}.

	\paragraph*{Structure Filter}\label{StructureFilter}

	The structural filter, especially developed for ELM by \cite{19852836},
	assesses accessibility and secondary structural context derived from
	experimentally solved protein structures. It maps putatively functional motif
	occurrences onto a representative domain structure and scores these motifs for
	solvent-accessibility and secondary structure context. ELM displays this
	information as overlay boxes in the graphical output, whereby the user needs to
	hover over individual instance entries within structural context (see the
	\motif{CLV\_PCSK\_SKI1\_1} example in fig.
	\ref{fig:predicting_p53_results_summary}).

	\paragraph*{Conservation Filter}\label{ConservationFilter}

	This filter method for scoring the conservation of linear motif instances was
	developed by \cite{18460207} and subsequently implemented into the ELM
	pipeline. It requires only primary sequence-derived information (e.g. a
	multiple alignment and the sequence tree) and implicitly takes into account the
	degenerate nature of short linear motif patterns. By auto-generating multiple
	sequence alignments from a non-redundant database, generating distance-trees
	and taking into account motif degeneracy, it assesses for each ELM motif class
	found in any given sequence its conservation. The conservation score ranges
	from 0.0 (the predicted instance is present only in the query sequence) to 1.0
	(full conservation of the motif regular expression in all the informative
	sequences).
	In the graphical representation, motif conservation is indicated by coloring
	instances in different shades of blue, whereby darker shades of blue represent
	higher conservation.
	In the case that not enough homologous sequences were detected to build an
	alignment, no conservation score can be calculated. Therefore all of the motif
	occurrences will be shown in a uniform shade of blue.

	Since conserved motifs in structural regions are most likely conserved for
	structural integrity rather than motif function, one always has to assess the
	context when inspecting conservation score. Generally, best motif candidates
	are those with high conservation scores in regions of unstructured, unconserved
	regions.


	% When investigating conservation of short linear motifs, one has to bear in mind
	% Further generalizing, taking into
	% account that the more likely it is a functional motif for this prediction.
	%
	% islands of conservation amidst a sea of unconserved disordered amino acid residues.

	\paragraph*{Taxon Filter}\label{TaxonFilter}

	Each ELM class is annotated with one or more taxonomic ranges, for which
	experimental evidence has been found for the particular class
	(see "Present in Taxon: Eukaryota" in figure
	\ref{fig:explore_content_doc_cyclin}). This information is then used to filter taxons
	outside the annotated range whenever a user submits a query sequence to the ELM
	database (see the \button{taxonomic context} field in the ELM search input form in fig.
	\ref{fig:predicting_p53_elm_search}).

	\paragraph*{Instance Mapper}\label{InstanceMapper}

	The ELM instance mapper takes all annotated instances from the ELM database,
	generates a BLAST database from it and uses PHI-BLAST
	to detect sequence stretches in
	the query sequence which are similar to sequences in this database \citep{9705509}. This allows
	the instance mapper to effectively map known instances (for which experimental
	evidence exists) onto homologous sequences of unknown function.

	%TODO: cite PHiblast?     PMID:2231712

	\begin{figure}[h!]
	\centering
	\includegraphics[width=\textwidth]{Figures/Introduction/functional_classification_of_SLiMs.png}
	\caption{
	For each ELM class, the
	functional category to which it belongs is indicated by a three-letter prefix.
	Each ELM class is defined by a regular expression. Peptide sequences in
	proteins that match the regular expression of a specific ELM class and that
	were experimentally validated to be functional motifs are captured as ELM
	instances of that class. Degrons are a specific subtype of enzyme-recruiting
	docking motifs (see text for a detailed description).
	}
	\label{fig:SLiMclasses}
	\end{figure}


	% TODO: I can't think of anything to put here. -Marc
	% \subsection*{Critical Parameters and Troubleshooting}\label{critical-parameters-and-troubleshooting}
	% 
	% \emph{Factors that influence the protocol and to which special attention should be paid.
	% Common problems with the protocols, their causes, and potential solutions. The information may be presented in tabular form or it may be combined with Critical Parameters.
	% }

\section*{Key References with Annotation}
\label{sec:key-references-with-annotation}

TODO: can someone check this text please? I've got no more inspiration :( Thanks! -Marc

\begin{itemize}
\item ``ELM 2016-data update and new functionality of the
	eukaryotic linear motif resource.'' \citep{26615199}
	This is the latest publication on the ELM database.
\end{itemize}


\section*{Internet Resources with Annotations}%
\label{sec:internet-resources-with-annotations}

\begin{itemize}

\item \rurl{www.clustal.org/omega} Clustal Omega \citep{21988835} is a tool
for the alignment of multiple nucleic acid and protein sequences.

\item \rurl{www.jalview.org} Jalview \citep{19151095} is a Java desktop
application (and browser applet) that employs web services for sequence
alignment and visualization.

\item \rurl{proviz.ucd.ie} ProViz \citep{27085803} is an interactive protein
exploration tool, which searches several databases for information about
a given query protein. Data relevant to the protein like an alignment of
homologues, linear motifs, post translational modifications, domains,
secondary structure, sequence variations and others are graphically
represented relative to their position in the protein.

\end{itemize}


\bibliography{references}

\end{document}
