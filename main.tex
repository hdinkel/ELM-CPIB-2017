\documentclass[12pt]{article}
%% \usepackage{color} enabled in preamble
\usepackage[normalem]{ulem}  %% enable \st{} strikethrough

% This version hides the annotations so that the rendered PDF does not highlight the changes made in this revision
\input{preamble_clean}
%%\usepackage{fontspec}
%%\setmainfont{Times New Roman}
\usepackage{mathptmx}
\newcounter{proto}

% use 'rurl' to hide the 'http://' in text, but include it in the link
\newcommand\rurl[1]{%
	\href{http://#1}{\nolinkurl{#1}}%
}

% use 'button' to define a standard way to display clickable buttons 
\newcommand\button[1]{%
	\textbf{#1}
}

% use 'motif' to define a standard way to display motifs 
\newcommand\motif[1]{%
	\textsc{\lowercase{#1}}
}

% use 'uniprot' to define a standard way to display uniprot names
\newcommand\uniprot[1]{%
	\textsc{\lowercase{#1}}
}

% use 'fileformat' to define a standard way to display file formats 
\newcommand\fileformat[1]{%
	\textit{#1}
}

% use 'code' to define a standard way to display inilne code
\newcommand\code[1]{%
	\texttt{#1}
}

\begin{document}


% Frontmatter 
% 
\thispagestyle{plain}
\begin{center}

\textbf{\LARGE \textit{Current Protocols in Bioinformatics}}\\\vspace{2mm}
\end{center}

\section*{Eukaryotic Linear Motifs on the ELM database}

{\small \it
Marc Gouw$^{1}$,
Hugo Samano$^{1}$,
Kim Van Roey$^{1}$,
Francesca Diella$^{1}$,
Toby Gibson$^{1}$,
Holger Dinkel$^{1,2}$

$^{1}$ Structural and Computational Biology, European Molecular Biology
Laboratory, Meyerhofstrasse 1, 69117 Heidelberg, Germany\\
$^{2}$ Leibniz-Institute on Aging -- Fritz Lipmann Institute (FLI),
Beutenbergstrasse 11, 07745 Jena, Germany
}

\section*{Keywords}

Linear motifs, Bioinformatics, Protein-Protein Interaction, Molecular
switches, Cell regulation

\section*{Significance statement}

% \emph{instructions: Provide a 120-word-maximum statement about the
% significance of the protocols/topic described in your manuscript. This
% should be understandable to undergraduate- educated scientists outside
% their field of specialty. The goal is to explain the relevance of the
% work in broad context to a broad readership. It will be used in
% promotion of the article following publication.}

\begin{abstract}

% \emph{instructions: brief overview, no references, max 150 words}

    The Eukaryotic Linear Motif (ELM) resource (\rurl{elm.eu.org}) is a
manually curated database of short linear motifs (SLiMs). This protocol
explains how to best use this resource and explains how to access the
database content (both manual and scripted access), how to interpret the
output, and how to predict novel putative motifs in any given protein
sequence.

\end{abstract}




\thispagestyle{plain}
\begin{center}

\textbf{\LARGE \textit{Current Protocols in Bioinformatics}}\\\vspace{2mm}
\end{center}

\section*{Eukaryotic Linear Motifs on the ELM database}

{\small \it
Marc Gouw$^{1}$,
Hugo Samano$^{1}$,
Kim Van Roey$^{1}$,
Francesca Diella$^{1}$,
Toby Gibson$^{1}$,
Holger Dinkel$^{1,2}$

$^{1}$ Structural and Computational Biology, European Molecular Biology
Laboratory, Meyerhofstrasse 1, 69117 Heidelberg, Germany\\
$^{2}$ Leibniz-Institute on Aging -- Fritz Lipmann Institute (FLI),
Beutenbergstrasse 11, 07745 Jena, Germany
}

\section*{Keywords}

Linear motifs, Bioinformatics, Protein-Protein Interaction, Molecular
switches, Cell regulation

\section*{Significance statement}

% \emph{instructions: Provide a 120-word-maximum statement about the
% significance of the protocols/topic described in your manuscript. This
% should be understandable to undergraduate- educated scientists outside
% their field of specialty. The goal is to explain the relevance of the
% work in broad context to a broad readership. It will be used in
% promotion of the article following publication.}

\begin{abstract}

% \emph{instructions: brief overview, no references, max 150 words}

    The Eukaryotic Linear Motif (ELM) resource (\rurl{elm.eu.org}) is a
manually curated database of short linear motifs (SLiMs). This protocol
explains how to best use this resource and explains how to access the
database content (both manual and scripted access), how to interpret the
output, and how to predict novel putative motifs in any given protein
sequence.

\end{abstract}



% Introdoction 
% \section{Introduction}\label{introduction}

The activity and function of a protein is tightly regulated by its
cellular environment. To interact with their surroundings, proteins use
various types of binding modules that each display distinct binding
properties (\cite{10550212}). One prominent type of binding module
consists of short linear motifs (SLiMs) (\cite{18508681}). These compact
binding sites are generally located in intrinsically disordered regions
(IDR) of the proteome and commonly bind to structured domains in their
interaction partners (\cite{21909575}). SLiMs mediate different types of
interactions that regulate protein functionality, and hence are
important regulators of the dynamic processes involved in cell
signalling (\cite{22480932}) (\cite{24926813}) (Figure 1). The number of
SLiM instances in the human proteome is currently suggested to be over
one million (\cite{25038412}). Identifying SLiMs and elucidating their
functionality is an essential step in understanding cell regulation. The
Eukaryotic Linear Motif (ELM) resource contributes to this process by
providing the necessary tools to researchers working on motifs. It
consists of a database and a prediction tool. The database provides a
categorised repository of experimentally validated linear motif classes
and instances that were manually annotated form the literature. The ELM
prediction tool in turn relies on annotated data, both from the ELM
database and other resources, to accurately analyse unknown sequences
for candidate motifs and assist researchers in selecting the most
plausible ones for experimental validation and discard likely false
positive hits, saving them valuable time and assets (\cite{22110040}).
The following protocols will guide users through the different ELM
applications, explaining how to browse the curated data available in
ELM, how to analyse a protein sequence for putative motifs, and how to
interpret these data and avoid common pitfalls in SLiM discovery.

\begin{figure}[h!]
\centering
\includegraphics[width=\textwidth]{Figures/functional_classification_of_SLiMs.png}
\caption{
\textbf{Figure functional\_classification\_of\_SLiMs}
For each ELM
class, the functional category to which it belongs is indicated by a
three-letter prefix. Each ELM class is defined by a regular expression.
Peptide sequences in proteins that match the regular expression of a
specific ELM class and that were experimentally validated to be
functional motifs are captured as ELM instances of that class. Degrons
are a specific subtype of enzyme-recruiting docking motifs (see text for
a detailed description).
}
\end{figure}

\section*{Introduction}
\label{sec:introduction}

The activity and function of a protein is tightly regulated by its
cellular environment. To interact with their surroundings, proteins use
various types of binding modules that each display distinct binding
properties \cite{10550212}. One prominent type of binding module
consists of short linear motifs (SLiMs) \cite{18508681}. These compact
binding sites are generally located in intrinsically disordered regions
(IDR) of the proteome and commonly bind to
surfaceof a globular domain in a protein \cite{21909575}. SLiMs mediate
different types of interactions that regulate protein functionality, and hence
are important regulators of the dynamic processes involved in cell
signalling (\cite{22480932}) (\cite{24926813}). The number of
SLiM instances in the human proteome is currently suggested to be over
one million \cite{25038412}. Identifying SLiMs and elucidating their
functionality is an essential step in understanding cell regulation. The
Eukaryotic Linear Motif (ELM) resource contributes to this process by
providing the necessary tools to researchers working on motifs. It
consists of a database and a prediction tool. The database provides a
categorised repository of experimentally validated linear motif classes
and instances that were manually annotated form the literature. The ELM
prediction tool in turn relies on annotated data, both from the ELM
database and other resources, to accurately analyse unknown sequences
for candidate motifs and assist researchers in selecting the most
plausible ones for experimental validation and discard likely false
positive hits, saving them valuable time and assets \cite{22110040}.
The following protocols will guide users through the different ELM
applications, explaining how to browse the curated data available in
ELM, how to analyse a protein sequence for putative motifs, and how to
interpret these data and avoid common pitfalls in SLiM discovery.


% Browsing and exploring content
% \section{(Basic): Explore the content of the ELM DB}
\label{sec:explore_content}

The core of the ELM database is a repository of manually annotated motifs and
instances. As of December 2016, ELM contains over 260 motif classes categorized
into 6 different types: DOC (docking), LIG (Ligand binding), DEG (degradation),
CLV (cleavage), MOD (post translational modifications), and TRG
(targeting/anchoring) motifs (Figure functional\_classification\_of\_SLiMs).
These motifs are derived from various types of experiments reported in
literature. Each manually annotated motif also has a set of bona fide instances
(occurrences) of this motif. Currently, there are over 3000 annotated instances
annotated from over 2500 publications. The motif classes and motif instances
have been uploaded by a large group of annotators from around the globe. The
complete catalogue of manually curated data can be searched, browsed and
explored on the ELM website

TODO MARC: explain: elms, annotations, classes, instances, regular expressions. And links to GO, methods PSI. we use DOC CYCLIN etc as example.

%
% Subsection: Database content overview 
%
\subsection{Database content overview}
\label{subsec:explore_content_database}

% Counter for steps
\begin{enumerate}

\begin{figure}[h!]
	\centering
	\includegraphics[width=\textwidth]{Figures/explore_content/home.png} 
	\caption{
		The homepage of the ELM database (\rurl{elm.eu.org}).
	}
	\label{fig:explore_content_home}
\end{figure}

\item The ELM database is an online web resource. Open a browser and navigate
	to \rurl{elm.eu.org} to visit the homepage
	(Fig. \ref{fig:explore_content_home}).
	This page shows a brief explanation of the ELM resource, and a form to 
	search for SLiMs (which we cover in further detail in 
	\ref{sec:predicting_p53} and \ref{sec:predicting_cv_0974}).
	The column to the right is the news column, and is continually
	updated with the latest news about changes and additions to the database. 

\begin{figure}[h!]
	\centering
	\includegraphics[width=\textwidth]{Figures/explore_content/stats.png} 
	\caption{
		The ELM database statistics overview page shows the most up to
		date database statistics. As of January 2017 ELM has just over
		3000 annotated instances in 262 different motif classes.
	}
	\label{fig:explore_content_stats}
\end{figure}

\item On the ELM homepage click on the menu link \button{ELM DB} for an overview of
	the database statistics (Fig. \ref{fig:explore_content_stats}).
	This page displays the types and amounts of annotations contained in
	the database and a few links to third-part databases.
	Each line contains at least one link which will take you
	to the corresponding contents page (for example, clicking on
	\button{ELM instances} will take you to the page displaying all of the
	annotated instances in the database).

%
% Subsection: Browsing motif classes and instances
%
\subsection{Browsing motif classes and annotated instances}
\label{subsec:explore_content_classes_and_instances}

\begin{figure}[h!]
	\centering
	\includegraphics[width=\textwidth]{Figures/explore_content/elms.png} 
	\caption{
		The list of all motif classes annotated in the ELM database.
	}
	\label{fig:explore_content_elms}
\end{figure}

\item Click on the sub-menu \button{ELM classes} under \button{ELM DB} to visit
	the page listing all of the ELM classes
	(Fig. \ref{fig:explore_content_elms}).
	For each class, the following information is provided: ELM identifier,
	short description, regular expression, number of instances annotated
	for each class, and number of structures available. For details on each
	class, click on the ELM identifier; to get a list of annotated
	instances for an individual class, click on the number of instances.

	\sdesc{Use the search bar at the top of the page to filter for certain
		motif classes. For example, typing ``MAPK'' and hitting submit
		will perform a full-text search on all motif classes in the ELM
		database containing the term ``MAPK''. The green buttons on the
		left can also be used to filter this table. For example,
		toggling the ``DOC'' button will remove all DOC classes
		from the table (and clicking it again will bring them back).
		Lastly, the yellow \button{tsv} link can be used to export all
		motif classes as a ``tab separated values'' file.} 

\begin{figure}[h!]
	\centering
	\includegraphics[width=\textwidth]{Figures/explore_content/doc_cyclin_1_class.png}
	\caption{
        The motif details page for \motif{DOC\_CYCLIN\_1}. This page
	contains all of the manual annotation details for the
    \motif{DOC\_CYCLIN\_1} motif, the biological background summarized from
	the scientific literature including links to the primary
	literature and to external resources (Pubmed \cite{27899561},
	the Gene Ontology \cite{27899567}, PDB (\cite{12037327}) and
	more).
	}
	\label{fig:explore_content_doc_cyclin}
\end{figure}

\item Search the table for the term \motif{DOC\_CYCLIN\_1} and click on 
	\button{DOC\_CYCLIN\_1} in the left column to
	navigate to the page with details about the
	\motif{DOC\_CYCLIN\_1} motif class
	(Fig. \ref{fig:explore_content_doc_cyclin}).
	This page contains a description of the
	functional site class (a Cyclin recognition site), and a short
	description of the ELM and its regular expression, as well as a
	probability score, the taxonomic distribution of the motif and which
	domain (if any) is responsible for the interaction.

	\sdesc{The probability score is the probability that the regular
		expression represents a random selection of amino acids
		(similar to an information content score). A lower score
		indicates that the motif pattern is more difficult to find by
		chance in a random sequence.}

\item Scroll further down the \motif{DOC\_CYCLIN\_1} page 
	(Fig. \ref{fig:explore_content_doc_cyclin}) to view
	more details about the manually annotated data and instances in the
	database (Fig. \ref{fig:explore_content_doc_cyclin_1_abstract_instances})
	The ``abstract'' contains a more detailed description of the motif
	annotation. Click on the \button{show} button next to the ``selected
	references'' header for a list of publications relevant to this motif.
	Click on \button{show} next to ``GO terms'' for a complete list of all
	Gene Ontology (GO) terms annotated for this motif.

\begin{figure}[h!]
	\centering
	\includegraphics[width=\textwidth]{Figures/explore_content/doc_cyclin_1_abstract_instances.png}
	\caption{
		The second part of the \motif{DOC\_CYCLIN\_1} motif details page
		shows the motif abstract GO terms, and the list of annotated
		instances.
	}
	\label{fig:explore_content_doc_cyclin_1_abstract_instances}
\end{figure}

\item Scroll further down the \motif{DOC\_CYCLIN\_1} page to view
	the ``Instances'' header
	(Fig. \ref{fig:explore_content_doc_cyclin_1_abstract_instances})
	This table contains the list of all annotated \motif{DOC\_CYCLIN\_1}
	instances in the database of this motif. This includes the protein
	identifier, the start and end positions of the instance, the specific
	sequence matching the regular expression representing the motif and
	the ``logic'' of the instance.
	The ``\# Ev.'' indicates the number of experimental evidences
	associated with the annotation. ``Organism'' indicates in which  
	species in which the protein is found. Lastly the ``Notes'' column
	contains links to any ``interactions'' or ``switches'' present in the
	database, as well as links to PDB if this structure exists in PDB.

	\sdesc{The instance ``logic'' is an annotation of whether this is a
		\textit{bona-fide} instance, or whether it is a non-functional
		instance. \textit{TP} (True positive) indicates the instance is
		annotated with experimental evidence showing it is functional.
		\textit{FP} (False Positive) instances have experimental
		evidence suggesting function, but are believed to be
		non-functional after careful examination by our annotators.
		\textit{TN} (True Negative) instances have been experimentally
		determined to be non-functional, and \textit{U} (Unknown)
		instances do not have enough evidence to determine whether it
		is functional or not. The overwhelming majority of instances in
		ELM are \textit{TP}s.}


\begin{figure}[h!]
	\centering
	\includegraphics[width=\textwidth]{Figures/explore_content/instances.png} 
	\caption{
		The ``instances'' page can be used to search for instances in
		the ELM database. 
	}
	\label{fig:explore_content_instances}
\end{figure}

\item Click on the sub-menu \button{ELM instances} in \button{ELM DB} to visit
	the page where you can search and browse the instances annotated in ELM.
	(Fig. \ref{fig:explore_content_instances}).
	Note that only the first 100 instances matching the search criteria are shown.
	The search form can be used to filter results by a full text search, by
	instance logic, or organisms.

	\sdesc{This table can be filtered by motif class using the green toggle
		filters on the left hand side. Lastly, the yellow buttons at
		the top of the page can be used to download the instances in
		the following formats: \fileformat{gff}, \fileformat{pir},
		\fileformat{fasta} or \fileformat{tsv.}}

\begin{figure}[h!]
	\centering
	\includegraphics[width=\textwidth]{Figures/explore_content/doc_cyclin_1_instance.png}
	\caption{
	The instance details page for the \motif{DOC\_CYCLIN\_1} instance annotated
	for protein \uniprot{P53\_HUMAN} with start/end position ``381-385''. This page
	also contains links to many external databases including Uniprot
	\cite{25348405}, PDB \cite{12037327}, NCBI taxonomy, Pubmed
	\cite{27899561}, and KEGG Pathways \cite{26476454}, as well as the
	PSI-MI controlled vocabulary \cite{17925023}.
	}
	\label{fig:explore_content_doc_cyclin_instance}
\end{figure}

\item Type ``p53\_human'' in the search box to search for ELM Instances in this
	protein. Find the row for the ELM class \motif{DOC\_CYCLIN\_1} and click on
	the instance subsequence (highlighted in red) to go to the instance
	details page of this
	instance (Fig. \ref{fig:explore_content_doc_cyclin_instance})
	The top part of the page contains details about the instance
	and the protein it was identified in, and link to the Uniprot entry for
	the protein \cite{25348405}.

\item Scroll down to the ``Instance Evidence'' header to view details on the
	experimental evidence used to annotate this instance. The each
	experimental method is annotated using the Proteomics Standards
	Initiative Method Identifier (PSI-MI) \cite{17925023} as well as the
	references in which the experiments were published.

	\sdesc{
		The ``biosource'' indicates whether method is \textit{in vivo},
		\textit{in vitro}, \textit{in sicilo} or a combination of these.
		The ``logic'' column indicates whether this experiment
		``supports'' or ``contradicts'' this instance being functional.
		Each method is also annotated with a ``reliability'', which can
		be any of ``certain'', ``likely'', ``unlikely'' or
		``unspecified''.}

%
% Subsection: Switches, pathways and other external resources. 
%
\subsection{Finding Switches and molecular interactions}
\label{subsec:explore_content_external_resoureces}

\begin{figure}[h!]
	\centering
	\includegraphics[width=\textwidth]{Figures/explore_content/doc_ww_pin_1_4_instance.png}
	\caption{
	The instance details page for the \motif{DOC\_WW\_Pin1\_4}
	instance found in Human P53 (\uniprot{P53\_HUMAN}) with start/end position
	``30-35''.
	}
	\label{fig:explore_content_doc_ww_instance}
\end{figure}

\item Repeat the previous search by clicking on the sub-menu \button{ELM instances}
	under \button{ELM DB} and type ``p53\_human'' in the search box. This time,
	find the ELM instance \motif{DOC\_WW\_Pin1\_4} motif with the
	start/end position ``30-35''. (You can sort the table by clicking on the
	header lines: click on ``Start'' to sort by start position ). Click on the
	start/end position or the subsequence which will take you to the details
	page (Fig. \ref{fig:explore_content_doc_ww_instance}). This
	page is similar to that described for the P53 instance \motif{DOC\_CYCLIN\_1}
	(Fig. \ref{fig:explore_content_doc_cyclin_instance}).
	Additionally, for this instance there is information available about
	its interaction partner and a molecular switch which is mediated by
	this motif instance.

\item Scroll down to the ``Interactions'' header to view information about this
	instance's interactions
	(Fig.  \ref{fig:explore_content_doc_ww_instance}). This instance
	interacts with \uniprot{PIN1\_Human} via the ``WW'' domain (Pfam identifier
	PF00397; found on position 7-37 in \uniprot{PIN1\_Human}. If available,
	binding affinities are also shown here. Interaction data is made
	available in \fileformat{mitab} and \fileformat{xml} format
	(\cite{17925023}), and can be downloaded by clicking on the yellow
	buttons in the right column.

\item Scroll further down to the ``Switches'' section for a brief overview of
	the switches details of this instance obtained form switches.ELM
	(\cite{23550212}) (Fig. \ref{fig:explore_content_doc_ww_instance}). This
	particular instance is involved in the switch phosphorylating P53.
	Clicking on the diagram will open an external link to the
	switches.ELM website.

%
% Subsection: Links to external resources
%
\subsection{Exploring Links External Protein Resources}
\label{subsec:explore_content_links_to_external_resources}

\begin{figure}[h!]
	\centering
	\includegraphics[width=\textwidth]{Figures/explore_content/methods.png} 
	\caption{
		The list of all experimental methods used in the ELM database,
		along with their PSI-MI identifiers.
	}
	\label{fig:explore_content_methods}
\end{figure}

\item Click on the sub-menu \button{ELM methods} in \button{ELM DB} to see a
	list of all experimental methods which have been used to identify
	motifs and instances
	(Fig \ref{fig:explore_content_methods}).
	This table shows the internal method
	identifier in the first column, a link to the corresponding entry in
	the PSI-MI database (\cite{17925023}), and the method name as annotated
	by the PSI-MI controlled vocabulary, as well as the type of experiment
	(in vitro/in vivo). Clicking on the link in the ``instances'' column
	will list all instances annotated using that method.  

	\sdesc{The filter bar on the top page can be used to filter the list of
		methods. The \fileformat{tsv} link creates a downloadable file
		in ``tab separated values'' format.}

\begin{figure}[h!]
	\centering
	\includegraphics[width=\textwidth]{Figures/explore_content/pdbs.png} 
	\caption{
	The list of all known structures in PDB which are also in ELM.
	}
	\label{fig:explore_content_pdbs}
\end{figure}

\item Click on the sub-menu \button{ELM pdb structures} in \button{ELM DB} to
	see a list of all macromolecular structures in the ELM database
	(Fig. \ref{fig:explore_content_pdbs}).
	Structures annotated in ELM ideally (but not always) show
	both interaction partners, motif and domain. This page also contains
	links to RCSB/PDB (\cite{12037327}), the individual instance and the motif
	class of that instance.

	\sdesc{The filter bar on the top page can be used to filter the list of
		structures shown. The yellow \fileformat{tsv} link creates a
		downloadable file in ``tab separated values'' format.}

\begin{figure}[h!]
	\centering
	\includegraphics[width=\textwidth]{Figures/explore_content/interactiondomains.png}
	\caption{
	A list of all interactions annotated in the database.
	}
	\label{fig:explore_content_interaction_domains}
\end{figure}

\item Click on the sub-menu \button{ELM binding domains} under \button{ELM DB}
	to see a complete list of all the interaction domains in ELM
	(Fig. \ref{fig:explore_content_interaction_domains}).
	This table shows the ELM classes which have been annotated
	with a corresponding interaction domain. This table shows the ELM
	class, a link to the Pfam \cite{26673716}, SMART \cite{25300481} or 
	InterPro (\cite{27899635}) domain, as well as the name of the
	interacting domain followed by a brief description.

	\sdesc{The filter bar on the top page can be used to filter the list of
		interactions shown. The \emph{tsv} link creates a downloadable
		file in ``tab separated values'' format.}

%
\begin{figure}[h!]
	\centering
	\includegraphics[width=\textwidth]{Figures/explore_content/switches.png} 
	\caption{
		A list of all switches annotated switches.ELM also contained in ELM.
	}
	\label{fig:explore_content_switches}
\end{figure}

\item Click on the sub-menu \button{ELM switches} in \button{ELM DB} to see a complete
	list of all the switches in ELM
	(Fig. \ref{fig:explore_content_switches}). This table shows
	the motif class, contains a link to Uniprot, and the start and stop
	positions of the motif mediating the switch. The last two columns have
	links to switches.ELM, and a brief description of the switch also taken
	from switches.ELM \cite{23550212}.

	\sdesc{The filter bar on the top page can be used to quickly filter
		the list of interactions shown.} 

%
% Subsection: Exploreing KEGG pathways
%

\subsection{Visualizing KEGG pathways from ELM}
\label{subsec:explore_content_kegg}

\begin{figure}[h!]
	\centering
	\includegraphics[width=\textwidth]{Figures/explore_content/pathways.png} 
	\caption{
	A list of all Pathways from KEGG with proteins in ELM.
	}
	\label{fig:explore_content_pathways}
\end{figure}

\item Click on the sub-menu \button{ELM pathways} in \button{ELM DB} to see a list of all
	KEGG pathways contained in ELM
	(Fig. \ref{fig:explore_content_pathways}).
	Pathways are from the ``Kyoto Encyclopedia of Genes and Genomes'' (KEGG
	\cite{26476454}) database mapped to ELM instances. 
	
\begin{figure}[h!]
	\centering
	\includegraphics[width=\textwidth]{Figures/explore_content/pathways_example.png} 
	\caption{
		A list of all KEGG pathways in \textit{Gallus gallus} involving proteins annotated in ELM.
	}
	\label{fig:explore_content_pathways_example}
\end{figure}

\item On the ``ELM pathways'' page
	(Fig.  \ref{fig:explore_content_pathways_example})
	click on the link \button{gallus gallus} to navigate to the page
	containing all pathways annotated for
	chicken.

\begin{figure}[h!]
	\centering
	\includegraphics[width=\textwidth]{Figures/explore_content/pathways_kegg.png} 
	\caption{
	A list of all annotated pathways for taxon \textit{Gallus gallus}
	}
	\label{fig:explore_content_pathways_kegg}
\end{figure}

\item One the page with chicken pathways
	(Fig. \ref{fig:explore_content_pathways_example})
	click on \button{Adherens junction} to the KEGG entry for this pathway,
	with each protein's color corresponding to ELM classes (see the color
	legend right side of figure \ref{fig:explore_content_pathways_kegg}).

%
% Subsection: Infections and Diseases
%
\subsection{Browsing Infections and Diseases}
\label{subsec:explore_content_infections_and_diseases}

\begin{figure}[h!]
	\centering
	\includegraphics[width=\textwidth]{Figures/explore_content/viruses.png} 
	\caption{
		A table of the ELM instances abused by viruses.
	}
	\label{fig:explore_content_viruses}
\end{figure}

\item Click on the sub-menu \button{ELM virus instances} under \button{ELM DB}
	to see a list of all instances in ELM that have been annotated as being
	abused by viruses (Fig. \ref{fig:explore_content_viruses}).
	The columns are identical to those listed in Step 7 (Fig.  
	\ref{fig:explore_content_instances}).

	\sdesc{The green buttons on the left can be used to filter this table
		by motif class. Click on the yellow links on the top right of
		the page to download the (complete) table in \fileformat{gff},
		\fileformat{pir}, \fileformat{fasta} or \fileformat{tsv}
		format.}

\begin{figure}[h!]
	\centering
	\includegraphics[width=\textwidth]{Figures/explore_content/diseases.png}
	\caption{
	A list of all diseases in ELM.
	}
	\label{fig:explore_content_diseases}
\end{figure}

\item Click on the sub-menu \button{ELM diseases} under \button{ELM DB} to see
	a list of all motif classes that have been annotated with a disease
	(Fig. \ref{fig:explore_content_diseases}). Disease information is taken
	from the Online Mendelian Inheritance in Man (OMIM) database
	\cite{17357067}.

%
% Subsection: Help page 
%
\subsection{Finding Help and Frequently Asked Questions}
\label{subsec:explore_content_help}

\begin{figure}[h!]
	\centering
	\includegraphics[width=\textwidth]{Figures/explore_content/help.png} 
	\caption{
		The ELM help and Questions \& Answers page.
	}
	\label{fig:explore_content_viruses}
\end{figure}

\item Click on the \button{Help} button on the right of the top navigation menu
	to visit the ELM Help page. This page has answers to the most
	Frequently asked questions, which you can see by clicking on a
	particular question. For example: Click on ``Regular expressions'' for
	a detailed description of the symbols used to build regular expressions
	to define motif classes.
\end{enumerate}


\section{(Basic): Explore the content of the ELM DB}
\label{sec:explore_content}

The core of the ELM database is a repository of manually annotated motifs and
instances. As of December 2016, ELM contains over 260 motif classes categorized
into 6 different types: DOC (docking), LIG (Ligand binding), DEG (degradation),
CLV (cleavage), MOD (post translational modifications), and TRG
(targeting/anchoring) motifs (Figure functional\_classification\_of\_SLiMs).
These motifs are derived from various types of experiments reported in
literature. Each manually annotated motif also has a set of bona fide instances
(occurrences) of this motif. Currently, there are over 3000 annotated instances
annotated from over 2500 publications. The motif classes and motif instances
have been uploaded by a large group of annotators from around the globe. The
complete catalogue of manually curated data can be searched, browsed and
explored on the ELM website

TODO MARC: explain: elms, annotations, classes, instances, regular expressions. And links to GO, methods PSI. we use DOC CYCLIN etc as example.

%
% Subsection: Necessary Resources
%
\subsection{Necessary Resources}
\subsubsection{Software \& Hardware}
A modern browser such as Firefox, Chrome, or Safari. ELM is best viewed
on a laptop or desktop computer, although tablets and smartphones will
also work.




%
% Subsection: Database content overview 
%
\subsection{Database content overview}
\label{subsec:explore_content_database}


% Counter for steps
\begin{enumerate}

\begin{figure}[h!]
	\centering
	\includegraphics[width=\textwidth]{Figures/explore_content/home.png} 
	\caption{
		The homepage of the ELM database (\rurl{elm.eu.org}).
	}
	\label{fig:explore_content_home}
\end{figure}

\item The ELM database is an online web resource. Open a browser and navigate
	to \rurl{elm.eu.org} to visit the homepage
	(Fig. \ref{fig:explore_content_home}).
	This page shows a brief explanation of the ELM resource, and a form to 
	search for SLiMs (which we cover in further detail in 
	\ref{sec:predicting_p53} and \ref{sec:predicting_cv_0974}).
	The column to the right is the news column, and is continually
	updated with the latest news about changes and additions to the database. 

\begin{figure}[h!]
	\centering
	\includegraphics[width=\textwidth]{Figures/explore_content/stats.png} 
	\caption{
		The ELM database statistics overview page shows the most up to
		date database statistics. As of January 2017 ELM has just over
		3000 annotated instances in 262 different motif classes.
	}
	\label{fig:explore_content_stats}
\end{figure}

\item On the ELM homepage click on the menu link \button{ELM DB} for an overview of
	the database statistics (Fig. \ref{fig:explore_content_stats}).
	This page displays the types and amounts of annotations contained in
	the database and a few links to third-part databases.
	Each line contains at least one link which will take you
	to the corresponding contents page (for example, clicking on
	\button{ELM instances} will take you to the page displaying all of the
	annotated instances in the database).

%
% Subsection: Browsing motif classes and instances
%
\subsection{Browsing motif classes and annotated instances}
\label{subsec:explore_content_classes_and_instances}

\begin{figure}[h!]
	\centering
	\includegraphics[width=\textwidth]{Figures/explore_content/elms.png} 
	\caption{
		The list of all motif classes annotated in the ELM database.
	}
	\label{fig:explore_content_elms}
\end{figure}

\item Click on the sub-menu \button{ELM classes} under \button{ELM DB} to visit
	the page listing all of the ELM classes
	(Fig. \ref{fig:explore_content_elms}).
	For each class, the following information is provided: ELM identifier,
	short description, regular expression, number of instances annotated
	for each class, and number of structures available. For details on each
	class, click on the ELM identifier; to get a list of annotated
	instances for an individual class, click on the number of instances.

	\sdesc{Use the search bar at the top of the page to filter for certain
		motif classes. For example, typing ``MAPK'' and hitting submit
		will perform a full-text search on all motif classes in the ELM
		database containing the term ``MAPK''. The green buttons on the
		left can also be used to filter this table. For example,
		toggling the ``DOC'' button will remove all DOC classes
		from the table (and clicking it again will bring them back).
		Lastly, the yellow \button{tsv} link can be used to export all
		motif classes as a ``tab separated values'' file.} 

\begin{figure}[h!]
	\centering
	\includegraphics[width=\textwidth]{Figures/explore_content/doc_cyclin_1_class.png}
	\caption{
        The motif details page for \motif{DOC\_CYCLIN\_1}. This page
	contains all of the manual annotation details for the
    \motif{DOC\_CYCLIN\_1} motif, the biological background summarized from
	the scientific literature including links to the primary
	literature and to external resources (Pubmed \cite{27899561},
	the Gene Ontology \cite{27899567}, PDB (\cite{12037327}) and
	more).
	}
	\label{fig:explore_content_doc_cyclin}
\end{figure}

\item Search the table for the term \motif{DOC\_CYCLIN\_1} and click on 
	\button{DOC\_CYCLIN\_1} in the left column to
	navigate to the page with details about the
	\motif{DOC\_CYCLIN\_1} motif class
	(Fig. \ref{fig:explore_content_doc_cyclin}).
	This page contains a description of the
	functional site class (a Cyclin recognition site), and a short
	description of the ELM and its regular expression, as well as a
	probability score, the taxonomic distribution of the motif and which
	domain (if any) is responsible for the interaction.

	\sdesc{The probability score is the probability that the regular
		expression represents a random selection of amino acids
		(similar to an information content score). A lower score
		indicates that the motif pattern is more difficult to find by
		chance in a random sequence.}

\item Scroll further down the \motif{DOC\_CYCLIN\_1} page 
	(Fig. \ref{fig:explore_content_doc_cyclin}) to view
	more details about the manually annotated data and instances in the
	database (Fig. \ref{fig:explore_content_doc_cyclin_1_abstract_instances})
	The ``abstract'' contains a more detailed description of the motif
	annotation. Click on the \button{show} button next to the ``selected
	references'' header for a list of publications relevant to this motif.
	Click on \button{show} next to ``GO terms'' for a complete list of all
	Gene Ontology (GO) terms annotated for this motif.

\begin{figure}[h!]
	\centering
	\includegraphics[width=\textwidth]{Figures/explore_content/doc_cyclin_1_abstract_instances.png}
	\caption{
		The second part of the \motif{DOC\_CYCLIN\_1} motif details page
		shows the motif abstract GO terms, and the list of annotated
		instances.
	}
	\label{fig:explore_content_doc_cyclin_1_abstract_instances}
\end{figure}

\item Scroll further down the \motif{DOC\_CYCLIN\_1} page to view
	the ``Instances'' header
	(Fig. \ref{fig:explore_content_doc_cyclin_1_abstract_instances})
	This table contains the list of all annotated \motif{DOC\_CYCLIN\_1}
	instances in the database of this motif. This includes the protein
	identifier, the start and end positions of the instance, the specific
	sequence matching the regular expression representing the motif and
	the ``logic'' of the instance.
	The ``\# Ev.'' indicates the number of experimental evidences
	associated with the annotation. ``Organism'' indicates in which  
	species in which the protein is found. Lastly the ``Notes'' column
	contains links to any ``interactions'' or ``switches'' present in the
	database, as well as links to PDB if this structure exists in PDB.

	\sdesc{The instance ``logic'' is an annotation of whether this is a
		\textit{bona-fide} instance, or whether it is a non-functional
		instance. \textit{TP} (True positive) indicates the instance is
		annotated with experimental evidence showing it is functional.
		\textit{FP} (False Positive) instances have experimental
		evidence suggesting function, but are believed to be
		non-functional after careful examination by our annotators.
		\textit{TN} (True Negative) instances have been experimentally
		determined to be non-functional, and \textit{U} (Unknown)
		instances do not have enough evidence to determine whether it
		is functional or not. The overwhelming majority of instances in
		ELM are \textit{TP}s.}


\begin{figure}[h!]
	\centering
	\includegraphics[width=\textwidth]{Figures/explore_content/instances.png} 
	\caption{
		The ``instances'' page can be used to search for instances in
		the ELM database. 
	}
	\label{fig:explore_content_instances}
\end{figure}

\item Click on the sub-menu \button{ELM instances} in \button{ELM DB} to visit
	the page where you can search and browse the instances annotated in ELM.
	(Fig. \ref{fig:explore_content_instances}).
	Note that only the first 100 instances matching the search criteria are shown.
	The search form can be used to filter results by a full text search, by
	instance logic, or organisms.

	\sdesc{This table can be filtered by motif class using the green toggle
		filters on the left hand side. Lastly, the yellow buttons at
		the top of the page can be used to download the instances in
		the following formats: \fileformat{gff}, \fileformat{pir},
		\fileformat{fasta} or \fileformat{tsv.}}

\begin{figure}[h!]
	\centering
	\includegraphics[width=\textwidth]{Figures/explore_content/doc_cyclin_1_instance.png}
	\caption{
	The instance details page for the \motif{DOC\_CYCLIN\_1} instance annotated
	for protein \uniprot{P53\_HUMAN} with start/end position ``381-385''. This page
	also contains links to many external databases including Uniprot
	\cite{25348405}, PDB \cite{12037327}, NCBI taxonomy, Pubmed
	\cite{27899561}, and KEGG Pathways \cite{26476454}, as well as the
	PSI-MI controlled vocabulary \cite{17925023}.
	}
	\label{fig:explore_content_doc_cyclin_instance}
\end{figure}

\item Type ``p53\_human'' in the search box to search for ELM Instances in this
	protein. Find the row for the ELM class \motif{DOC\_CYCLIN\_1} and click on
	the instance subsequence (highlighted in red) to go to the instance
	details page of this
	instance (Fig. \ref{fig:explore_content_doc_cyclin_instance})
	The top part of the page contains details about the instance
	and the protein it was identified in, and link to the Uniprot entry for
	the protein \cite{25348405}.

\item Scroll down to the ``Instance Evidence'' header to view details on the
	experimental evidence used to annotate this instance. The each
	experimental method is annotated using the Proteomics Standards
	Initiative Method Identifier (PSI-MI) \cite{17925023} as well as the
	references in which the experiments were published.

	\sdesc{
		The ``biosource'' indicates whether method is \textit{in vivo},
		\textit{in vitro}, \textit{in sicilo} or a combination of these.
		The ``logic'' column indicates whether this experiment
		``supports'' or ``contradicts'' this instance being functional.
		Each method is also annotated with a ``reliability'', which can
		be any of ``certain'', ``likely'', ``unlikely'' or
		``unspecified''.}

%
% Subsection: Switches, pathways and other external resources. 
%
\subsection{Finding Switches and molecular interactions}
\label{subsec:explore_content_external_resoureces}

\begin{figure}[h!]
	\centering
	\includegraphics[width=\textwidth]{Figures/explore_content/doc_ww_pin_1_4_instance.png}
	\caption{
	The instance details page for the \motif{DOC\_WW\_Pin1\_4}
	instance found in Human P53 (\uniprot{P53\_HUMAN}) with start/end position
	``30-35''.
	}
	\label{fig:explore_content_doc_ww_instance}
\end{figure}

\item Repeat the previous search by clicking on the sub-menu \button{ELM instances}
	under \button{ELM DB} and type ``p53\_human'' in the search box. This time,
	find the ELM instance \motif{DOC\_WW\_Pin1\_4} motif with the
	start/end position ``30-35''. (You can sort the table by clicking on the
	header lines: click on ``Start'' to sort by start position ). Click on the
	start/end position or the subsequence which will take you to the details
	page (Fig. \ref{fig:explore_content_doc_ww_instance}). This
	page is similar to that described for the P53 instance \motif{DOC\_CYCLIN\_1}
	(Fig. \ref{fig:explore_content_doc_cyclin_instance}).
	Additionally, for this instance there is information available about
	its interaction partner and a molecular switch which is mediated by
	this motif instance.

\item Scroll down to the ``Interactions'' header to view information about this
	instance's interactions
	(Fig.  \ref{fig:explore_content_doc_ww_instance}). This instance
	interacts with \uniprot{PIN1\_Human} via the ``WW'' domain (Pfam identifier
	PF00397; found on position 7-37 in \uniprot{PIN1\_Human}. If available,
	binding affinities are also shown here. Interaction data is made
	available in \fileformat{mitab} and \fileformat{xml} format
	(\cite{17925023}), and can be downloaded by clicking on the yellow
	buttons in the right column.

\item Scroll further down to the ``Switches'' section for a brief overview of
	the switches details of this instance obtained form switches.ELM
	(\cite{23550212}) (Fig. \ref{fig:explore_content_doc_ww_instance}). This
	particular instance is involved in the switch phosphorylating P53.
	Clicking on the diagram will open an external link to the
	switches.ELM website.

%
% Subsection: Links to external resources
%
\subsection{Exploring Links External Protein Resources}
\label{subsec:explore_content_links_to_external_resources}

\begin{figure}[h!]
	\centering
	\includegraphics[width=\textwidth]{Figures/explore_content/methods.png} 
	\caption{
		The list of all experimental methods used in the ELM database,
		along with their PSI-MI identifiers.
	}
	\label{fig:explore_content_methods}
\end{figure}

\item Click on the sub-menu \button{ELM methods} in \button{ELM DB} to see a
	list of all experimental methods which have been used to identify
	motifs and instances
	(Fig \ref{fig:explore_content_methods}).
	This table shows the internal method
	identifier in the first column, a link to the corresponding entry in
	the PSI-MI database (\cite{17925023}), and the method name as annotated
	by the PSI-MI controlled vocabulary, as well as the type of experiment
	(in vitro/in vivo). Clicking on the link in the ``instances'' column
	will list all instances annotated using that method.  

	\sdesc{The filter bar on the top page can be used to filter the list of
		methods. The \fileformat{tsv} link creates a downloadable file
		in ``tab separated values'' format.}

\begin{figure}[h!]
	\centering
	\includegraphics[width=\textwidth]{Figures/explore_content/pdbs.png} 
	\caption{
	The list of all known structures in PDB which are also in ELM.
	}
	\label{fig:explore_content_pdbs}
\end{figure}

\item Click on the sub-menu \button{ELM pdb structures} in \button{ELM DB} to
	see a list of all macromolecular structures in the ELM database
	(Fig. \ref{fig:explore_content_pdbs}).
	Structures annotated in ELM ideally (but not always) show
	both interaction partners, motif and domain. This page also contains
	links to RCSB/PDB (\cite{12037327}), the individual instance and the motif
	class of that instance.

	\sdesc{The filter bar on the top page can be used to filter the list of
		structures shown. The yellow \fileformat{tsv} link creates a
		downloadable file in ``tab separated values'' format.}

\begin{figure}[h!]
	\centering
	\includegraphics[width=\textwidth]{Figures/explore_content/interactiondomains.png}
	\caption{
	A list of all interactions annotated in the database.
	}
	\label{fig:explore_content_interaction_domains}
\end{figure}

\item Click on the sub-menu \button{ELM binding domains} under \button{ELM DB}
	to see a complete list of all the interaction domains in ELM
	(Fig. \ref{fig:explore_content_interaction_domains}).
	This table shows the ELM classes which have been annotated
	with a corresponding interaction domain. This table shows the ELM
	class, a link to the Pfam \cite{26673716}, SMART \cite{25300481} or 
	InterPro (\cite{27899635}) domain, as well as the name of the
	interacting domain followed by a brief description.

	\sdesc{The filter bar on the top page can be used to filter the list of
		interactions shown. The \emph{tsv} link creates a downloadable
		file in ``tab separated values'' format.}

%
\begin{figure}[h!]
	\centering
	\includegraphics[width=\textwidth]{Figures/explore_content/switches.png} 
	\caption{
		A list of all switches annotated switches.ELM also contained in ELM.
	}
	\label{fig:explore_content_switches}
\end{figure}

\item Click on the sub-menu \button{ELM switches} in \button{ELM DB} to see a complete
	list of all the switches in ELM
	(Fig. \ref{fig:explore_content_switches}). This table shows
	the motif class, contains a link to Uniprot, and the start and stop
	positions of the motif mediating the switch. The last two columns have
	links to switches.ELM, and a brief description of the switch also taken
	from switches.ELM \cite{23550212}.

	\sdesc{The filter bar on the top page can be used to quickly filter
		the list of interactions shown.} 

%
% Subsection: Exploreing KEGG pathways
%

\subsection{Visualizing KEGG pathways from ELM}
\label{subsec:explore_content_kegg}

\begin{figure}[h!]
	\centering
	\includegraphics[width=\textwidth]{Figures/explore_content/pathways.png} 
	\caption{
	A list of all Pathways from KEGG with proteins in ELM.
	}
	\label{fig:explore_content_pathways}
\end{figure}

\item Click on the sub-menu \button{ELM pathways} in \button{ELM DB} to see a list of all
	KEGG pathways contained in ELM
	(Fig. \ref{fig:explore_content_pathways}).
	Pathways are from the ``Kyoto Encyclopedia of Genes and Genomes'' (KEGG
	\cite{26476454}) database mapped to ELM instances. 
	
\begin{figure}[h!]
	\centering
	\includegraphics[width=\textwidth]{Figures/explore_content/pathways_example.png} 
	\caption{
		A list of all KEGG pathways in \textit{Gallus gallus} involving proteins annotated in ELM.
	}
	\label{fig:explore_content_pathways_example}
\end{figure}

\item On the ``ELM pathways'' page
	(Fig.  \ref{fig:explore_content_pathways_example})
	click on the link \button{gallus gallus} to navigate to the page
	containing all pathways annotated for
	chicken.

\begin{figure}[h!]
	\centering
	\includegraphics[width=\textwidth]{Figures/explore_content/pathways_kegg.png} 
	\caption{
	A list of all annotated pathways for taxon \textit{Gallus gallus}
	}
	\label{fig:explore_content_pathways_kegg}
\end{figure}

\item One the page with chicken pathways
	(Fig. \ref{fig:explore_content_pathways_example})
	click on \button{Adherens junction} to the KEGG entry for this pathway,
	with each protein's color corresponding to ELM classes (see the color
	legend right side of figure \ref{fig:explore_content_pathways_kegg}).

%
% Subsection: Infections and Diseases
%
\subsection{Browsing Infections and Diseases}
\label{subsec:explore_content_infections_and_diseases}

\begin{figure}[h!]
	\centering
	\includegraphics[width=\textwidth]{Figures/explore_content/viruses.png} 
	\caption{
		A table of the ELM instances abused by viruses.
	}
	\label{fig:explore_content_viruses}
\end{figure}

\item Click on the sub-menu \button{ELM virus instances} under \button{ELM DB}
	to see a list of all instances in ELM that have been annotated as being
	abused by viruses (Fig. \ref{fig:explore_content_viruses}).
	The columns are identical to those listed in Step 7 (Fig.  
	\ref{fig:explore_content_instances}).

	\sdesc{The green buttons on the left can be used to filter this table
		by motif class. Click on the yellow links on the top right of
		the page to download the (complete) table in \fileformat{gff},
		\fileformat{pir}, \fileformat{fasta} or \fileformat{tsv}
		format.}

\begin{figure}[h!]
	\centering
	\includegraphics[width=\textwidth]{Figures/explore_content/diseases.png}
	\caption{
	A list of all diseases in ELM.
	}
	\label{fig:explore_content_diseases}
\end{figure}

\item Click on the sub-menu \button{ELM diseases} under \button{ELM DB} to see
	a list of all motif classes that have been annotated with a disease
	(Fig. \ref{fig:explore_content_diseases}). Disease information is taken
	from the Online Mendelian Inheritance in Man (OMIM) database
	\cite{17357067}.

%
% Subsection: Help page 
%
\subsection{Finding Help and Frequently Asked Questions}
\label{subsec:explore_content_help}

\begin{figure}[h!]
	\centering
	\includegraphics[width=\textwidth]{Figures/explore_content/help.png} 
	\caption{
		The ELM help and Questions \& Answers page.
	}
	\label{fig:explore_content_viruses}
\end{figure}

\item Click on the \button{Help} button on the right of the top navigation menu
	to visit the ELM Help page. This page has answers to the most
	Frequently asked questions, which you can see by clicking on a
	particular question. For example: Click on ``Regular expressions'' for
	a detailed description of the symbols used to build regular expressions
	to define motif classes.
\end{enumerate}


%%% General Search
% \section{(Alternate) General Search Box}
\label{sec:general_search}

A general search text box is available to query the entire collection of
manually curated information in ELM DB. This search field can be found at the
top of almost all pages on the ELM website (for example in Fig.
\ref{fig:general_search_TP53_instances}). This search is a full-text
search across multiple selected data sources in the database, including
protein and ELM class. 

\begin{enumerate}

\begin{figure}[h!]
	\centering
	\includegraphics[width=\textwidth]{Figures/general_search/P04637_instances.png} 
	\caption{
		The instances retrieved when performing a general search for
		\uniprot{P53\_Human} using its Uniprot identifier ``P04637''.
	}
	\label{fig:general_search_P04637_instances}
\end{figure}

\begin{figure}[h!]
	\centering
	\includegraphics[width=\textwidth]{Figures/general_search/P04637_switches.png} 
	\caption{
		The switches found when performing a general search for
		\uniprot{P53\_Human} using its Uniprot identifier ``P04637'',
	}
	\label{fig:general_search_P04637_switches}
\end{figure}


\item Use the general search field (on the top of the page) to do a general
	seach for P53 using its Uniprot identifier by typing ``P04637'' in the
	search field and hitting ``Enter''. This will search
	across instances, motif classes and switches to find any matches to the
	search query ``P04637''. 
	The results are grouped into matching instances
	(Fig. \ref{fig:general_search_P04637_instances})
	candidate classes and switches
	(Fig. \ref{fig:general_search_P04637_switches}).
	As there are no classes with ``P04637'' in the name, no classes are
	returned with this query.

	\sdesc{
		The ``candidate classes'' are a separate part of ELM, where
		users can propose novel classes to be annotated. Although these
		are returned by the general search, we would advise users not
		to use this data, as it is still pending curation.
	}

\begin{figure}[h!]
	\centering
	\includegraphics[width=\textwidth]{Figures/general_search/TP53_instances.png} 
	\caption{
		The results retrieved when performing a general search for
		\uniprot{P53\_Human} using the query ``p53''.
	}
	\label{fig:general_search_TP53_instances}
\end{figure}

\item Perform a search using the keyword ``p53'' in the general search field
	instead of its Uniprot identifier ``P04637''.
	The set of results retrieved using this term as search query
	(Fig. \ref{fig:general_search_TP53_instances})
	in this case are different, returning 31 instances and 44
	switches (instead of 14 and 11). The reason for this is that the
	phrase ``P53'' also matches the Uniprot identifier of
	\uniprot{CDH1\_YEAST} (P53197). This is important to keep in mind when
	using the general search field.

\end{enumerate}

\section{(Alternate) General Search Box}
\label{sec:general_search}

A general search text box is available to query the entire collection of
manually curated information in ELM DB. This search field can be found at the
top of almost all pages on the ELM website (for example in Fig.
\ref{fig:general_search_TP53_instances}). This search is a full-text
search across multiple selected data sources in the database, including
protein and ELM class. 

%
% Subsection: Necessary Resources
%
\subsection{Necessary Resources}
\subsubsection{Software \& Hardware}
%A modern browser such as Firefox, Chrome, or Safari. ELM is best viewed
on a laptop or desktop computer, although tablets and smartphones will
also work.



%
% Subsection: General search 
%
\subsection{Using the General Search}
\label{subsec:general_search_using}
A modern browser such as Firefox, Chrome, or Safari. ELM is best viewed
on a laptop or desktop computer, although tablets and smartphones will
also work.



\begin{enumerate}

\begin{figure}[h!]
	\centering
	\includegraphics[width=\textwidth]{Figures/general_search/P04637_instances.png} 
	\caption{
		The instances retrieved when performing a general search for
		\uniprot{P53\_Human} using its Uniprot identifier ``P04637''.
	}
	\label{fig:general_search_P04637_instances}
\end{figure}

\begin{figure}[h!]
	\centering
	\includegraphics[width=\textwidth]{Figures/general_search/P04637_switches.png} 
	\caption{
		The switches found when performing a general search for
		\uniprot{P53\_Human} using its Uniprot identifier ``P04637'',
	}
	\label{fig:general_search_P04637_switches}
\end{figure}


\item Use the general search field (on the top of the page) to do a general
	seach for P53 using its Uniprot identifier by typing ``P04637'' in the
	search field and hitting ``Enter''. This will search
	across instances, motif classes and switches to find any matches to the
	search query ``P04637''. 
	The results are grouped into matching instances
	(Fig. \ref{fig:general_search_P04637_instances})
	candidate classes and switches
	(Fig. \ref{fig:general_search_P04637_switches}).
	As there are no classes with ``P04637'' in the name, no classes are
	returned with this query.

	\sdesc{
		The ``candidate classes'' are a separate part of ELM, where
		users can propose novel classes to be annotated. Although these
		are returned by the general search, we would advise users not
		to use this data, as it is still pending curation.
	}

\begin{figure}[h!]
	\centering
	\includegraphics[width=\textwidth]{Figures/general_search/TP53_instances.png} 
	\caption{
		The results retrieved when performing a general search for
		\uniprot{P53\_Human} using the query ``p53''.
	}
	\label{fig:general_search_TP53_instances}
\end{figure}

\item Perform a search using the keyword ``p53'' in the general search field
	instead of its Uniprot identifier ``P04637''.
	The set of results retrieved using this term as search query
	(Fig. \ref{fig:general_search_TP53_instances})
	in this case are different, returning 31 instances and 44
	switches (instead of 14 and 11). The reason for this is that the
	phrase ``P53'' also matches the Uniprot identifier of
	\uniprot{CDH1\_YEAST} (P53197). This is important to keep in mind when
	using the general search field.

\end{enumerate}

%%% Predicting with P53
% \section{(Basic) Predicting ELMs in sequences}
\label{sec:predicting_p53}

One of the most useful (and used) features in ELM is the ability to
detect motifs in proteins and sequences. Given a protein's amino acid
sequence, the ``EML Predictions'' pipeline searches for occurrences of
each motif class using regular expressions, apply a set of filters to
help judging results, and to visualize resulting set of putative motifs.

In this protocol we will be viewing the manually annotated data of a
typical protein, using p53 (Uniprot ID: P53\_HUMAN/P04637) as an
example. We will cover how to find the manually annotated motifs and
instances, and how to find the motif instances, the references used to
annotate each instance, the experimental protocols used, and additional
information including relationships to biological pathways (such as KEGG
\cite{26476454}), diseases (from OMIM \cite{17357067}) and molecular
switches (in switches.ELM \cite{23550212}).

%
% Subsection: Necessary Resources
%

\subsection{Necessary Resources}\label{necessary-resources}

\subsubsection{Software \& Hardware}\label{software-hardware}

A modern browser such as Firefox, Chrome, or Safari. ELM is best viewed
on a laptop or desktop computer, although tablets and smartphones will
also work.

\begin{enumerate}

%
% Subsection:Predicting ELM insstances
%

\subsection{Predicting ELM instances using data from ELM database}
\label{predicting-elm-instances-using-data-from-elm-database}

\begin{figure}[h!]
	\centering
	\includegraphics[width=\textwidth]{Figures/predicting_p53/elm_search.png} 
	\caption{
	\textbf{Figure TP53-BP1-1}
	The query input page for ELM for predicting motifs in a
	given protein sequence.
	}
\end{figure}

\item Open a browser, and navigate to the ELM homepage: http://elm.eu.org.
	Enter the Uniprot ID ``P53\_HUMAN'' in the search field labelled
	``Enter a uniprot identifier or accession number''. The page should
	autocomplete/suggest the protein ``P53\_HUMAN / P04637 (Homo
	sapiens)''. Click on this entry to confirm that we want to search for
	SLiM data for this protein. Click on ``Submit'' to view the motif
	instance data for p53. (Fig. TP53-BP1-1)

	\sdesc{The autocompletion mechanism queries uniprot.org for protein
		identifier; if it succeeds, then additional information from
		uniprot will be used to pre-populate the filter boxes. In this
		example, P53\_HUMAN is recognized as a Human protein, and so
		``Homo sapiens'' is automatically filled in the ``Taxonomic
		Context'' field. Also, P53 has been annotated (by Uniprot) to
		be localized to nucleus, cytosol, endoplasmic reticulum and
		mitochondrion, so these are also automatically applied as
		search criteria. The motif cutoff of ``100'' is a sufficiently
		high (lenient) threshold to allow all other detected motifs to
		be shown.}

\item Select the search criteria (optional). It is possible to limit the
	results by ``cell compartment'', ``taxonomic context'' or by changing
	the ``motif probability cutoff''. To restrict the search to include
	SLiM's that are active in certain cellular compartments, select one or
	more from the list (use the ``control'' key to select more than one
	option). It is also possible to select a ``taxonomic context'' to
	restrict the search to SLiMs from certain species. Start typing a
	species name in the ``taxonomic context'' input field to get an
	auto-completed list of species to select from. Additionaly, a ``Motif
	probability cutoff'' can be used to only retain ELM classes whose
	pattern probability is below the given value. For the current protocol,
	leave all of these at their default values: ``not specified'', ``100''
	and no ``taxonomic context''

TODO: Repeat search using stringent filters (homo sapiens, nucleus,
0.01)

%
% Subsection: Interpreting results: graphical summary
%

\subsection{Interpreting the prediction results: Graphical Summary}
\label{subsec:predicting_p53_graphical_summary}

\item Click ``submit'' to start the searching for motifs. You will be brought
	to an intermediate page indicating that your results are being
	processed, and you should be redirected to the final results page
	within a minute. You can bookmark this page: The results are stored for
	a week.

	\sdesc{The Results are summarized in the first figure on the results
		page (see figure TP53-BP1-2). The graphical summary shows the
		results generated by the ELM prediction pipeline, combined with
		additional filters and information from external resources. The
		visualization should help you interpreting the results and to
		assess whether or not a motif is present in a sequence, as well
		as how likely it is to be functional based on its structural
		context and evolutionary conservation. Motif instances which
		are manually annotated in the database appear as red (TP) or
		yellow (FP) ovals in the graphic. Blue/gray squares represent
		predicted motif occurrences.}

\begin{figure}[h!]
	\centering
	\includegraphics[width=\textwidth]{Figures/predicting_p53/elm_results_summary.png} 
	\caption{
	\textbf{Figure TP53-BP1-2}
	The graphical results summary of the ELM Prediction pipeline for
	``P53\_HUMAN''. Note that not all motif detections are shown (the image
	is truncated at the bottom). The top five rows show a set of structural
	features. Annotated and predicted motifs are shown as differently
	colored ovals/boxes.
	}
	\label{fig:predictiing_p53_results_summary}
\end{figure}

\item The first row contains phosphorylation sites as retrieved from
	Phospho.ELM (\cite{21062810}), and whether the phosphorylated amino
	acid is a serine, threonine or tyrosine. Phospho.ELM is a database of
	manually annotated phosphorylation sites obtained from scientific
	publications from low and high-throughput experiments. You can follow
	the link to Phospho.ELM by clicking on the phosphorylation site in the
	image to get more information on individual phosphorylation sites.

	\sdesc{Phosphorylation sites are only available when the search is
		performed with a protein accession (eg. \emph{not} with FASTA
		sequence alone) in step XXX and there is relevant information
		annotated in the Phospho.ELM database. Phosphorylation sites
		are relevant to interpret ELM motif predictions when the
		predicted motif requires to be phosphorylated (as in several
		docking and ligand binding motifs) and naturally, for the
		prediction of phosphorylation motifs.}

\item The second row shows SMART and Pfam domains detected by the SMART
	database (\cite{9600884},\cite{25300481}, \cite{9600884}). Hover the
	mouse over these domains to see their names and exact start and end
	positions.

	\sdesc{ In order to be functional SLiMs need to be accessble, and
		therefore they are usually not found within globular domains
		and structured regions (\cite{21909575}). Any SLiMs detected by
		the ELM prediction pipeline are less likely to be functional,
		and are indicated with a red background (see also the
		``structural filter'' described in step XXX). }

\item The third row shows globular and disordered regions in the sequence as
	predicted by GlobPlot (\cite{12824398}). The fourth and fifth row
	contains results from IUPred (\cite{15955779}), another predictor of
	disordered protein regions. Protein segments with an IUPred score above
	0.5 considered to be disordered.

	\sdesc{SLiMs are typically only functional when found in intrinsically
		disordered regions. Any motif occurrence detected by the ELM
		prediction pipeline that falls within disordered regions are
		more likely to be functional.}

\item The 5th row contains information on secondary structure. The secondary
	structure is predicted using a pipeline mapping motif occurrence onto
	high quality reference domain structures (\cite{19852836}). Check the
	graphical representation, if the output of the secondary structure
	filter and the disorder predictors agree with respect to wihch parts of
	the sequence are considered structured and which disordered.

\item The remainder of the figure (below ``secondary structure'' output)
	displays predicted and annotated motif instances, overlayed by the
	structural context from rows 2 and 3 (SMART domains and GlobPlot). A
	blue square indicates a single motif occurence, intensity of the color
	indicates the conservation of this sequence in homologous proteins.
	Boxes in gray are motif occurences which have been filtered out by the
	``structure filter''. Boxes that are blue \& gray are neutral (eg.
	residing in structural context, but the secondary structure detected a
	loop region). If the sequence is already present in the ELM database,
	any motif instances that have already been annotated are shown as
	ovals.  Lastly, any motifs detected, which are annotated to be
	functional in homologous sequences, are shown as red/blue rectangles.

TODO: EXPLAIN / SHOW ANNOTATED INSTANCES

	\sdesc{ In the case that not enough homologous sequences were detected
		to build an alignment, no conservation score can be calculated.
		Therefore all of the motif occurences will be shown in a
		uniform shade of blue. }

\item Place the cursor over the blue box for motif occurence ``MOD\_PLK'' at
	position 6-12. This motif is in a disordered region, and has not been
	filtered out by the structural filter. However, its conservation score
	is very low: 0.16, indicating it is not conserved in homologous
	proteins.

	\sdesc{ The confidence score is based on how conserved the sequence is
		across a set of homolous proteins from other sequences. An full
		description of the method can be found in \cite{18460207}.}

\item Mouse over a gray rectangle (indicating motifs which have been filtered
	out) to find out why this hit was filtered out. It shows scores for all
	of the individual criteria used by the secondary structure filter: The
	name of the domain, the \emph{accessibility score} , \emph{secondary
	structure score}, \emph{combined total score}, and the associated
	\emph{total score P-value} (\cite{19852836}).

\begin{figure}[h!]
	\centering
	\includegraphics[width=\textwidth]{Figures/predicting_p53/elm_results_alignments_filtering_domains.png}
	\caption{
	\textbf{Figure BACT-BP-3:}
	This section of the results contains
	additional details of alignment of homologous proteins, filtering
	results and globular domains.
	}
	\label{predicting_p53_elm_results_alignemnt_filtering_domains}
\end{figure}

TODO: INSERT/CHANGE FIGURE/NAME

\item Scroll down to below the results graphic to find additional information
	on the ELM Predction pipeline's results (figure BACT-BP-3).  The first
	section contains links to download or view the multiple sequence
	alignments of homologous proteins used to calculate the conservation
	score. Click on the link ``Click here to enable the multiple sequence
	alignment viewer'' to open the alignment in Jalview (note: this
	requires the Java browser plugin, which might not be available on some
	browsers). Alternatively you can also download the ``alignment'',
	``conservation features'' and ``phosphosite features'' files separately
	to view on a desktop (non-browser) installation of Jalview
	(\cite{19151095}).

	\sdesc{ The search for possible homologs is performed against the
		UniRef90 database, a dataset of protein sequences with less
		than 90 percent identity between any two of them
		(\cite{17379688}). It is also possible that the BLAST results
		are not finished when the results page is shown: We suggest to
		refresh the page if you see the message ``Either not enough
		data available to calculate a sequence alignment or the
		calculations haven't finished yet''. In some cases it is also
		possible that no homologs will be detected. If you have
		refreshed the page after waiting for more than 3 minutes, this
		is most likely the case.}

\item Scroll down to the section titled ``Filtering Summary'' to view some
	statistics about how many motifs and instances were filtered out
	(figure TP53-BP1-2). The first two lines contain information on whether
	and which filters were applied in step XXX of this protocol. The next
	two lines (SMART \& Structural score) show how many motifs and
	instances were removed by the SMART and Secondary structure filters.
	The ``Retained by'' section shows how many motif hits were not filtered
	out by the ``Smart'' or ``Structural Score'' filter. In this example a
	total of XXX instances (of XXX different motifs were identified), of
	which XXX instances (and XXX motifs) were filtered out as they occured
	in a SMART domain.

	\sdesc{Note that the graphical summary above does not contain sequences
		filtered out by the ``cell compartment'' and ``taxonomic
		context'' filters (in step XXX). However those filtered out by
		the SMART and Structural scores are shown in the graphic above
		(as gray rectangles).  If any ``cell compartment'' or
		``taxonomic context'' filters are selected in step XXX, the
		number of motifs and instances are also shown in this table.
	}

\begin{figure}[h!]
	\centering
	\includegraphics[width=\textwidth]{Figures/predicting_p53/elm_results_known.png} 
	\caption{
	\textbf{Figure TP53-BP1-3}
	}
	\label{predicting_p53_elm_results_known}
\end{figure}

\item On the results page, scroll down to the heading: ``The ELMs in the
	following table are known instances annotated from the literature''
	(Fig TP53-BP1-3). This table has details of SLiMs which have been
	manually annotated in the ELM database. The columns show each motif
	name, the sequence(s) that matched the motif as well as their starting
	and ending positions and the logic of the annotation followed by a
	short description of each motif, to which cell compartments its has
	been associated, and finally the regular expression of the motif.

	\sdesc{The ``Logic'' column indicates whether this motif is an example
		of a functional (True Positive, TP) or non-functional (False
		Positive, FP) motif. A TP instance is an instance annotated
		with experimental evidence showing this instance to be
		functional, whereas a FP is an instance with experimental
		evidence hinting at a function, but after careful inspection
		our annotators believe this instance to be non-functional.
		There are only rare cases of a true negative (TN) instance,
		which is an annotated instance where experiments have shown it
		to be non-functional.}

TODO: INSERT/CHANGE FIGURE/NAME

\item Scroll down to the section with the header ``Globular domains/ TM domains
	and signal peptide detected by the SMART server'' (Figure BACT-BP-3).
	This section contains information on which domains were detected by the
	SMART server, and their positions. Clicking on their names will bring
	you to the SMART entry for that domain on the SMART homepage.

TODO: INSERT/CHANGE FIGURE/NAME

\begin{figure}[h!]
	\centering
	\includegraphics[width=\textwidth]{Figures/predicting_p53/elm_results_motifs.png}
	\caption{
	\textbf{Figure BACT-BP-7:} This table contains the list of motifs
	detected in the sequence (only the top part of the table is shown).
	}
	\label{predicting_p53_elm_results_motifs}
\end{figure}

TODO: INSERT/CHANGE FIGURE/NAME

\item Scroll further down to the section title ``Results of ELM motif search
	after globular domain filtering, structural filtering and context
	filtering'' to obtain an overview of all of the motifs and motif
	instances detected (Figure BACT-BP-7). Each row also contains
	information on the Motif name, the matching peptide sequence and its
	position. Additional information is shown about the ELM, cell
	compartment and its regular expression. If the motif was detected in a
	homologue, the column called ``PHI-Blast Instance mapping'' contains
	links to the Sequence alignment of the homologous protein, and a
	summary of the ELM instance mapper output. If a motif instance has been
	filtered out due to Structural criteria (SMART or Structure), this
	column contains a link to a page with details on how individual
	criteria that make up this filter. The last column contains information
	on the Probability filter: the probability reflects the chance to
	observe this motif in any random amino acid sequence.  

TODO: INSERT/CHANGE FIGURE/NAME

\begin{figure}[h!]
\centering
	\includegraphics[width=\textwidth]{Figures/predicting_p53/elm_results_motifs_filtered.png}
	\caption{
	\textbf{Figure BACT-BP-8:}
	This table contains the list of motifs detected in the sequence (only
	the top part of the table is shown) which were excluded due to
	structural filters.
	}
	\label{predicting_p53_elm_results_motifs_filtered}
\end{figure}

TODO: INSERT/CHANGE FIGURE/NAME

\item Scroll further down to the heading ``List of excluded ELMs falling inside
	SMART/PFAM domains and/or scoring poorly with the structural filter (if
	applicable).'' (Figure BACT-BP-8). This table is (almost) identical to
	the one above, but shows motif instances which were rejected by the
	Structural filter or SMART filter.

\end{enumerate}

\section{(Basic) Predicting ELMs in proteins}
\label{sec:predicting_p53}

One of the most useful (and used) features in ELM is the ability to
detect motifs in proteins and sequences. Given a protein's amino acid
sequence, the ``EML Predictions'' pipeline searches for occurrences of
each motif class using regular expressions, applies a set of filters to
remove false positives and creates a diagram to visualize resulting
set of putative motifs.

In this protocol we will be viewing the manually annotated data of a
typical protein, using p53 (Uniprot ID: \uniprot{P53\_HUMAN}/P04637) as an
example. We will cover how to find the manually annotated motifs and instances,
and how to find the motif instances, the references used to annotate each
instance, the experimental protocols used, and additional information including
relationships to biological pathways (KEGG), diseases (OMIM) and molecular
switches (switches.ELM).

%
% Subsection: Necessary Resources
%
\subsection{Necessary Resources}
\subsubsection{Software \& Hardware}
A modern browser such as Firefox, Chrome, or Safari. ELM is best viewed
on a laptop or desktop computer, although tablets and smartphones will
also work.



\begin{enumerate}

%
% Subsection:Predicting ELM insstances
%

\subsection{Predicting ELM instances: Input form}
\label{subsec:predicting_p53_input}

\begin{figure}[h!]
	\centering
	\includegraphics[width=\textwidth]{Figures/predicting_p53/elm_search.png} 
	\caption{
	The ELM input page for predicting motifs in a protein.
	}
	\label{fig:predicting_p53_elm_search}
\end{figure}

\item Open a browser, and navigate to the ELM homepage: http://elm.eu.org.
	Enter the Uniprot ID \uniprot{P53\_HUMAN} in the search field labelled
	``Enter a uniprot identifier or accession number''. The page should
	autocomplete/suggest the protein ``\uniprot{P53\_HUMAN} / P04637 (Homo
	sapiens)'' (Fig. \ref{fig:predicting_p53_elm_search}).
	Click on this entry to confirm that we want to search for
	motifs in this protein. Click on \button{Submit} to submit the query to
	the server.

	\sdesc{The autocompletion mechanism queries Uniprot for protein
		identifier; if it succeeds, then additional information from
		Uniprot will be used to pre-populate the filter boxes. In this
		example, \uniprot{P53\_HUMAN} is recognized as a Human protein,
		and so ``Homo sapiens'' is automatically filled in the
		``Taxonomic Context'' field. Also, P53 has been annotated (by
		Uniprot) to be localized to nucleus, cytosol, endoplasmic
		reticulum and mitochondrion, so these are also automatically
		applied as search criteria. The motif cutoff of ``100'' is a
		sufficiently high (lenient) threshold to allow all other
		detected motifs to be shown.}

\item Select the search criteria (optional). It is possible to limit the
	results by ``cell compartment'', ``taxonomic context'' or by changing
	the ``motif probability cutoff''. To restrict the search to include
	motifs that are active in certain cellular compartments, select one or
	more from the list (use the ``control'' key to select more than one
	option). It is also possible to select a ``taxonomic context'' to
	restrict the search to motifs from certain species. Start typing a
	species name in the ``taxonomic context'' input field to get an
	auto-completed list of species to select from. Additionally, a ``Motif
	probability cutoff'' can be used to only retain ELM classes whose
	pattern probability is below the given value. For the current protocol,
	leave all of these at their default values: ``not specified'', ``100''
	and no ``taxonomic context''

TODO: Repeat search using stringent filters (homo sapiens, nucleus,
0.01) Do we want to do this? - Marc

%
% Subsection: Interpreting results: graphical summary
%

\subsection{Interpreting the prediction results: Graphical Summary}
\label{subsec:predicting_p53_graphical_summary}

\item Click \button{submit} to start searching for motifs. You will be brought
	to an intermediate page indicating that your results are being
	processed, and should be redirected to the final results page
	within a minute. You can bookmark this page: The results are stored for
	a week.

	\sdesc{The Results are summarized in the first figure on the results
		page (see figure \ref{fig:predicting_p53_results_summary}).
		The graphical summary shows the
		results generated by the ELM prediction pipeline, combined with
		additional filters and information from external resources. The
		visualization should help you interpreting the results and to
		assess whether or not a motif is present in a sequence, as well
		as how likely it is to be functional based on its structural
		context and evolutionary conservation. Motif instances which
		are manually annotated in the database appear as red (TP) or
		yellow (FP) ovals in the graphic. Blue/gray squares represent
		predicted motif occurrences.}

\begin{figure}[h!]
	\centering
	\includegraphics[width=\textwidth]{Figures/predicting_p53/elm_results_summary.png} 
	\caption{
	The graphical results summary of the ELM Prediction pipeline for
	\uniprot{P53\_HUMAN}. Note that not all motif detections are shown (the
	image is truncated at the bottom). The top five rows show a set of
	structural features. Annotated and predicted motifs are shown as
	differently colored ovals/boxes. The info screens for two motifs are
	shown: \motif{CLV\_C14\_Caspase3-7} and \motif{CLV\_PCSK\_SKI1\_1}.
	}
	\label{fig:predicting_p53_results_summary}
\end{figure}

\item The first row contains phosphorylation sites as retrieved from
	phospho.ELM \cite{21062810}, and whether the phosphorylated amino
	acid is a serine, threonine or tyrosine. Phospho.ELM is a database of
	manually annotated phosphorylation sites obtained from scientific
	publications from low and high-throughput experiments. You can follow
	the link to phospho.ELM by clicking on the phosphorylation site in the
	image to get more information on individual phosphorylation sites.

	\sdesc{Phosphorylation sites are only available when the search is
		performed with a protein accession (eg. \emph{not} with a FASTA
		sequence alone) in step 1 and there is relevant information
		annotated in the phospho.ELM database. Phosphorylation sites
		are relevant to interpret ELM motif predictions when the
		predicted motif requires to be phosphorylated (as in several
		docking and ligand binding motifs) and for predicting 
		phosphorylation motifs.}

\item The second row shows SMART and Pfam domains detected by the SMART
	database \cite{9600884, 25300481, 9600884}
	(Fig. \ref{fig:predicting_p53_results_summary}). Hover the
	mouse over these domains to see their names and exact start and end
	positions.

	\sdesc{ In order to be functional motifs to be accessible, and
		therefore they are usually not found within globular domains
		and structured regions (\cite{21909575}). Any motifs detected
		by the ELM prediction pipeline inside of a smart domain are
		less likely to be functional, and are shown as a gray box
		background (see also the ``structural filter'' described in
		step XXX). }

\item The third row shows globular and disordered regions in the
	sequence as predicted by GlobPlot (\cite{12824398}). The fourth
	and fifth rows
	contain results from IUPred (\cite{15955779}), another
	predictor of disordered protein regions. Protein segments with
	an IUPred score above 0.5 are considered to be disordered.

	\sdesc{Motifs are typically only functional when found in intrinsically
		disordered regions. Any motif occurrence detected by the ELM
		prediction pipeline that falls within disordered regions are
		more likely to be functional.}

\item The 5th row (Fig. \ref{fig:predicting_p53_results_summary}) contains
	information on secondary structure. The secondary structure is
	predicted using a pipeline mapping motif occurrence onto high quality
	reference domain structures \cite{19852836}. Check the graphical
	representation, and if the output of the secondary structure filter and
	the disorder predictors agree with respect to which parts of the
	sequence are considered structured and which disordered.

\item The remainder of the figure (below ``secondary structure'' output)
	displays predicted and annotated motif instances, overlayed with the
	structural context from rows 2 and 3 (SMART domains and GlobPlot). A
	blue square indicates a single motif occurrence, and intensity of the
	color indicates the conservation of this sequence across a group of in
	homologous proteins.
	Boxes in gray are motif occurrences which have been filtered out by the
	structure filter. Boxes that are blue \& gray are neutral (
	residing in structural context, but the secondary structure detected a
	loop region). If the sequence is already present in the ELM database,
	any motif instances that have already been annotated are shown as
	ovals. Lastly, any motifs detected which are annotated to be
	functional in homologous sequences, are shown as red \& blue
	rectangles.

	\sdesc{ In the case that not enough homologous sequences were detected
		to build an alignment, no conservation score can be calculated.
		Therefore all of the motif occurrences will be shown in a
		uniform shade of blue. }

TODO: EXPLAIN / SHOW ANNOTATED INSTANCES Marc: Use mouse over in this figure.

\item Place the cursor over the blue box for motif occurrence 
	\motif{CLV\_C14\_Caspase3-7} at the end of the sequence (
	position 388-392). This will trigger the green and yellow 
	information screen shown on the top right in Fig.
	\ref{fig:predicting_p53_results_summary}.
	This motif is in a disordered region, and has not been
	filtered out by the structural filter. Also, its conservation score
	of 0.910 is very high, indicating that this motif is highly conserved.

	\sdesc{The confidence score is based on how conserved the sequence is
		across a set of homologous proteins from other sequences. An
		full description of the method can be found in \cite{18460207}.
		The higher the conservation score (max. 1), the more conserved
		the motif's sequence is, and the more likely it is a functional
		motif for this prediction.}

\item Place the cursor over the blue \& gray rectangle for motif
	\motif{CLV\_PCSK\_SKI1\_1} at position 120-124, a motif 
	which was flagged as ``neutral'' by the ELM prediction pipeline.
	This will trigger the information screen (with the pink header) shown
	in Fig. \ref{fig:predicting_p53_results_summary} to appear.
	This motif resides inside of the P53 Pfam domain, and thus has been
	subjected to ``structural filtering''. However, the secondary structure
	prediction suggests this motif occurs within the looped regioun of this
	domain, so may be accessible.

	\sdesc{The information screen pop-up 
	shows scores for all of the individual criteria used by the secondary
	structure filter: The name of the domain, the \emph{accessibility
	score} , \emph{secondary structure score}, \emph{combined total score},
	and the associated \emph{total score P-value} \cite{19852836}.}

\begin{figure}[h!]
	\centering
	\includegraphics[width=\textwidth]{Figures/predicting_p53/elm_results_alignments_filtering_domains.png}
	\caption{
	This section of the results contains additional details on the
	homologue alignments used to calculate the conservation score,
	filtering results and globular domains.
	}
	\label{fig:predicting_p53_elm_results_alignemnt_filtering_domains}
\end{figure}

\item Scroll down to below the results graphic to find additional information
	on the ELM prediction pipeline's results
	(Fig. \ref{fig:predicting_p53_elm_results_alignemnt_filtering_domains}).
	The first
	section contains links to download or view the multiple sequence
	alignments of homologous proteins used to calculate the conservation
	score. Click on the link ``Click here to enable the multiple sequence
	alignment viewer'' to open the alignment in Jalview (note: this
	requires the Java browser plugin, which might not be available on some
	browsers). Alternatively you can also download the ``alignment'',
	``conservation features'' and ``phosphosite features'' files separately
	to view on a desktop (non-browser) installation of Jalview
	(\cite{19151095}).

	\sdesc{ The search for possible homologues is performed against the
		UniRef90 database, a dataset of protein sequences with less
		than 90 percent identity between any two of them
		\cite{17379688}. It may occur that the BLAST results
		are not finished when the results page is shown: We suggest to
		refresh the page if you see the message ``Either not enough
		data available to calculate a sequence alignment or the
		calculations haven't finished yet''. In some cases it is also
		possible that no homologues will be detected. If you have
		refreshed the page after waiting for more than 3 minutes, this
		is most likely the case.}

\item Scroll down to the section titled ``Filtering Summary'' to view some
	statistics about how many motifs and instances were filtered out
	(Fig.
	\ref{fig:predicting_p53_elm_results_alignemnt_filtering_domains}).
	The first two lines contain information on whether
	and which filters were applied in step 1 of this protocol.
	In this case 4 motifs (elms) representing 26 instances were filtered
	out as they did not occur in \textit{Homo sapies}. An additional 5
	motifs (representing 11 instances) were filtered out becuase they are
	not annotated to the cell compartments automatically filled in on the
	search page (Step 1).
	The next three lines (``SMART'' \& ``Structural score'') show how many
	motifs and instances were not removed by the SMART and Secondary
	structure filters. A total of 42 motifs (representing 106 instances)
	passed the structural filter.

	\sdesc{Note that the graphical summary above does not contain sequences
		filtered out by the ``cell compartment'' and ``taxonomic
		context'' filters. However those filtered out by
		the SMART and Structural scores are shown in the graphic above
		(as gray rectangles).}

\begin{figure}[h!]
	\centering
	\includegraphics[width=\textwidth]{Figures/predicting_p53/elm_results_known.png} 
	\caption{
		The ELM prediction pipeline section displaying the P53 motifs
		that are ``known'', and have been annotated in the ELM
		database.
	}
	\label{fig:predicting_p53_elm_results_known}
\end{figure}

\item Scroll down to the section with the header ``Globular domains/ TM domains
	and signal peptide detected by the SMART server''
	(Fig. \ref{fig:predicting_p53_elm_results_alignemnt_filtering_domains}).
	This section contains information on which domains were detected by the
	SMART server, and their positions. Clicking on their names will bring
	you to the entry for that domain on the SMART or Pfam homepage.
	In this case the only domains detected is the ``P53'' Pfam domain.

\item On the results page, scroll down to the heading: ``The ELMs in the
	following table are known instances annotated from the literature''
	(\ref{fig:predicting_p53_elm_results_known}).
	This table has details of the motifs and instances which have been
	manually annotated in the ELM database. The columns show each motif
	name, the sequence(s) that matched the motif as well as their starting
	and ending positions and the logic of the annotation followed by a
	short description of each motif, to which cell compartments its has
	been associated, and finally the regular expression of the motif.

\begin{figure}[h!]
	\centering
	\includegraphics[width=\textwidth]{Figures/predicting_p53/elm_results_motifs.png}
	\caption{
	This table contains the list of motifs detected in the sequence (only
	the top part of the table is shown). These are predictions in the sense
	that the sequence in present, however its is known whether they are
	\emph{bona-fide} motifs which are biologically functional.
	}
	\label{fig:predicting_p53_elm_results_motifs}
\end{figure}

\item Scroll further down to the section title ``Results of ELM motif search
	after globular domain filtering, structural filtering and context
	filtering'' to obtain an overview of all of the motifs and motif
	instances detected
	(\ref{fig:predicting_p53_elm_results_motifs})
	Each of the rows is a ``predicted'' motif: A sequence matching a
	motif's regular expression has been detected that has also passed the
	``structural filter''.
	Each row displays the motif identified, the matching peptide
	sequence and its position. Additional information is shown about the
	motif, its cell compartment and its regular expression. If the motif
	was detected in a homologue, the column ``PHI-Blast Instance
	mapping'' contains a link to the multiple sequence alignment of the
	homologous proteins. If a motif instance has been filtered out 
	by the ``structural filter'', the ``Structural filter info'' column
	contains a link to a page with details on why.
	The last column contains information on the Probability filter: the
	probability reflects the chance to observe this motif in any random
	amino acid sequence (see section \ref{sec:explore_content})

\begin{figure}[h!]
\centering
	\includegraphics[width=\textwidth]{Figures/predicting_p53/elm_results_motifs_filtered.png}
	\caption{
	This table contains the list of motifs detected in the sequence (only
	the top part of the table is shown) which were excluded by the
	structural filter.
	}
	\label{fig:predicting_p53_elm_results_motifs_filtered}
\end{figure}

\item Scroll further down to the heading ``List of excluded ELMs falling inside
	SMART/Pfam domains and/or scoring poorly with the structural filter (if
	applicable).''
	(Fig.  \ref{fig:predicting_p53_elm_results_motifs_filtered})
	This table is similar to the one described above, but shows motif
	matches which were rejected by the structural filter.
	
\end{enumerate}

%%% Predicting with CV_0974
% \section{(Alternate) Predicting ELMs in novel sequences}
\label{sec:predicting_cv_0974}

TODO: DESCRIBE MOST PROBABLE MOTIF INSTANCES (COMPARED TO FILTERED)

We will use protein \uniprot{CV\_0974} (uniprot ID: Q7NZE8) as an example, a
``probable tyrosine phosphatase'' from \emph{Chromobacterium violaceum}.
This protein is predicted to be a tyrosine phosphates because it has a
``tyrosine phosphatase'' (PTPc) domain.

%
% Subsection: Necessary Resources
%
\subsection{Necessary Resources}
\subsubsection{Software \& Hardware}
A modern browser such as Firefox, Chrome, or Safari. ELM is best viewed
on a laptop or desktop computer, although tablets and smartphones will
also work.



\begin{enumerate}

%
% Subsection: Submitting a query
%
\subsection{Submitting a query to ELM}
\label{subsec:predicting_cv_0974_submitting}

\begin{figure}[h!]
	\centering
	\includegraphics[width=\textwidth]{Figures/predicting_cv_0974/elm_search.png} 
	\caption{
	\textbf{Figure BACT-BP-1:}
	The input query page for finding motifs in ELM. The sequence
	for \emph{C. vilaceum protein} CV\_0974 was used as an example for this
	protocol.
	}
	\label{fig:predicting_cv_0974_search}
\end{figure}

\item Click on the ``ELM Predictions'' button in the menu to access the search
	query page (Fig. \ref{fig:predicting_cv_0974_search}).
	Here you can provide either a protein
	accession (from uniprot) or an amino acid sequence (simply the
	sequence, or a FASTA formatted entry) in which you want to detect
	SLiMs.  Retrieve the FASTA formatted sequence from Uniprot
	(http://www.uniprot.org/uniprot/Q7NZE8.fasta), and enter it into the
	``sequence input text box''.

TODO: MENTION NOT TO USE ``CHROMOBACTERIUM VIOLACEUM'' IN THE ORGANISM
BOX AND WHY

\begin{figure}[h!]
	\centering
	\includegraphics[width=\textwidth]{Figures/predicting_cv_0974/elm_results_summary.png}
	\caption{
	\textbf{Figure BACT-BP-2:}
	The graphical results summary of the ELM Prediction pipeline for
	Probable Tyrosine phosphate (CV\_0974). Note that not all motif
	detections are shown (the image is truncated at the bottom). The top
	five rows show a handfull of structural features. The motif occurence
	are shown as blue boxes, the intensity of which indicates the
	conservation score. See steps XXX to YYY for more information.
	}
	\label{fig:predicting_cv_0974_results_summary}
\end{figure}

\item The Results are summarized in the first figure on the results page
	(see Fig. \ref{fig:predicting_cv_0974_results_summary})
	The Graphical summary shows all of the final and
	intermediate results generated by the ELM Prediction pipeline, and can
	be used infer whether or not a motif is present in a sequence, as well
	as now likely it is to be functional based on its structural context
	and evolutionary conservation.

\item Check the first row to see whether there are for phosphorylation sites
	acid is a serine, threonine or tyrosine. In this case, no
	phosphorylation data could be found in the Phospho.ELM database
	(\cite{21062810}).

\item Check the second row showing SMART and Pfam domains. Hover the mouse over
	these domains to see their names and exact start and end positions.

\item The third row shows globular and disordered regions in the sequence as
	predicted by GlobPlot (\cite{12824398}). The 4th \& 5th rows contain
	results from IUPred (\cite{15955779}), another unstructured region
	prediction tool. Protein segments with an IUpred score above 0.5 are
	95\% likely to be disorered (REF???).  

\item Place the cursor over the blue box for motif occurence
	``DOC\_USP7\_MATH\_1'' at position 129-133. This motif is in a
	disorered region, and has not been filtered out by the structural
	filter. However, its conservation score is extremely low: 0.000,
	indicating it is not conserved in homologous proteins. Place the cursor
	over motif ``DOC\_MAPK\_DCC\_7'' at positions ``334-343''. Despite the
	high conservation score (1.000), this motif is inside the PTPc domain
	(and a Globular regions), and therefore has been filtered out.

TODO: CHECK CONSERVATION FILTER

%
% Subsection: Submitting a query
%
\subsection{Interpreting the prediction results: Additional Information}
\label{subsec:predicting_cv_0974_additional_information}

TODO: DESCRIBE HOW TO INTERPRETE THE PREDICTIONS USING THIS BACTERIAL
EXAMPLE (OF WHICH NOT MUCH IS KNOWN). FOCUS ON HOW ONE SHOULD INTERPRETE
THESE PREDICTIONS (LOOK AT DISORDER/GLOBULARITY, CONSERVATION)

\end{enumerate}

\section{(Alternate) Predicting ELMs in novel sequences}
\label{sec:predicting_cv_0974}

TODO: DESCRIBE MOST PROBABLE MOTIF INSTANCES (COMPARED TO FILTERED)

We will use protein \uniprot{CV\_0974} (uniprot ID: Q7NZE8) as an example, a
``probable tyrosine phosphatase'' from \emph{Chromobacterium violaceum}.
This protein is predicted to be a tyrosine phosphates because it has a
``tyrosine phosphatase'' (PTPc) domain.

%
% Subsection: Necessary Resources
%
\subsection{Necessary Resources}
\subsubsection{Software \& Hardware}
A modern browser such as Firefox, Chrome, or Safari. ELM is best viewed
on a laptop or desktop computer, although tablets and smartphones will
also work.



\begin{enumerate}

%
% Subsection: Submitting a query
%
\subsection{Submitting a query to ELM}
\label{subsec:predicting_cv_0974_submitting}

\begin{figure}[h!]
	\centering
	\includegraphics[width=\textwidth]{Figures/predicting_cv_0974/elm_search.png} 
	\caption{
	\textbf{Figure BACT-BP-1:}
	The input query page for finding motifs in ELM. The sequence
	for \emph{C. vilaceum protein} CV\_0974 was used as an example for this
	protocol.
	}
	\label{fig:predicting_cv_0974_search}
\end{figure}

\item Click on the ``ELM Predictions'' button in the menu to access the search
	query page (Fig. \ref{fig:predicting_cv_0974_search}).
	Here you can provide either a protein
	accession (from uniprot) or an amino acid sequence (simply the
	sequence, or a FASTA formatted entry) in which you want to detect
	SLiMs.  Retrieve the FASTA formatted sequence from Uniprot
	(http://www.uniprot.org/uniprot/Q7NZE8.fasta), and enter it into the
	``sequence input text box''.

TODO: MENTION NOT TO USE ``CHROMOBACTERIUM VIOLACEUM'' IN THE ORGANISM
BOX AND WHY

\begin{figure}[h!]
	\centering
	\includegraphics[width=\textwidth]{Figures/predicting_cv_0974/elm_results_summary.png}
	\caption{
	\textbf{Figure BACT-BP-2:}
	The graphical results summary of the ELM Prediction pipeline for
	Probable Tyrosine phosphate (CV\_0974). Note that not all motif
	detections are shown (the image is truncated at the bottom). The top
	five rows show a handfull of structural features. The motif occurence
	are shown as blue boxes, the intensity of which indicates the
	conservation score. See steps XXX to YYY for more information.
	}
	\label{fig:predicting_cv_0974_results_summary}
\end{figure}

\item The Results are summarized in the first figure on the results page
	(see Fig. \ref{fig:predicting_cv_0974_results_summary})
	The Graphical summary shows all of the final and
	intermediate results generated by the ELM Prediction pipeline, and can
	be used infer whether or not a motif is present in a sequence, as well
	as now likely it is to be functional based on its structural context
	and evolutionary conservation.

\item Check the first row to see whether there are for phosphorylation sites
	acid is a serine, threonine or tyrosine. In this case, no
	phosphorylation data could be found in the Phospho.ELM database
	(\cite{21062810}).

\item Check the second row showing SMART and Pfam domains. Hover the mouse over
	these domains to see their names and exact start and end positions.

\item The third row shows globular and disordered regions in the sequence as
	predicted by GlobPlot (\cite{12824398}). The 4th \& 5th rows contain
	results from IUPred (\cite{15955779}), another unstructured region
	prediction tool. Protein segments with an IUpred score above 0.5 are
	95\% likely to be disorered (REF???).  

\item Place the cursor over the blue box for motif occurence
	``DOC\_USP7\_MATH\_1'' at position 129-133. This motif is in a
	disorered region, and has not been filtered out by the structural
	filter. However, its conservation score is extremely low: 0.000,
	indicating it is not conserved in homologous proteins. Place the cursor
	over motif ``DOC\_MAPK\_DCC\_7'' at positions ``334-343''. Despite the
	high conservation score (1.000), this motif is inside the PTPc domain
	(and a Globular regions), and therefore has been filtered out.

TODO: CHECK CONSERVATION FILTER

%
% Subsection: Submitting a query
%
\subsection{Interpreting the prediction results: Additional Information}
\label{subsec:predicting_cv_0974_additional_information}

TODO: DESCRIBE HOW TO INTERPRETE THE PREDICTIONS USING THIS BACTERIAL
EXAMPLE (OF WHICH NOT MUCH IS KNOWN). FOCUS ON HOW ONE SHOULD INTERPRETE
THESE PREDICTIONS (LOOK AT DISORDER/GLOBULARITY, CONSERVATION)

\end{enumerate}

%%% Predicting via REST API
% \section{(Alternate) Predicting ELMS in sequences using the API}
\label{sec:predicting_REST}

Querying ELM for motifs in a given sequence (as described in
\ref{sec:predicting_p53} and \ref{sec:predicting_cv_0974}), gives you a nice
overview of putative and possibly
annotated motifs in your query protein with a graphical representation
using colors to highlight different regions of the protein sequence (eg.
disordered vs.~globular). It is however difficult to analyse a large set
of protein sequences in this manner. Therefore, the ELM server
provides an interface which you can use to submit your sequence in a
programmatic way. Of course, this way, you won't receive the graphical
output representation, but are limited to textual data representation.

Currently, there exists a single URL (\code{http://elm.eu.org/start\_search/})
to accept such queries. You can choose to either submit a uniprot name
or accession (eg. \code{http://elm.eu.org/start\_search/P53\_HUMAN.tsv}) or
submit your raw sequence
(e.g.\code{http://elm.eu.org/start\_search/MAPRGFSCLLLLTSEIDLPVKRRA'}.)
If the URL ends in `.tsv' then the server assumes you
are using a Uniprot id or accession; if it doesn't, then it assumes you
are using raw sequence. See below for details.

%
% Subsection: Necessary Resources
%
\subsection{Necessary Resources}

\subsubsection{Software}

Ideally use \code{curl} \rurl{https://curl.haxx.se/} on the command line.
This program can be launched from the terminal in any of the major
operating systems: OSX, Windows and Linux. However \code{curl} is
only one of many different ways to access web content programatically,
and we suggest anyone to use which ever program they feel is better
suited for their tasks.

\begin{enumerate}

%
% Subsection: Submitting a query via the REST API
%
\subsection{Submitting a query to ELM via the REST API}
\label{subsec:predicting_REST_submitting}

\begin{figure}[h!]
	\centering
	\includegraphics[width=\textwidth]{Figures/predicting_REST/curl_P53.png}
	\caption{
	The commandline output when \code{curl} is used to
	download all motifs predicted in Human P53. Note that we used a more
	advanced command that \code{curl} alone to make the columns align
	nicely (see text for an explanation).
	}
	\label{fig:predicting_REST_curl_p53}
\end{figure}


\item Use \code{curl} to query ELM for all motifs predicted to occur in Human
	P53 by typing the following into a terminal: `curl
	`http://elm.eu.org/start\_search/P53\_HUMAN.tsv'. Each row represents a
	motif detection, and the first column ``elm\_identifier'' indicates
	which class was identified. The columns ``start'' and ``stop'' show
	that first and last amino acid positions that matched form part of the
	motif.  Column ``is annotated'' is True if this motif has been
	annotated in the database as an (experimentally validated) motif
	instance. The coulmn ``is phiblastmatch'' is True if ?????. The column
	``is filtered'' shows whether or not this motif was rejected by the
	ELM Prediction structure filter. Coulumn ``phibast'' indicates whether
	?????.  The ``topodomfilter'' and ``taxonfilter'' shown whether ?????.
	The last column ``structure'' ?????

	\sdesc{In figure \ref{fig:predicting_REST_curl_p53}
		we use a slightly more
		advanced command to get the output to look nice in the
		terminal. We specified the \code{-s} option to silence all
		\code{curl} output other than the downloaded file, and piped
		the output directly to the
		\code{column} command (this command exists on most Linux and
		OSX machines).}

TODO: Holger: what are the columns that are not described in the above section for?

\begin{figure}[h!]
	\centering
	\includegraphics[width=\textwidth]{Figures/predicting_REST/predictions_query.png} 
	\caption{
	It is possible to send amino acid sequences to the ELM Prediction
	pipeline. In this case we have used
	the curl option \code{-o} to download directly to the file
	\code{query.tsv}, and use a combination of the \code{head} and
	\code{column} commands to display the first 10 rows to the terminal.
	}
	\label{fig:predicting_REST_query}
\end{figure}

\item Use \code{curl} to query ELM via protein sequence by using the URL
	\code{http://elm.eu.org/start\_search/MAPRGFSCLLLLTSEIDLPVKRRA}
	(Fig. \ref{fig:predicting_REST_query}).
	In this case the query is an arbitrary short
	peptide sequence, but this can (of course) contain any sequence you are
	interested in analysing. The output format is exactly the same as in the
	previous step.

	\sdesc{ This way of querying ELM is unfortunately not stable for long
		protein sequences. Different browsers and computers have
		different maximum lengths for URLs, and the excess text is
		often simply ignored. We recommend not using this method for
		sequences longer than 2000 amino acids.}

\end{enumerate}

\section{(Alternate) Predicting ELMS in sequences using the API}
\label{sec:predicting_REST}

Querying ELM for motifs in a given sequence (as described in
\ref{sec:predicting_p53} and \ref{sec:predicting_cv_0974}), gives you a nice
overview of putative and possibly
annotated motifs in your query protein with a graphical representation
using colors to highlight different regions of the protein sequence (eg.
disordered vs.~globular). It is however difficult to analyse a large set
of protein sequences in this manner. Therefore, the ELM server
provides an interface which you can use to submit your sequence in a
programmatic way. Of course, this way, you won't receive the graphical
output representation, but are limited to textual data representation.

Currently, there exists a single URL (\code{http://elm.eu.org/start\_search/})
to accept such queries. You can choose to either submit a uniprot name
or accession (eg. \code{http://elm.eu.org/start\_search/P53\_HUMAN.tsv}) or
submit your raw sequence
(e.g.\code{http://elm.eu.org/start\_search/MAPRGFSCLLLLTSEIDLPVKRRA'}.)
If the URL ends in `.tsv' then the server assumes you
are using a Uniprot id or accession; if it doesn't, then it assumes you
are using raw sequence. See below for details.

%
% Subsection: Necessary Resources
%
\subsection{Necessary Resources}
\subsubsection{Software}
Ideally use \code{curl} \rurl{curl.haxx.se/} on the commandline. This
program can be launched from the terminal in any of the major operating
systems: OSX, Windows and Linux. Of course \code{curl} is only one of
many different ways to access web content programatically, and we
suggest anyone to use which ever program they feel is better suited for
their tasks.




\begin{enumerate}

%
% Subsection: Submitting a query via the REST API
%
\subsection{Submitting a query to ELM via the REST API}
\label{subsec:predicting_REST_submitting}

\begin{figure}[h!]
	\centering
	\includegraphics[width=\textwidth]{Figures/predicting_REST/curl_P53.png}
	\caption{
	The commandline output when \code{curl} is used to
	download all motifs predicted in Human P53. Note that we used a more
	advanced command that \code{curl} alone to make the columns align
	nicely (see text for an explanation).
	}
	\label{fig:predicting_REST_curl_p53}
\end{figure}


\item Use \code{curl} to query ELM for all motifs predicted to occur in Human
	P53 by typing the following into a terminal: `curl
	`http://elm.eu.org/start\_search/P53\_HUMAN.tsv'. Each row represents a
	motif detection, and the first column ``elm\_identifier'' indicates
	which class was identified. The columns ``start'' and ``stop'' show
	that first and last amino acid positions that matched form part of the
	motif.  Column ``is annotated'' is True if this motif has been
	annotated in the database as an (experimentally validated) motif
	instance. The coulmn ``is phiblastmatch'' is True if ?????. The column
	``is filtered'' shows whether or not this motif was rejected by the
	ELM Prediction structure filter. Coulumn ``phibast'' indicates whether
	?????.  The ``topodomfilter'' and ``taxonfilter'' shown whether ?????.
	The last column ``structure'' ?????

	\sdesc{In figure \ref{fig:predicting_REST_curl_p53}
		we use a slightly more
		advanced command to get the output to look nice in the
		terminal. We specified the \code{-s} option to silence all
		\code{curl} output other than the downloaded file, and piped
		the output directly to the
		\code{column} command (this command exists on most Linux and
		OSX machines).}

TODO: Holger: what are the columns that are not described in the above section for?

\begin{figure}[h!]
	\centering
	\includegraphics[width=\textwidth]{Figures/predicting_REST/predictions_query.png} 
	\caption{
	It is possible to send amino acid sequences to the ELM Prediction
	pipeline. In this case we have used
	the curl option \code{-o} to download directly to the file
	\code{query.tsv}, and use a combination of the \code{head} and
	\code{column} commands to display the first 10 rows to the terminal.
	}
	\label{fig:predicting_REST_query}
\end{figure}

\item Use \code{curl} to query ELM via protein sequence by using the URL
	\code{http://elm.eu.org/start\_search/MAPRGFSCLLLLTSEIDLPVKRRA}
	(Fig. \ref{fig:predicting_REST_query}).
	In this case the query is an arbitrary short
	peptide sequence, but this can (of course) contain any sequence you are
	interested in analysing. The output format is exactly the same as in the
	previous step.

	\sdesc{ This way of querying ELM is unfortunately not stable for long
		protein sequences. Different browsers and computers have
		different maximum lengths for URLs, and the excess text is
		often simply ignored. We recommend not using this method for
		sequences longer than 2000 amino acids.}

\end{enumerate}

%%% Searching via REST API
% \section{(Alternate) Searching the ELM database using the REST API}
\label{sec:search_REST}

Many researchers are interested in large-scale analyses rather than
information about individual protein sequences. To this end, individual
queries to the ELM webserver with a single protein id at a time, are not
practical.

For this reason, as much information as possible is made available via a
REST interface (\cite{Fielding2002}). This allows the user to interact
with the ELM database and ELM webserver via scriptable URL requests.
Each request can easily be tested in the browser before it is being
automated in a script.

In this section we will explore the various ways in which data can
downloaded both in using the browser as well as via the commandline.

%
% Subsection: Necessary Resources
%
\subsection{Necessary Resources}\label{necessary-resources-3}

\subsubsection{Software}\label{software-1}

Ideally use \texttt{curl} https://curl.haxx.se/ on the commandline. This
program can be launched from the terminal in any of the major operating
systems: OSX, Windows and Linux. Of course \texttt{curl} is only one of
many different ways to access web content programatically, and we
suggest anyone to use which ever program they feel is better suited for
their tasks.

\begin{enumerate}

%
% Subsection: Necessary Resources
%
\subsection{Downloading all ELM
classes}\label{downloading-all-elm-classes}

\begin{figure}[h!]
	\centering
	\includegraphics[width=\textwidth]{Figures/search_REST/elm_downloads_html.png}
	\caption{
	\textbf{Figure ELM-Downloads:} The ELM downloads page, which holds
	information about the different types of data (such as ``Classes'',
	``Instances'', etc; see menu to the right) that can be obtained from the
	server. The orange boxes are clickable links, the URL following them are
	used to highlight the URL scheme used by the server (bold font denotes
	specifics used in the examples such as query terms, or formats).
	}
\end{figure}

\item Direct your browser to the URL `http://elm.eu.org/downloads' or select
	`ELM Downloads' from the main Menu (Figure ELM-Downloads) This webpage
	contains links and descriptions on how to download ELM data in text
	format. The datasets are split into several smaller collections (for
	example ``Classes'', ``Instances'', etc). Each table contains links (in
	orange) to download the data in various formats.

	\sdesc{ Each table also shows the `last modified date' indicating when
		the data was last updated. This is useful if you want to know
		when to update your local data with the most up to date ELM
		data. }

\item Click on the first orange `html' link in the table ``Classes'' to
	navigate to the following URL:
	`http://elm.eu.org/elms/elm\_index.html'. This page shows all of the
	annotated ELM classes in the database. This page is the same one as
	shown in Figure \emph{TP53-BP1-classses}

\item Navidate to the folling URL: `http://elm.eu.org/elms.html?q=CSK',
	specifying ``q=CSK'' to limit the list of ELMs to those matching the
	search query ``CSK''. This page is again similar to the one shown in
	Figure \emph{TP53-BP1-classses}, but with less classes.

	\sdesc{ This search result is identical to the result you would obtain
		by doing a ``manual'' search on the ELM Classes page
		(http://elm.eu.org/elms.html). The column descriptions are also
		the same as described in Step XXX in Protocol YYY.}

\begin{figure}[h!]
	\centering
	\includegraphics[width=\textwidth]{Figures/search_REST/elm_curl_classes_CSK.png}
	\caption{
	\textbf{Figure ELM-Curl-Classes}: Screenshot of a terminal window using
	\texttt{curl} to download all ELM classes matching the term `CSK'.
	}
	\label{fig:search_REST}
\end{figure}

\item Open the following URL: `http://elm.eu.org/elms.tsv?q=CSK' to download a
	list of classes that match the search query ``CSK'' (as in the previous
	step) in the ``tab separated values'' format. By exchanging the `.html'
	part of the url with `.tsv', we ask the webserver to give us the data
	in TSV (tab-separated values) format.

	\sdesc{ Depending on which browser you are using, the file may open
		directly in your browser, or you may be prompted to download
		the file or save it to a separate location. In the latter two
		cases you can open the downloaded file using a (plain) text
		file viewer, or possible a spreadsheet viewer (such as
		Microsoft Excel or LibreOffice Calc).
	}

\item Type the follwing command into a command line terminal to
	download the same data from the previous step directly into the
	terminal: \texttt{curl
	'http://elm.eu.org/elms/elms\_index.tsv?q=CSK'}.  The output
	should look similar to \emph{Figure ELM-Curl-Classes}. The
	column names are still the same ones as shown in the
	\emph{classes} table in Figure
	\emph{BACT-AP2-Elm-classes-downloads}.

	\sdesc{ Use the curl option \texttt{-o} to save the results directly to
		a file.  For example: \texttt{curl -o classes.tsv
		'http://elm.eu.org/elms/elms\_index.tsv?q=CSK'} will save the
		data to a file called \emph{classes.tsv}.}

\item To download a list of all motif instances detected in Human P53, type the
	followin command into a terminal: \texttt{curl
	'http://elm.eu.org/instances.gff?q=p53\_human'}. The output should look
	similar to that shown in figure \emph{Figure ELM-Curl-P53}.  The output
	is in the ``General Feature Format''
	(http://www.ensembl.org/info/website/upload/gff.html\#moreinfo), with
	the FASTA formatted sequence appended to the end of the output.  

	\sdesc{ Many other file formats are available for downloading instances
		annotations, including the FASTA, GFF, PIR, or PSI-MI format
		(either XML or MiTab) \cite{24067240}.}

\begin{figure}[h!]
	\centering
	\includegraphics[width=\textwidth]{Figures/search_REST/elm_curl_instances_p53_human.png}
	\caption{
	\textbf{Figure ELM-Curl-Instances-P53}: Screenshot of a terminal window
	using \texttt{curl} to download all ELM instances annotated for sequence
	p53\_human.
	}
	\label{fig:search_REST_instances}
\end{figure}

\item To download a list of all instances matchin th search query ``CLV'' in
	the yellow fever mosquito (Aedes agypti), enter the following command
	into a terminal: `curl
	`http://elm.eu.org/instances.tsv?q=CLV\&taxon=aedes+aegypti'. In
	general any species name can be used, always replacing the ``space''
	with a ``+''. This should return a single instance, the only one
	matching CLV in \emph{A. aegypti}.

\item More data (interactions, domains, methods, etc.) can be downloaded from
	ELM in analogous fashion as shown in the preceeding steps. Take a look
	at the ELM Downloads page (http://elm.eu.org/downloads, Figure
	\emph{BACT-AP2-Elm-downloads}) for an overview of which datasets can be
	downloaded, and what the different possible filters and formats are for
	each dataset.  

\% NOTE: TODO: Mention ELM software license agreement?

\end{enumerate}

\section{(Alternate) Searching the ELM database using the REST API}
\label{sec:search_REST}

Many researchers are interested in large-scale analyses rather than
information about individual protein sequences. To this end, individual
queries to the ELM webserver with a single protein id at a time, are not
practical.

For this reason, as much information as possible is made available via a
REST interface \cite{Fielding2002}. This allows the user to interact
with the ELM database and ELM webserver via scriptable URL requests.
Each request can easily be tested in the browser before it is being
automated in a script.

In this section we will explore the various ways in which data can
downloaded both in using the browser as well as via the commandline.

%
% Subsection: Necessary Resources
%
\subsection{Necessary Resources}
\subsubsection{Software}
Ideally use \code{curl} \rurl{curl.haxx.se/} on the commandline. This
program can be launched from the terminal in any of the major operating
systems: OSX, Windows and Linux. Of course \code{curl} is only one of
many different ways to access web content programatically, and we
suggest anyone to use which ever program they feel is better suited for
their tasks.



\begin{enumerate}

%
% Subsection: Necessary Resources
%
\subsection{Downloading all ELM
classes}\label{downloading-all-elm-classes}

\begin{figure}[h!]
	\centering
	\includegraphics[width=\textwidth]{Figures/search_REST/elm_downloads_html.png}
	\caption{
	The ELM downloads page, which holds
	information about the different types of data (such as ``Classes'',
	``Instances'', etc; see menu to the right) that can be obtained from the
	server. The orange boxes are clickable links, the URL following them are
	used to highlight the URL scheme used by the server (bold font denotes
	specifics used in the examples such as query terms, or formats).
	}
	\label{fig:search_REST_downloads}
\end{figure}

\item Direct your browser to the URL \rurl{elm.eu.org/downloads} or select
	`ELM Downloads' from the main menu
	(Fig. \ref{fig:search_REST_downloads}).
	This page contains links and descriptions on how to download ELM data
	in text format. The datasets are split into several smaller collections
	(for example ``Classes'', ``Instances'', etc). Each table contains
	links (in orange) to download the data in various formats.

	\sdesc{Each table also shows the `last modified date' indicating when
		the data was last updated. This is useful if you want to know
		when to update your local data with the most up to date ELM
		data. }

\item Click on the first orange `html' link in the table ``Classes'' to
	navigate to the following URL:
	\rurl{elm.eu.org/elms/elm\_index.html}. This page shows all of the
	annotated ELM classes in the database. This page is the same one as
	shown in figure \ref{fig:explore_content_elms}.

\item Navidate to the folling URL: \rurl{elm.eu.org/elms.html?q=CSK'}
	specifying \code{q=CSK} to limit the list of ELMs to those matching the
	search query ``CSK''. This page is again similar to the one shown in
	figure \ref{fig:explore_content_elms}, but with less classes.

	\sdesc{ This search result is identical to the result you would obtain
		by doing a ``manual'' search on the ELM Classes page
		described in step 3 of \ref{sec:explore_content}
		(Fig. \ref{fig:explore_content_elms}).}


\item Open the following URL: \rurl{elm.eu.org/elms.tsv?q=CSK} to download a
	list of classes that match the search query ``CSK'' (as in the previous
	step) in the ``tab separated values'' format. By exchanging the
	\fileformat{.html} part of the url with \fileformat{.tsv}', we ask the
	webserver to give us the data in ``tab-separated values'' format.

	\sdesc{ Depending on which browser you are using, the file may open
		directly in your browser, or you may be prompted to download
		the file or save it to a separate location. In the latter two
		cases you can open the downloaded file using a (plain) text
		file viewer, or possible a spreadsheet viewer (such as
		Microsoft Excel or LibreOffice Calc).}

\begin{figure}[h!]
	\centering
	\includegraphics[width=\textwidth]{Figures/search_REST/elm_curl_classes_CSK.png}
	\caption{
	A screenshot of a terminal window using
	\code{curl} to download all ELM classes matching the term `CSK'.
	}
	\label{fig:search_REST_curl_csk}
\end{figure}

\item Type the follwing command into a command line terminal to
	download the same data from the previous step directly into the
	terminal:
	\code{curl 'http://elm.eu.org/elms/elms\_index.tsv?q=CSK'}.  The output
	should look similar to figure \ref{fig:search_REST_curl_csk}.
	The column names are the same ones as shown in the
	table in figure \ref{fig:explore_content_elms}.

	\sdesc{ Use the curl option \code{-o} to save the results directly to
		a file.  For example: \code{curl -o classes.tsv
		'http://elm.eu.org/elms/elms\_index.tsv?q=CSK'} will save the
		data to a file called \emph{classes.tsv}.}

\begin{figure}[h!]
	\centering
	\includegraphics[width=\textwidth]{Figures/search_REST/elm_curl_instances_p53_human.png}
	\caption{
	Screenshot of a terminal window using \code{curl} to download all ELM
	instances annotated for sequence p53\_human.
	}
	\label{fig:search_REST_curl_p53}
\end{figure}

\item To download a list of all motif instances detected in Human P53, type the
	followin command into a terminal: \code{curl
	'http://elm.eu.org/instances.gff?q=p53\_human'}. The output should look
	similar to that shown in figure \ref{fig:predicting_REST_curl_p53}. The
	output is in the ``General Feature Format''
	(see \rurl{www.ensembl.org/info/website/upload/gff.html\#moreinfo}),
	with the FASTA formatted sequence appended to the end of the output.  

	\sdesc{ Many other file formats are available for downloading instances
		annotations, including the \fileformat{fasta},
		\fileformat{gff}, \fileformat{pir}, OR PSI-MI format (either
		\fileformat{xml} or \fileformat{MiTab})}


\item To download a list of all instances matchin th search query ``CLV'' in
	the yellow fever mosquito (\textit{Aedes agypti}), enter the following
	command into a terminal:
	\code{curl `http://elm.eu.org/instances.tsv?q=CLV\&taxon=aedes+aegypti'}.
	In general any species name can be used, always replacing the ``space''
	with a ``+''. This should return a single instance, the only one
	matching CLV in \textit{A. aegypti}.

\item More data (interactions, domains, methods, etc.) can be downloaded from
	ELM in analogous fashion as shown in the preceeding steps. Take a look
	at the ELM Downloads page (http://elm.eu.org/downloads, figure
	\ref{fig:search_REST_downloads}) for an overview of which datasets can
	be downloaded, and what the different possible filters and formats are
	for each dataset.  

\end{enumerate}

%%% Commentary 
% \section{Guidelines for Interpreting
Results}\label{guidelines-for-interpreting-results}

\emph{instructions: A brief discussion of the theory and applications of
your}

\emph{notes: Maybe mention how findings are relevant to the lab? For
example: Manually annotated content should be reliable, although one
should look at the `confidence' in the instance annotation. Predictions
are probably trustworthy, but you need to take into account the
`confidence score', and other features like whether its in a domain,
etc\ldots{}}

\section{Commentary:}\label{commentary}

\emph{instructions: A brief discussion of the theory and applications of
your}

\subsection{Background Information}\label{background-information}

\textbf{Im still not sure what's going to happen here}

In order to interpret the data contained in ELM and the results produced
by the ELM prediction tool, it is important to have a basic
understanding of SLiM's and how they are affected by their structural
and biological context.

Taking into account additional information, besides a match to a
sequence pattern defining a SLiM, can greatly narrow the selection of
putative motifs for experimental validation.

SLiM's operate via with interactions with other proteins, typically the
surface of a globular, domain in a protein, although some are known to
bind to disordered regions. As their name suggests, SLiMs are compact,
being composed of a limited number of adjacent amino acids. Most of a
motif's binding specificity however is conferred by only a subset of
these amino acids. Those few residues that directly interact with the
binding partner are evolutionary conserved, although in many cases a
subset of amino acids that share certain properties (such as similar
charge, size or hydrophobicity) are allowed in these hotspot positions.
In the motif positions that contribute little to the interaction, there
are even less constraints, i.e.~a broader range of amino acids is
allowed in these positions (\cite{21909575}). A first consequence of
this degeneracy is that SLiMs co-operatively engage in interactions of
relatively low affinity. Hence these binding events are transient and
reversible, and can be readily modulated, for instance by PTM.

\begin{figure}[h!]
\centering
\includegraphics[width=\textwidth]{Figures/functional_classification_of_SLiMs.png}
\caption{
\textbf{Figure functional\_classification\_of\_SLiMs} For each ELM
class, the functional category to which it belongs is indicated by a
three-letter prefix. Each ELM class is defined by a regular expression.
Peptide sequences in proteins that match the regular expression of a
specific ELM class and that were experimentally validated to be
functional motifs are captured as ELM instances of that class. Degrons
are a specific subtype of enzyme-recruiting docking motifs (see text for
a detailed description).
}
\end{figure}

\subsection{Critical Parameters and
Troubleshooting}\label{critical-parameters-and-troubleshooting}

\emph{instructions: optionally 2 separate sections.}

\section{Internet Resources with
Annotations}\label{internet-resources-with-annotations}

http://www.clustal.org/omega Clustal Omega (\cite{21988835}) is a tool
for the alignment of multiple nucleic acid and protein sequences.

http://www.jalview.org Jalview (\cite{19151095}) is a Java desktop
application (and browser applet) that employs web services for sequence
alignment and visualization.

http://proviz.ucd.ie ProViz (\cite{27085803}) is an interactive protein
exploration tool, which searches several databases for information about
a given query protein. Data relevant to the protein like an alignment of
homologues, linear motifs, post translational modifications, domains,
secondary structure, sequence variations and others are graphically
represented relative to their position in the protein.

\section*{Guidelines for Interpreting
Results}\label{guidelines-for-interpreting-results}

\emph{instructions: A brief discussion of the theory and applications of
your}

\emph{notes: Maybe mention how findings are relevant to the lab? For
example: Manually annotated content should be reliable, although one
should look at the `confidence' in the instance annotation. Predictions
are probably trustworthy, but you need to take into account the
`confidence score', and other features like whether its in a domain,
etc\ldots{}}

\section*{Commentary:}\label{commentary}

\emph{instructions: A brief discussion of the theory and applications of
your}

\subsection*{Background Information}\label{background-information}

In order to interpret the data contained in ELM and the results produced by the
ELM prediction tool, it is important to have a basic understanding of SLiM's
and how they are affected by their structural and biological context. This 
background information summarises the different functionalities of SLiMs, 
describes the degenerate nature of motif sequences, and emphasises the need for 
contextual data for confident SLiM prediction.

\subsubsection*{ELM categorises SLiMs depending on their functionality}

SLiMs mediate different types of interactions, and based on this functionality, 
the ELM classes annotated in the ELM database are grouped into six main ELM 
types (Figure \ref{fig:SLiMclasses}) (\cite{24214962}). They can function as ligand binding sites or 
as sites for post-translational modification (PTM). Some ligand SLiMs are 
recognised by components of the cellular transport machinery and function as 
localisation signals that target proteins to specific sub-cellular compartments 
(TRG type). Other ligand SLiMs are abundantly present in interfaces that mediate 
the assembly of large macromolecular complexes and in highly modular scaffold 
proteins that act as multivalent platforms for protein complex assembly 
(LIG type). Docking motifs are ligand SLiMs that recruit modification enzymes to 
their substrates by binding to a site on the enzyme that is distinct from the 
active site (DOC type). A subset of these, known as degrons, recruit ubiquitin 
ligases, which subsequently polyubiquitylate their substrates and hence target 
them for proteasomal degradation (DEG type). SLiMs that act as sites for PTM can 
be targeted by specific enzymes for the addition or removal of a small chemical 
group (e.g. phosphorylation), a sugar molecule (e.g. glycosylation), a protein 
(e.g. ubiquitylation), or another moiety (e.g. lipidation) (MOD type). Other PTM 
SLiMs mediate proteolytic cleavage by acting as target site for proteolytic 
enzymes (CLV type), or are recognised for structural modification by isomerases 
that catalyse cis-trans isomerisation of the peptide backbone (DOC type) 
\cite{24926813, 24773235}.

\subsubsection*{ELM regular expressions reflect the degenerate nature of SLiMs}

As their name suggests, SLiMs are compact, being composed of a limited number of 
adjacent amino acids. Most of a motif’s binding specificity however is conferred 
by only a subset of these amino acids. Those few residues that directly interact 
with the binding partner are evolutionary conserved, although in many cases a 
subset of amino acids that share certain properties (such as similar charge, 
size or hydrophobicity) are allowed in these hotspot positions. In the motif 
positions that contribute little to the interaction, there are even less 
constraints, i.e. a broader range of amino acids is allowed in these positions 
\cite{21909575}. This sequence flexibility is captured in the regular 
expressions that are defined for each motif class. A first consequence of this 
degeneracy is that SLiMs co-operatively engage in interactions of relatively low 
affinity. Hence these binding events are transient and reversible, and can be 
readily modulated, for instance by PTM. These characteristics make SLiM-based 
interactions ideal mediators of the dynamic processes involved in cell 
signalling \cite{22480932}. Another consequence is that it might take only a few 
or even a single point mutation to generate or disrupt a functional motif in a 
protein. The associated ability to evolve convergently might underlie the 
proliferation of SLiMs and the rewiring of interactomes \cite{26589632,
22346764}. Conversely, several SLiM-associated diseases have been 
characterised to date, for instance Liddle syndrome \cite{15483078}.

\subsubsection*{ELM integrates data to increase the confidence of SLiM prediction*}

Due to their degenerate nature, motif sequences contain only very little 
information, and many short sequences in a proteome will match motif patterns. 
However, most of these matches will not represent functional motifs, and hence, 
when scanning a proteome for putative motifs using only the motif sequence 
patterns will yield a large number of false positive instances, far exceeding 
the number of true motifs. Therefore, reliable motif detection cannot go without 
experimental validation of candidate motifs, using different types of 
experiments and techniques \cite{26581338}. This however does not mean that 
bioinformatics analysis cannot guide researchers towards a subset of candidate 
motifs that have a higher probability to be functional and help rule out those 
candidate motifs that are likely to be false positives. Taking into account 
additional information, besides a match to a sequence pattern defining a SLiM, 
can greatly narrow the selection of putative motifs for experimental validation. 
Additional data for in silico analysis include conservation of the motif 
sequence, the location of the motif within the protein’s structure and its 
accessibility for its binding partner, validated interaction with the binding 
partner, and in-cell co-localisation with the binding partner. The availability 
and usefulness of these additional data for SLiM discovery depends on their 
extensive and correct biocuration. A vast and increasing amount of biological 
data is available in a wide variety of sources, including the literature and 
large-scale datasets. In order to facilitate integration of data, they need to 
be collected, annotated and formatted in central data and knowledge 
repositories. The ELM database provides such a repository for experimentally 
validated linear motif classes and instances. The ELM prediction tool in turn 
relies on annotated data, both from the ELM database and other resources, to 
accurately analyse unknown sequences for candidate motifs and assist researchers 
in selecting the most plausible ones for experimental validation and discard 
likely false positive hits, saving them valuable time and assets 
\cite{22110040}.

\begin{figure}[h!]
\centering
\includegraphics[width=\textwidth]{Figures/Introduction/functional_classification_of_SLiMs.png}
\caption{
\textbf{Figure functional\_classification\_of\_SLiMs} For each ELM class, the
functional category to which it belongs is indicated by a three-letter prefix.
Each ELM class is defined by a regular expression. Peptide sequences in
proteins that match the regular expression of a specific ELM class and that
were experimentally validated to be functional motifs are captured as ELM
instances of that class. Degrons are a specific subtype of enzyme-recruiting
docking motifs (see text for a detailed description).
}
\label{fig:SLiMclasses}
\end{figure}

\subsection*{Critical Parameters and
Troubleshooting}\label{critical-parameters-and-troubleshooting}

\emph{instructions: optionally 2 separate sections.}

\section*{Internet Resources with
Annotations}\label{internet-resources-with-annotations}

http://www.clustal.org/omega Clustal Omega (\cite{21988835}) is a tool
for the alignment of multiple nucleic acid and protein sequences.

http://www.jalview.org Jalview (\cite{19151095}) is a Java desktop
application (and browser applet) that employs web services for sequence
alignment and visualization.

http://proviz.ucd.ie ProViz (\cite{27085803}) is an interactive protein
exploration tool, which searches several databases for information about
a given query protein. Data relevant to the protein like an alignment of
homologues, linear motifs, post translational modifications, domains,
secondary structure, sequence variations and others are graphically
represented relative to their position in the protein.


\bibliography{references}

\end{document}
