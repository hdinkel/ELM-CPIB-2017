\section*{Eukaryotic Linear Motifs on the ELM database}

Marc Gouw$^{1}$,
Hugo Samano$^{1}$,
Kim Van Roey$^{1}$,
Francesca Diella$^{1}$,
Toby Gibson$^{1}$,
Holger Dinkel$^{2}$

$^{1}$ Structural and Computational Biology, European Molecular Biology
Laboratory, Meyerhofstrasse 1, 69117 Heidelberg, Germany

$^{2}$ Leibniz-Institute on Aging -- Fritz Lipmann Institute (FLI),
Beutenbergstrasse 11, 07745 Jena, Germany

\section*{Keywords}

Linear motifs, Bioinformatics, Protein-Protein Interaction, Molecular
switches, Cell regulation

\section*{Significance statement}

\emph{instructions: Provide a 120-word-maximum statement about the
significance of the protocols/topic described in your manuscript. This
should be understandable to undergraduate- educated scientists outside
their field of specialty. The goal is to explain the relevance of the
work in broad context to a broad readership. It will be used in
promotion of the article following publication.}

\begin{abstract}

\emph{instructions: brief overview, no references, max 150 words}

The Eukaryotic Linear Motif (ELM) resource (http://elm.eu.org) is a
manually curated database of short linear motifs (SLiMs). This protocol
explains how to best use this resource and explains how to access the
database content (both manual and scripted access), how to interpret the
output, and how to predict novel putative motifs in any given protein
sequence.

\end{abstract}


